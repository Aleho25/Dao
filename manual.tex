%% This is ConTeXt
%% http://wiki.contextgarden.net/Main_Page
%% Compiling:
%$ context manual.tex

\enableregime[utf]
\mainlanguage[russian]

\setuppapersize[A4]
\setupbodyfont[computer-modern-unicode,10pt,rm]

\definedescription[materialdesc][alternative=serried, width=broad, headstyle=boldslanted]
\definedescription[usagedesc][alternative=serried, width=broad, headstyle=bold]
\definedescription[problemdesc][alternative=serried, width=broad, headstyle=bold]

\setupexternalfigures[directory={images}]

\setupunit[language=russian]

\starttext

\completecontent

\starttitle[title={Введение, которое обычно никто не читает}]

Обычно мало смысла от прочтения введения, там нет никакой полезной информации,
поэтому можно смело перелистывать дальше. А тем, кто читает всё подряд, я
расскажу что это за пособие, зачем оно было написано и прочую не очень важную
информацию.

Много полезного есть в книгах и в интернете, но нет места где были бы собраны в
одном месте:\footnote{Если знаете такое место — поделитесь со мной pavel.serkov@gmail.com}
информация о материалах, зачем они вообще нужны, как с ними работать, как их обрабатывать.
И что бы с картинками, и без высоколобого академизма. Не секрет, что большинство учебников
по материаловедению скучны.

Что вы найдете в пособии:

\startitemize
\item Практически нет страшных математических формул
\item Автор инженер, меньше теории, больше практики
\item Максимально дружелюбно для новичков и гуманитариев
\item Много цветных картинок — почти все фотографии автор делал лично, а не тырил из интернета.
\item Сильный перекос в сторону электроники. Если вам хочется больше узнать про
сорта стали, сплавы цветных металлов — вы будете разочарованы.
\stopitemize

Плюшки данного пособия:

\startitemize
\item Весь текст написан полностью мной, если не указано иное. Я не включал
информацию, в достоверности или актуальности которой я бы сомневался. Поэтому
доля брехни по тексту в среднем ниже, чем в маркетинговых текстах
перепродавцов-поставщиков, но выше, чем в хорошем советском учебнике.
\item Пособие живое, я вношу в него дополнения, исправляю ошибки которые
находят мои читатели. Сверяйтесь с актуальной версией пособия, первоисточник —
мой блог http://serkov.me
\item Большую часть материалов я хотя бы щупал, использовал в своих
конструкциях, а не видел только на картинке.
\item Пособие полностью\footnote{Чтобы быть до конца честным — за исключением одной картинки,
которую пришлось рисовать в чем умел.} подготовлено с использованием OpenSource продуктов
(Linux, GIMP, LibreOffice). Просто из спортивного интереса.
\item Некоторые разделы имеют пункт «Источники» - советы по поиску материалов —
где купить, под какими названиями искать. Конечно, всё можно купить на
Алиэкспресс и на Ebay, поэтому такой вариант не указывается. Пункт может быть
полезен если материал нужен «здесь и сейчас».
\item Для кого будет полезно это пособие прежде всего? Для всех, расширить
кругозор, узнать что то новое для себя. Особенно рекомендую тем, кто хочет
что-либо делать своими руками. Внимательно читайте подписи под картинками,
многие фотографии одновременно относятся к разным разделам.
\stopitemize

Я обеими руками поддерживаю движение Open Source и Open Hardware, считаю, что
обмен знаниями должен быть свободным, это принесет пользу для всех, поэтому
пособие распространяется под лицензией Creative Commons 3.0 BY-NC-SA, что
значит, вы можете делать с ним что угодно, выкладывать, распространять,
модифицировать, соблюдая только три ограничения:

\startitemize
\item Ссылка на меня обязательна (в.т.ч. производных работах).
\item Зарабатывать на моем пособии без договоренности со мной нельзя (запрет на
использование в коммерческих целях).
\item Все производные работы должны распространяться на тех же условиях.
\stopitemize

Если вам понравилось пособие, я буду рад вашим комментариям. С удовольствием
приму в коллекцию дары в виде экзотичных образцов материалов, химических
элементов. Если у вас есть дополнения, или вы нашли ошибку - обязательно
напишите мне письмо на pavel.serkov@gmail.com.

В документе часто встречаются ссылки на различные странички в интернете. Чтобы
ссылки не пострадали при распечатке этого документа, я их продублировал в
открытом виде в сноске.

\stoptitle

%%%%%%%%%%%%%%%%%%%%%%%%%%%%%%%%%%%%%%%%%%%%%%%%%%%%%%%%%%%%%%%%%%%%%%%%%%%%%%%%%%%%%%%%%%%%%%%%%%%

\startchapter[title={Проводники}]

Двадцатый век — век пластмасс. До появления широкого спектра синтетических
полимерных материалов, человек использовал в конструировании металлы, и
материалы природного происхождения — дерево, кожу и т.д. Сегодня мы завалены
пластмассовыми изделиями, начиная от одноразовой посуды, заканчивая
тяжелонагруженными деталями двигателей автомобилей. Пластмассы во многом
превосходят металлы, но никогда не вытеснят их полностью, поэтому рассказ
начнется с металлов. Металлам посвящены сотни книг, дисциплина, посвященная им
называется "металловедение".

Нас интересуют металлы с точки зрения электронной техники. Как проводники, как
часть электронных приборов. Все остальные применения - например такие как
конструкционные материалы в данное пособие пока не входят.

Главное для электронной техники свойство металлов - это способность хорошо
проводить электрический ток. Посмотрим на таблицу удельного сопротивления
различных металлов:

\starttabulate[|l|l|]
\HL
\NC Металл \NC \rho, \unit{ohm square millimeter per meter} \NR
\HL
\NC Серебро     \NC 0,015...0,0162  \NR
\NC Медь        \NC 0,01724...0,018 \NR
\NC Золото      \NC 0,023           \NR
\NC Алюминий    \NC 0,0262...0,0295 \NR
\NC Иридий      \NC 0,0474          \NR
\NC Вольфрам    \NC 0,053...0,055   \NR
\NC Молибден    \NC 0,054           \NR
\NC Цинк        \NC 0,059           \NR
\NC Никель      \NC 0,087           \NR
\NC Железо      \NC 0,098           \NR
\NC Платина     \NC 0,107           \NR
\NC Олово       \NC 0,12            \NR
\NC Свинец      \NC 0,217...0,227   \NR
\NC Титан       \NC 0,5562...0,7837 \NR
\NC Висмут      \NC 1,2             \NR
\HL
\stoptabulate

Видим лидеров нашего списка: Ag, Cu, Au, Al.

\startsubject[title={Серебро}]

\materialdesc{Ag - Серебро.} Драгоценный металл.\footnote{Понятие «драгоценный
металл» означает в том числе особые условия по работе с металлом,
устанавливаемые законодательством.}
Серебро - самый дешевый из драгоценных металлов, но тем не менее слишком дорог,
чтобы делать из него провода. На 5\% лучшая электропроводность по сравнению с
медью, при разнице в цене почти в 100 раз.

\startsubsubject[title={Примеры применения}]

  \usagedesc{В виде покрытий проводников в СВЧ технике.} Ток высокой частоты, из-за
  скин-эффекта\footnote{https://ru.wikipedia.org/wiki/Скин-эффект} течет по
  поверхности проводника, а не в его толще, поэтому тонкое покрытие волновода
  серебром дает бОльший прирост проводимости, чем покрытие
  серебром проводника для постоянного тока

  \usagedesc{В сплавах контактных групп.} Контакты силовых, сигнальных реле, рубильников,
  выключателей чаще всего изготовлены из сплава с содержанием серебра. Переходное
  сопротивление такого контакта получается ниже медного, он меньше подвержен
  окислению. Так как контакт обычно миниатюрен, стоимость этой малой добавки
  серебра к стоимости изделия незначительно. Хотя при утилизации большого количества реле,
  стоимость серебра делает
  целесообразным\footnote{http://www.weeerecycling.ru/2016/10/04/контакты-от-пускателей/}
  работу бокорезами по отделению контактов в кучку для последующего аффинажа.

    \placefigure[here]{
        Контакты силового реле на 16А.
        Согласно документации производителя контакты содержат серебро и кадмий.
    }{\externalfigure[ag-relay.jpg][width=\textwidth]}

    \placefigure[here]{
        Различные реле. Верхнее реле имеет даже посеребренный корпус с
        характерной патиной.  Содержание драгметаллов в изделиях, выпущенных в СССР
        было указано в паспортах на изделия.
    }{\externalfigure[ag-relais.jpg][width=\textwidth]}

  \usagedesc{В качестве присадки в припоях.} Качественные припои (как твёрдые так и мягкие)
  часто содержат серебро.

  \usagedesc{Проводящие покрытия на диэлектриках.} Например для получения
  контактной площадки на керамике, на неё наносится суспензия из серебряных
  частиц с последующим запеканием в печи (метод "вжигания").

  \usagedesc{Компонент электропроводящих клеев и красок.} Электропроводящие чернила
  часто содержат суспензию серебряных частиц. По мере высыхания таких чернил,
  растворитель испаряется, частицы в растворе оказываются всё ближе, слипаясь и
  создавая проводящие мостики, по которым может протекать ток. Хорошее
  видео\footnote{https://www.youtube.com/watch?v=dfNByi-rrO4}
  с рецептом по созданию таких чернил.

\stopsubsubject

\startsubsubject[title={Недостатки}]

  Несмотря на то, что серебро - благородный металл, он окисляется в среде с содержанием серы:

  \chemical{4Ag,+,2H_2S,+,O_2,->,2Ag_2S,+,2H_2O}

  Образуется темный налет - "патина". Также источником серы может служить резина,
  поэтому провод в резиновой изоляции и посеребренные контакты - плохое
  сочетание.

  Потемневшее серебро можно очистить
  химически.\footnote{http://and-ep.livejournal.com/6299.html}
  В отличии от чистки абразивными пастами (в том числе зубной пастой) это самый
  нежный способ чистки, не оставляющий царапин.

\stopsubsubject

\stopsubject


\startsubject[title={Медь}]

\materialdesc{Cu - медь.} Основной металл проводников тока. Обмотки
электродвигателей, провода в изоляции, шины, гибкие проводники - чаще всего это
именно медь. Медь нетрудно узнать по характерному красноватому цвету. Медь
достаточно устойчива к коррозии.

\startsubsubject[title={Примеры применения}]

  \usagedesc{Провода.} Основное применение меди в чистом виде. Любые добавки
  снижают электропроводность, поэтому сердцевина проводов обычно - чистейшая
  медь.

  \placefigure[here]{
    Гибкие многожильные провода различного сечения.
  }{\externalfigure[cu-wires.jpg][width=\textwidth]}

  \usagedesc{Гибкие тоководы.} Если проводники для стационарных устройств можно
  в принципе изготовить из любого металла, то гибкие проводники делают почти
  всегда только из меди, алюминий для этих целей слишком ломкий. Содержат
  множество тоненьких медных жилок.

  \usagedesc{Теплоотводы.} Медь не только на 56\% лучше алюминия проводит ток,
  но ещё имеет почти вдвое лучшую теплопроводность. Из меди изготавливают тепловые
  трубки, радиаторы, теплораспределяющие пластины. Так как медь дороже алюминия,
  часто радиаторы делают составными, сердцевина из меди, а остальная часть из
  более дешевого алюминия.

  \placefigure[here]{
    Радиаторы охлаждения процессора. Центральный стержень изготовлен из меди,
    он хорошо отводит тепло от кристалла процессора, а алюминиевый радиатор с
    развитым оребрением уже охлаждает сам стержень.
  }{\externalfigure[cu-cooler.jpg][width=\textwidth]}

  \usagedesc{При изготовлении фольгированных печатных плат.} Печатные платы, в
  любом электронном устройстве изготовлены из пластины диэлектрика, на который
  наклеена медная фольга. Все соединения между элементами печатной платы
  выполнены дорожками из медной фольги.

\stopsubsubject

\startsubsubject[title=Интересные факты о меди]
\startitemize

  \item Медь - достаточно дорогой металл, поэтому китайцы стараются экономить
  на нем. Занижают сечение проводов (когда написано 0,75 мм^2, а фактически 
  0,11 мм^2.\footnote{http://serkov.su/blog/?p=2129} Окрашивают алюминий
  "под медь" в обмотках, внешне обмотка медная, а стоит соскрести изоляцию -
  алюминиевая. Этим грешат\footnote{http://www.owen.ru/forum/showthread.php?t=22703}
  и отечественные производители, кабель сечением 2,5 мм^2 вполне может оказаться
  сечением 2,3 мм^2, поэтому запас прочности и входной контроль не будут лишними.
  Разумеется, надежность контакта в электроарматуре жилы сечением 2,3 мм^2,
  рассчитанной на жилу 2,5 мм^2, будет ненадежной.

  \item Медь окрашивает пламя в зелёный цвет, это свойство использовали для
  обнаружения меди в руде, когда не был доступен химический анализ.
  Зеленый след в пламени — показатель наличия меди.

  \item Медь - мягкий металл, но если добавить к меди хотя бы 10\% олова,
  получается твёрдый, упругий сплав - бронза. Именно освоение получения бронзы
  послужило названием к исторической эпохе - бронзовому веку1. 

  \item Медь - один из немногих мягких металлов с высокой температурой
  плавления, поэтому из меди изготавливают уплотнительные прокладки например для
  высокотемпературной, или вакуумной техники. Например уплотнительная прокладка
  пробки картера двигателя.

  \item При механической обработке (например волочении) медь уплотняется и
  становится жёсткой. Для восстановления исходной мягкости и пластичности медь
  "отжигают" в защитной атмосфере, нагревая до 500-700°С и выдерживая некоторое время.
  Поэтому некоторые медные изделия твёрдые, а некоторые мягкие, например медные трубы.

  \item Медь не даёт искр. Для работы во взрывоопасных местах, например на газопроводе,
  используют искробезопасный инструмент, стальной инструмент покрытый слоем меди,
  или инструмент изготовленный из сплавов меди - бронз. Если таким инструментом
  случайно чиркнуть по поверхности он не даст опасных искр.

  \item Так как температурный коэффициент сопротивления для чистой меди
  известен, из меди изготавливают термометры сопротивления (тип ТСМ — Термометр
  Сопротивления Медный, есть еще ТСП — Термометр Сопротивления Платиновый).
  Термометр сопротивления — это точно изготовленный резистор, навитый из медной
  проволоки. Измерив его сопротивление, можно по таблице или по формуле
  определить его температуру достаточно точно.

\stopitemize
\stopsubsubject

\stopsubject

\startsubject[title=Алюминий]

\materialdesc{Al - Алюминий.} "Крылатый металл" третий по проводимости после
серебра и меди. Алюминий хоть и проводит ток почти в полтора раза хуже меди, но
он легче в 3,4 раза и в три раза дешевле. А если посчитать проводимость, то
эквивалентный медному проводник из алюминия будет дешевле в 6,5 раз! Алюминий
бы вытеснил медь, как проводник везде, если бы не пара его противных свойств,
но об этом в недостатках.

Относительно невысокая температура плавления (660°С) алюминия делает возможным
отливку деталей в песочные формы в условиях гаража/мастерской.

\startsubsubject[title={Примеры применения}]

  \usagedesc{Провода.} Алюминий дешев, поэтому толстые силовые кабели,
  СИП\footnote{СИП — Самонесущий Изолированный Провод},
  ЛЭП\footnote{ЛЭП — Линия ЭлектроПередач} выгодно делать из алюминия. В старых
  домах квартирная проводка сделана алюминиевым проводом (с 2001 года ПУЭ
  запрещает в квартирах использовать алюминиевый провод, только медный,
  см ниже\footnote{ПУЭ 7 издание п. 7.1.34}) Также алюминий не используется как
  проводник в ответственных применениях.

  \placefigure[here]{
     Слева старый алюминиевый провод. Справа алюминиевые кабели различного сечения,
     пригодные для укладки в грунт. В частности кабелем справа был подключен к
     электроэнергии целый этаж здания. Кабель помимо наружной резиновой оболочки
     имеет бронирующую стальную ленту, для защиты нижележащей изоляции от повреждений,
     к примеру лопатой при раскопке.
  }{\externalfigure[al-wires.jpg][width=\textwidth]}

  \usagedesc{Теплоотводы.} Не только домашние батареи делают из алюминия, но и
  радиаторы у микросхем, процессоров, делают из алюминия.

  \placefigure[here]{
    Различные алюминиевые радиаторы.
  }{\externalfigure[al-cooler.jpg][width=\textwidth]}

  \usagedesc{Корпуса приборов.} Корпус жёсткого диска в вашем компьютере отлит
  из алюминиевого сплава. Небольшая добавка кремния улучшает прочностные качества
  алюминия, сплав силумин\footnote{https://ru.wikipedia.org/wiki/Силумин} -
  это корпуса жёстких дисков, бытовых приборов, редукторов и т.~д.
  Анодированный алюминий (алюминий, у которого электрохимическим путем окисная
  пленка на поверхности сделана потолще и прочнее) хорошо окрашивается и просто красив.
  Окисная пленка (\chemical{Al_2O_3} - из того  же вещества состоят драгоценные
  камни рубины и сапфиры) достаточно твёрдая и износостойкая, но к сожалению
  алюминий под ней мягок, и при сильном воздействии ломается как лёд на воде.

  \usagedesc{Экраны.} Электромагнитное экранирование часто делается из алюминиевой
  фольги или тонкой алюминиевой жести. Можете провести простой эксперимент, мобильный
  телефон завернутый в фольгу потеряет сеть - он будет заэкранирован.

  \usagedesc{Отражающее покрытие у зеркал.} Тонкая пленка алюминия на стекле
  отражает 89\%\footnote{значения примерные, точное значение зависит от длины
  волны и типа покрытия} падающего света (Серебро 98\%, но на воздухе темнеет
  из-за сернистых соединений). Любой лазерный принтер содержит вращающееся зеркало,
  покрытое тонким слоем алюминия.

  \placefigure[here]{
    Зеркала от оптической системы планшетного сканера. Обратите внимание,
    оптические зеркала имеют металлизацию стекла снаружи, в отличии от привычных
    бытовых зеркал, где отражающее покрытие для защиты за стеклом. Бытовые зеркала
    дают двойное отражение — от поверхности стекла и от отражающего покрытия,
    что не так критично в быту, как защищенность отражающего покрытия.
  }{\externalfigure[al-mirror.jpg][width=\textwidth]}

  \usagedesc{Электроды обкладок конденсаторов.} Алюминиевая фольга, разделенная
  слоем диэлектрика и туго свернутая в цилиндр - часть электрических конденсаторов.
  (Впрочем, для уменьшения габаритов конденсаторов фольгу заменяют алюминиевым напылением.)
  Тот факт, что пленка оксида алюминия тонкая, прочная и не проводит ток, используется
  в электролитических конденсаторах, обладающими огромными для своих габаритов
  электрическими емкостями.

\stopsubsubject

\startsubsubject[title={Недостатки}]

  \problemdesc{Алюминий - металл активный,} но на воздухе покрывается оксидной пленкой,
  которая предохраняет металл от разрушения и скрывает его активную натуру. Если
  не дать алюминию формировать стабильную защитную пленку, например капелькой
  ртути, алюминий активно реагирует\footnote{https://www.youtube.com/watch?v=Z7Ilxsu-JlY}
  с водой. В щелочной среде алюминий  растворяется, попробуйте залить
  алюминиевую фольгу средством для прочистки труб - реакция будет бурная, с выделением
  взрывоопасного водорода.  Химическая активность алюминия, в паре с большой
  разницей в электрооотрицательности с медью делает невозможным прямое соединение
  проводов из этих двух металлов. В  присутствии влаги (а она в воздухе есть почти всегда)
  начинает протекать гальваническая коррозия\footnote{http://lab115.com/?p=8}
  с разрушением алюминия.

  \problemdesc{Алюминий ползуч.} Если алюминиевый провод очень сильно сжать, он деформируется и
  сохранит новую форму - это называется "пластическая деформация". Если сжать его
  не так сильно, чтобы он не деформировался, но оставить под нагрузкой надолго -
  алюминий начнет "ползти" меняя форму постепенно. Это пакостное свойство ведет к
  тому, что хорошо затянутая клемма с алюминиевым проводом спустя 5-10-20 лет
  постепенно ослабнет и будет болтаться, не обеспечивая былого электрического
  контакта. Это одна из причин, почему ПУЭ запрещает тонкий алюминиевый провод
  для разводки электроэнергии по потребителям в зданиях.
  \footnote{См п. 7.1.34 ПУЭ 7 издания} В промышленности не
  сложно обеспечить регламент - так называемая "протяжка" щитка, когда электрик
  периодически проверяет затяжку всех клемм в щитке. В домашних же условиях,
  обычно пока розетка с дымом не сгорит - никто и не озаботится качеством
  контакта. А плохой контакт - причина пожаров.

  \problemdesc{Алюминий, по сравнению с медью менее пластичный,}
   риска от ножа на жиле, при сьёме изоляции с провода быстрее приведет к
  сломавшейся жиле, чем у меди, поэтому изоляцию с алюминиевых проводов надо
  счищать как с карандаша, под углом, а не в торец.

\stopsubsubject

\startsubsubject[title=Интересные факты об алюминии]
\startitemize

  \item Алюминий - хороший восстановитель, что используется для восстановления
  других металлов, например титана из состояния диоксида.
  Теодор Грейi\footnote{Настоятельно рекомендую книги Теодора Грея «Элементы.
    Путеводитель по периодической таблице», «Научные опыты с периодической
    таблицей», «Эксперименты. Опыты с периодической таблицей». Они очень хорошо
    сделаны визуально, и опыты в них не приторно безопасные, как в большинстве
    современных пособий, могут и бабахнуть.}
  в домашних условиях проводил\footnote{http://graysci.com/chapter-five/homemade-titanium/}
  такой опыт. В смеси с окислом железа алюминиевая пудра
  образует термит\footnote{https://ru.wikipedia.org/wiki/Термитная_смесь}
  адская смесь, которая горит разогреваясь до 2400°С при этом
  восстанавливается железо и весело стекает вниз, что используется для сварки
  рельсов, иным способом такой кусок железа качественно и быстро не прогреть.
  Термитные карандаши позволяют в полевых условиях сваривать провода, а бравый
  спецназовец термитной горелкой 
  пережжет\footnote{http://www.empi-inc.com/tec_torch.html} дужку самого крепкого замка.

  \item Чтобы сделать бисквит нежным и воздушным используется пекарский
  порошок\footnote{https://ru.wikipedia.org/wiki/Пекарский_порошок}.
  Такой же порок есть для того, что бы сделать пористым бетон - Алюминий + щелочь (ССЫЛКА)

  \item Алюминий - активный металл, но он быстро покрывается окисной пленкой,
  которая защищает его от разрушения. Рубин, сапфир, корунд - это всё названия
  одного и того же вещества - оксида алюминия \chemical{Al_2O_3} Белые точильные
  круги и бруски состоят из корунда - оксида алюминия.

  \item Можно убедиться в активности алюминия простым опытом. Нарежьте
  алюминиевую фольгу в стакан, добавьте медный купорос и поваренную соль, залейте
  холодной водой. Спустя некоторое время смесь закипит, алюминий будет
  окисляться, восстанавливая медь, с выделением тепла.

  \item Алюминий неплохо поддается экструзии. Корпуса приборов из нарезанного и
  обработанного экструдированного профиля значительно дешевле литых.

  \placefigure[here]{
    Алюминиевый корпус внешнего аккумулятора для телефона. Экструдированный
    анодированный окрашенный профиль.
  }{\externalfigure[al-extrusion.jpg][width=\textwidth]}

  \item Алюминий весьма посредственно
  паяется\footnote{http://aluminium-guide.ru/myagkie-pripoi-dlya-alyuminiya/}
  мягкими (оловянно-свинцовыми) припоями, неплохо паяется цинковыми припоями.
  При конструировании приборов это стоит помнить, соединить провод с алюминиевым
  шасси проще прикрутив винтом к запрессованной стойке, чем припаять.

\stopitemize
\stopsubsubject

Еще раз важное замечание. {\bf Алюминиевые и медные проводники напрямую соединять нельзя!
Для соединения проводников из меди и алюминия используйте промежуточный металл -
например стальную клемму.}

\startsubsubject[title={Источники}]

  В крупных строительных магазинах (OBI, Leroy Merlin, Castorama) обычно есть в
  наличии алюминиевый профиль разных размеров и форм. Неплохим источником может
  послужить штампованная алюминиевая посуда — она очень дешева и существует
  разных форм.

\stopsubsubject

\stopsubject


\startsubject[title={Железо}]

\materialdesc{Fe - железо.} Основной конструкционный материал в промышленности
используется также и в электронной технике. Плохая, по сравнению с медью,
электропроводность компенсируется очень низкой ценой. И, что важнее в россии,
меньшей привлекательностью для охотников за металлом, заземление из толстой
ржавой трубы простоит без охраны дольше красивой медной шины

\startsubsubject[title={Примеры применения}]

  \usagedesc{Метизы.} Винты, шайбы, гайки из стали изготавливаются огромными
  количествами на специально разработанном для этого оборудовании. Метизы из
  других металлов встречаются очень редко и значительно дороже. Поэтому в
  большинстве случаев медный наконечник медного провода будет притянут к медной
  же шине стальным болтом. Также важным является высокая прочность стали, медный
  болт не затянуть с усилием стального.

  \usagedesc{Клеммые колодки, соединители.} Всем известные "орехи" содержат
  стальные пластинки с защитным покрытием от коррозии. Также, применение стали
  необходимо для предотвращения гальванической коррозии, при соединении медных
  и алюминиевых проводов

  \placefigure[here]{
    Соединитель «орех». Внутри пластиковой оболочки комплект стальных пластин с
    винтами, позволяет сделать ответвление от жилы кабеля не разрезая саму жилу.
    Также позволяет перейти от алюминиевой жилы на медную.
  }{\externalfigure[fe-connector.jpg][width=\textwidth]}

  \usagedesc{Контуры заземления.} Требования электробезопасности обязывают
  предусматривать заземление. Часто, в промышленных условиях, заземляющую шину
  изготавливают из стального проката, закрепленного по периметру стены. Плохая
  электропроводность стали компенсируется большим сечением проводника.

  \placefigure[here]{
    Стальная полоса, огибающая колонну — шина заземления.
  }{\externalfigure[fe-grounding.jpg][width=\textwidth]}


  \usagedesc{Широко используются магнитные свойства стали}
   — из стальных пластин собирают сердечники трансформаторов, дросселей.

\stopsubsubject

\startsubsubject[title={Недостатки}]

  \problemdesc{Коррозия.} Железо ржавеет, при этом плотность ржавчины ниже
  плотности исходного железа, из-за этого конструкция
  распухает\footnote{https://en.wikipedia.org/wiki/Rust\#/media/File:Rust_wedge.jpg}. Поэтому железо
  необходимо покрывать защитными покрытиями - оцинковка, лужение, хромирование,
  окраска и т.~д.

\stopsubsubject

\stopsubject


\startsubject[title={Золото}]

\materialdesc{Au - Золото.} Самый бестолковый драгоценный металл. Имеет меньше
всего применений в технике из всех драгоценных металлов, но является символом
богатства. На удивление дороже платины (2017 г.), что лишено здравого смысла и
является лишь результатом спекуляций.

\startsubsubject[title={Примеры применения}]

  \usagedesc{Покрытия контактов.} Благодаря тому, что золото на воздухе не
  окисляется, контакты покрывают очень тонким слоем золота.

  \placefigure[here]{
    Золотое покрытие на различных электронных компонентах: покрытие на
    контактах платы для установки в слот, покрытие на контактах мембранных кнопок
    мобильного телефона, покрытие на штырьках процессора.
  }{\externalfigure[au-coating.jpg][width=\textwidth]}

  \usagedesc{Защита от коррозии.} В некоторых ответственных применениях
  используется золотое покрытие для защиты проводников от коррозии (в основном -
  военка)

\stopsubsubject

\startsubsubject[title={Интересные факты о золоте}]

  Золото - один из четырех металлов, имеющим оттенок в не окислившемся
  состоянии. Все остальные металлы белые. (золото, цезий - желтоватые, медь -
  красноватая, осмий имеет голубой отлив)

  Плотность золота отличается от плотности вольфрама незначительно (19,32 г/см^3
  у золота, 19,25 г/см^3 у вольфрама), этим пользуются для
  подделки\footnote{http://www.zerohedge.com/news/tungsten-filled-10-oz-gold-bar-found-middle-manhattans-jewelry-district}
  золотых слитков, покрывая вольфрамовый слиток слоем золота. Возможно это одна из
  причин, почему американцы никому не дают проверить подлинность их золотого
  запаса. И возможно поэтому они отдали Германии их золото не
  сразу\footnote{https://utro.ru/articles/2016/03/21/1275137.shtml}.

  Можно извлечь\footnote{https://geektimes.ru/post/258242/} золото химически из горы старой
  электроники, но это не всегда экономически целесообразно и
  преследуется по закону (ст. 191, 192 УК РФ).

\stopsubject


\startsubject[title={Никель}]

\materialdesc{Ni - Никель.} Замечательный металл, но в электронной технике
основное применение - как дешевая альтернатива золоту - покрытие контактов.
Если контакт покрыт белым блестящим металлом, то это скорее всего никель.

\startsubsubject[title={Примеры применения}]

  \usagedesc{Покрытие контактов.} Наносится на медь, пластик для надежного
  контакта и для декоративных целей. Жадные китайцы иногда вообще делают контакты
  из пластмассы, покрывая сверху слоем никеля и хрома, внешне  выглядит
  нормальным, даже как то работает, но ни о какой надежности речи не идет.

  \placefigure[here]{
    Различные разъемы, покрытые никелем для надежного контакта.
  }{\externalfigure[ni-coating.jpg][width=\textwidth]}

  К сожалению образец я выбросил, вот фото двух разъемов, у левого центральный
  стержень цельнометаллический, а у правого - латунная трубочка, заполненная
  пластиком.

  \placefigure[here]{
    У разъема справа для экономии металла сердцевина штыря сделана
    полой с заливкой пластиком. Латунная никелированная трубочка, из которой сделан
    штырь, не самый худший вариант.
  }{\externalfigure[ni-connectors.jpg][width=\textwidth]}

  \usagedesc{Тоководы у ламп.} Сплав Платинит (46\% Ni, 0,15\% C, остальное - Fe)
  не содержит платины, но имеет очень близкое к платине значение линейного
  температурного расширения, что позволяет делать из него надежные электроды,
  проходящие через стекло. Такие электроды при изменении температуры не вызывают
  растрескивания стекла и потерю герметичности.

  \usagedesc{Промежуточные защитные слои.} Для защиты от коррозии, взаимной диффузии
  металлов при создании покрытий могут формироваться промежуточные слои из никеля.
  Жала современных паяльников защищены слоем никеля, жало из голой меди медленно
  растворяется в олове, теряя форму.

\stopsubsubject

\stopsubject


\startsubject[title={Вольфрам}]

\materialdesc{W - Вольфрам.} Тугоплавкий металл, температура плавления 3422°С,
что определяет основное использование - нити накала и электроды.

\startsubsubject[title={Примеры применения}]

  \usagedesc{Нити накала.} В лампах накаливания, в галогеновых лампах спираль
  изготовлена из вольфрама, нагревается электрическим током до белого каления,
  при этом сохраняя свою форму. Также катоды в радиолампах изготавливаются из
  вольфрама, но раскаливаются не до таких высоких температур, как осветительные
  лампы, специальное покрытие на катоде позволяет протекать термоэлектронной
  эмиссии при невысоких температурах.

  \placefigure[here]{
    Нить накаливания этой галогеновой лампы изготовлена из вольфрама.
    Использование галогенов во внутренней атмосфере лампы позволяет повысить
    температуру накала спирали и уменьшить габарит лампы без страха, что вольфрам
    постепенно осядет на стенках колбы.
  }{\externalfigure[w-filament.jpg][width=\textwidth]}

  \placefigure[here]{
    Мощная лампа накаливания от проектора. Даже тугоплавкий вольфрам со временем
    испаряется и оседает на стенках колбы в виде темного налета. Данного недостатка
    лишены галогеновые лампы, см. фото выше.
  }{\externalfigure[w-lamp.jpg][width=\textwidth]}

  \usagedesc{Электроды дуговых ламп и сварочные электроды.} В ксеноновых дуговых лампах,
  ртутных дуговых лампах электроды должны выдерживать температуру электрической
  дуги при этом не расплавляясь и не изменяя своей формы, что под силу только
  вольфраму. Также электроды для сварки неплавящимся электродом изготовлены из
  вольфрама (TIG сварка)

  \usagedesc{Аноды у рентгеновских трубок.} Поток электронов от катода в рентгеновской
  трубке, разогнанный высоким напряжением тормозится бомбардируя анод, очень сильно
  нагревая его, поэтому такие аноды зачастую изготавливаются из вольфрама (но не всегда).

\stopsubsubject

\startsubsubject[title={Источники}]

  Вольфрам — не очень пластичный материал, поэтому спиральку из лампы накаливания
  вряд ли удастся выпрямить и использовать по своему разумению. Если вдруг понадобится
  вольфрамовый стержень — вам пригодится любой магазин по сварочному делу, электрод для
  TIG-горелки без содержания лантана и других присадок.

\stopsubsubject

\stopsubject


\startsubject[title={Ртуть}]

\materialdesc{Hg - Ртуть.} При комнатной температуре - блестящий, собирающийся в шарики
жидкий металл. По экологическим соображениям использование ртути сокращается, но она
широко использовалась в старых приборах, поэтому заслуживает упоминания.

\startsubsubject[title={Примеры применения}]

  \usagedesc{Жидкий контакт} в датчиках положения, ртутных электроконтактных термометрах.

  \placefigure[here]{
    Различные ртутные приборы. Слева — мощный ртутный переключатель, замыкающий/размыкающий
    цепь при наклоне. Ниже на чёрных платках — аналогичные китайские ртутные
    переключатели — датчики положения из детского набора с Arduino. Сверху — колба
    ртутного электроконтактного термометра. В стекло вплавлены проволочки так, что
    при температуре 70°С столбик ртути в капилляре замыкает цепь (температура
    указана на корпусе).
  }{\externalfigure[hg-devices.jpg][width=\textwidth]}

  \usagedesc{Пары ртути — рабочий газ люминесцентных ламп.} Несмотря на то, что люминесцентная
  лампа должна содержать небольшое количество ртути, в некоторых лампах ртути добавлено от души,
  и видно, как в колбе перекатывается шарик ртути. Пары ртути при возбуждении их электрическим
  током излучают ультрафиолетовые лучи.

  \usagedesc{В ртутных счетчиках времени наработки.} В старой технике ртутный капиллярный
  кулономер использовался как счетчик часов, которые проработал прибор. Гениальная по простоте
  и надёжности конструкция. Увы в моей коллекции такого нет, но вот хорошее
  видео\footnote{http://benkrasnow.blogspot.ru/2014/05/unusual-usage-hours-counter-with.html}.

\stopsubsubject

  Все изделия, содержащие ртуть, должны утилизироваться специализированной службой.
  Также все разливы ртути собраны, а поверхности демеркуризованы.
  {\bf НЕ СОБИРАЙТЕ РТУТЬ ПЫЛЕСОСОМ!} Ртуть хорошо
  испаряется\footnote{https://www.youtube.com/watch?v=TZprgqZh4IE} при комнатной температуре,
  поэтому закатившийся в щель шарик ртути долгое время будет отравлять воздух.

  Некоторые в детстве играли шариками ртути, и "с ними ничего не было".
  Металлическая ртуть относительно безопасна, контакт с ней не вызывает
  мгновенную смерть. Проблема наступает, когда ртуть попадает в организм/природу,
  и становится частью органических соединений. Диметилртуть настолько ядовита,
  что не нашла применения из-за чрезвычайной опасности при работе с ней.
  Вспомните об этом при соблазне выкинуть люминесцентную лампу в бак с обычным
  мусором.

\startsubsubject[title={Дополнительные сведения}]
  Не пытайтесь продать найденную ртуть (ст. 234 УК РФ). Ртуть следует сдавать на переработку в
  специализированные службы в вашем городе. Единственный широко доступный источник ртути
  (если вдруг понадобится в научной работе) — медицинские термометры.
\stopsubsubject

\stopsubject

\stopchapter

%%%%%%%%%%%%%%%%%%%%%%%%%%%%%%%%%%%%%%%%%%%%%%%%%%%%%%%%%%%%%%%%%%%%%%%%%%%%%%%%%%%%%%%%%%%%%%%%%%%

\startchapter[title={Так себе проводники}]

\startsubject[title={Углерод}]

\materialdesc{С - углерод.} Не совсем металл но тоже проводник. Графит,
угольная пыль - не такие хорошие проводники как металлы, но зато очень дешевые,
не подвержены коррозии.

\startsubsubject[title={Примеры применения}]

  \usagedesc{Компонент резисторов.} В виде пленок, в виде объемных брусков в диэлектрической
  оболочке.

  \usagedesc{Добавка в полимеры для придания электропроводности.} Для защиты от образования
  статического электричества достаточно ввести в состав полимера мелкодисперсный
  графит, и пластик из диэлектрика становится очень плохим проводником,
  достаточным, что бы статический заряд с него стекал. При работе с изделиями из
  такого пластика они не будут прилипать и искрить, что важно при пожароопасности
  или работе с электроникой.

  \placefigure[here]{
    Токопроводящий лак на базе мелкодисперсного графита. Покрыв пластиковую деталь
    таким лаком её электропроводность становится достаточной для выращивания слоя
    металла методом гальванопластики.
  }{\externalfigure[c-spray.jpg][width=\textwidth]}

  \usagedesc{На базе полимеров, заполненных мелкодисперсным графитом основаны различные
  нагреватели} - пленочные электронагреватели теплых полов, греющие кабели для
  систем водоснабжения, нагреватели для одежды и т.д. Высокий коэффициент
  расширения полимеров при нагреве приводит к отрицательной обратной связи, что
  делает такие нагреватели саморегулирующимися и потому безопасными. При
  пропускании тока через такой полимер, он нагревается, от нагрева расширяется,
  контакт между частичками углерода в матрице из полимера ухудшается, от этого
  увеличивается сопротивление - уменьшается протекаемый ток, уменьшается нагрев.
  В итоге, устанавливается некоторая температура полимера, стабильно
  поддерживающаяся этим механизмом обратной связи без каких либо внешних
  устройств.

  \placefigure[here]{
    Нагреватель от печки лазерного принтера. Основа — фарфор, проводники — серебро.
    Нагреватель — углеродная композиция, покрыта для защиты слоем глазури.
  }{\externalfigure[c-heater.jpg][width=\textwidth]}

  \usagedesc{Аналогично устроены полимерные самовосстанавливающиеся предохранители.} Если ток
  через такой предохранитель превысит номинальный, от нагрева полимер в составе
  расширяется, и резко увеличившееся сопротивление прерывает ток через
  предохранитель до некоторого небольшого значения. Такие предохранители
  обеспечивают медленную защиту, но не требуют замены предохранителя после каждой
  аварии.

  \usagedesc{Угольный сварочный электрод} - используется для сварки, когда от электрода
  требуется только поддерживать дугу не плавясь. Уголь значительно дешевле
  вольфрама, но менее прочен и постепенно сгорает на воздухе.

  \placefigure[here]{
    Электроды от дуговой лампы, использовавшейся для киносъемок. Марка электродов
    КСБ — Уголь КиноСьемочный Белопламенный неомедненный.
  }{\externalfigure[c-electrode.jpg][width=\textwidth]}

  \usagedesc{Медно-графитовые материалы.}
  Получают спеканием порошка меди и графита в разных пропорциях. В зависимости от
  состава могут быть от чёрных как уголь до темно красных с медным блеском.
  Используется как материал скользящих контактов - щеток электрических приборов. 
  Такие щетки обеспечивают низкое сопротивление вращению — хорошо скользят по
  контактам коллектора. Кроме того их твёрдость заметно ниже твёрдости металла
  коллектора, так что в процессе работы истираются и подлежат замене дешевые
  щетки а не дорогой ротор.

  \placefigure[here]{
    Изношенные щетки от двигателя стиральной машины. Плохой контакт щеток с
    коллектором — причина повышенного искрения.
  }{\externalfigure[c-brush.jpg][width=\textwidth]}

\stopsubsubject

\startsubsubject[title={Источники}]

  Если вдруг понадобился срочно угольный электрод, например сварить термопару,
  самый доступный способ — вытащить центральный электрод из солевой батарейки
  (маркировка которой начинается с R а не LR, щелочные («алкалиновые») не подойдут).

\stopsubsubject

\stopsubject


\startsubject[title={Нихромы}]

Для изготовления нагревателей, мощных сопротивлений требуются сплавы с
следующими требованиями:

\startitemize
  \item Относительно высокое удельное сопротивление - иначе
  нагреватель придется делать длинным и тонким, что отрицательно скажется на
  долговечности.

  \item Устойчивость к окислению на воздухе. Если в колбу лампы накаливания
  попадет воздух, то спираль очень быстро сгорит. При высоких температурах
  скорости химических реакций растут, и кислород воздуха начинает окислять даже
  стойкие при комнатной температуре металлы.

  \item Иметь приемлемые механические
  характеристики. Низкая пластичность и повышенная хрупкость негативно скажется
  на надежности изделия.
\stopitemize

Нагреватели изготавливают из следующих сплавов:

\materialdesc{Нихром} (55-78\% никеля, 15-23\% хрома) до 1100°С хотя нихромы - это целый класс
сплавов с небольшой разницей в составе.

\materialdesc{Фехраль}, название образовано от состава FeCrAl (12-27\% Cr, 3.5-5.5\% Al,
1\% Si, 0.7\% Mn, остальное Fe) до 1350°С (Иногда называют канталом — kanthal, это не марка
сплава, а торговая марка\footnote{Принадлежит компании Sandvik Materials Technology},
которая стала нарицательным, как например «термос»)

Добавка хрома обеспечивает образование защитной пленки на поверхности сплава,
благодаря чему нагреватели из нихрома могут длительное время работать на
воздухе с высокой температурой поверхности.

Фехраль после нагрева становится ломким. Нихром после нагрева еще можно
как-то гнуть. При этом фехраль дешевле нихрома, в рознице не так заметно, но
ощутимо в оптовых партиях.

Нихромовая спиралька с фитилем внутри — испаритель электронной сигареты.
Нихромовой струной, подогреваемой электрическим током, режут пенополистирол.
Также из нихрома изготавливают термосьемники изоляции — на сегодняшний день
самый надежный способ снять изоляцию с провода и не повредить токопроводящую
жилу.

На удивление, достаточно трудно купить нихром в виде проволоки в небольших
количествах, местные продавцы о количествах менее килограмма даже слышать не
хотят. Так что, если понадобится изготовить нагревательный элемент - то проще
перемотать нихром с какого-нибудь неисправного тепловентилятора. 

Концы нагревательных элементов обычно приваривают к тоководам или зажимают
механически — винтом или опрессовкой.

\stopsubject


\startsubject[title={Сплавы для изготовления термостабильных сопротивлений}]

У всех материалов есть ТКС - температурный коэффициент сопротивления, мера
того, насколько изменяется сопротивление с изменением температуры. Он может
быть положительным - как у металлов, с  ростом температуры сопротивление
растет, может быть отрицательным, как у полупроводников, с ростом температуры
сопротивление падает. При изготовлении точных измерительных приборов необходимо
иметь сопротивления с минимальным дрейфом номинала в зависимости от
температуры. Для этого изобрели сплавы с минимальным ТКС:

Константан (59\% Cu, 39-41\% Ni, 1-2\% Mn)

Мангалин (85\% Cu, 11.5-13.5\% Mn, 2.5-3.5\% Ni)

Таблица, с указанием температурного коэффициента (обозначается как α) для различных металлов:

\starttabulate[|l|l|]
  \HL
  \NC Материал  \NC Температурный коэффициент \alpha, \unit{one per degree celsius} \NR
  \HL
  \NC Кремний       \NC -0,075      \NR
  \NC Германий      \NC -0,048      \NR
  \NC Манганин      \NC 0,00002     \NR
  \NC Константан    \NC 0,00005     \NR
  \NC Нихром        \NC 0,0004      \NR
  \NC Ртуть         \NC 0,0009      \NR
  \NC Сталь 0,5\% С \NC 0,003       \NR
  \NC Цинк          \NC 0,0037      \NR
  \NC Титан         \NC 0,0038      \NR
  \NC Серебро       \NC 0,0038      \NR
  \NC Медь          \NC 0,00386     \NR
  \NC Свинец        \NC 0,0039      \NR
  \NC Платина       \NC 0,003927    \NR
  \NC Золото        \NC 0,004       \NR
  \NC Алюминий      \NC 0,00429     \NR
  \NC Олово         \NC 0,0045      \NR
  \NC Вольфрам      \NC 0,0045      \NR
  \NC Никель        \NC 0,006       \NR
  \NC Железо        \NC 0,00651     \NR
  \HL
\stoptabulate

Если упростить, то коэффициент \alpha\ говорит, во сколько раз изменится
сопротивление проводника при изменении температуры на один градус Цельсия.

\stopsubject

\stopchapter

%%%%%%%%%%%%%%%%%%%%%%%%%%%%%%%%%%%%%%%%%%%%%%%%%%%%%%%%%%%%%%%%%%%%%%%%%%%%%%%%%%%%%%%%%%%%%%%%%%%

\startchapter[title={Припои}]

Пайка — это процесс соединения двух деталей при помощи припоя, материала с
температурой плавления меньшей, чем у соединяемых деталей. Например соединение
двух медных проводников при помощи олова. Именно использование припоя —
основное отличие от сварки, когда детали соединяются расплавом из самих себя,
например стальной крюк к стальной двери приваривается при помощи стального
плавящегося сварочного электрода.

Припои чаще классифицируют на две группы — тугоплавкие (температура плавления
400°С и более) и легкоплавкие. Или, иногда, на твёрдые и мягкие. Учитывая что
мягкие припои обычно легкоплавкие, то часто твёрдые припои синоним тугоплавких,
а мягкие припои — легкоплавких.

В электронной технике припои используют для создания надежного электрического
контакта. Основные припои в электронной технике — мягкие, на базе олова и
оловянно-свинцовых сплавов. Все остальные экзотические припои рассматриваться
не будут.


\startsubject[title={Олово}]

\materialdesc{Sn - Олово.} Основной компонент мягких припоев. Олово - относительно
легкоплавкий металл, что позволяет использовать его для соединения проводников.
В чистом виде не используется (см. факты). Из-за дороговизны олова (а также
других причин, см. ниже), его в припоях разбавляют свинцом. Припой из 61\% олова
и 39\% свинца образует эвтектику\footnote{https://ru.wikipedia.org/wiki/Эвтектика},
такой смесью, ПОС-61 (Припой Оловянно-Свинцовый - 61\% олова) паяют радиодетали на
платах, провода. В менее ответственных узлах (шасси, теплоотводы, экраны и т.п.)
олово в припоях разбавляют сильнее, до 30\% олова, 70\% свинца.

Электронные устройства долгое время паяли оловянно-свинцовыми припоями. Затем
набежали экологи и заявили, что свинец - металл тяжелый, токсичный, и проблемы
бы не было, если бы все эти ваши айфоны, компьютеры и прочие гаджеты не
оказывались на свалке, откуда свинец попадает в окружающую среду. Поэтому
придумали\footnote{https://ru.wikipedia.org/wiki/Restriction_of_Hazardous_Substances_Directive}
серию бессвинцовых припоев, когда олово разбавлено висмутом, или
вовсе используется в чистом виде, стабилизированное добавками например серебра.
Но эти припои дороже, хуже по характеристикам, более тугоплавкие. Поэтому
оловянно-свинцовые припои надолго останутся в ответственных изделиях военного,
космического, медицинского применения.

Кроме того, бессвинцовые припои склонны к образованию "усов". Оловянные усы -
длинные тонкие кристаллы, вырастающие из оловянного припоя - причина отказов и
сбоев аппаратуры. К сожалению присадки в припои не позволяют на 100\% прекратить
рост "усов", поэтому оловянно-свинцовые припои, как проверенные временем,
используются в критичных системах - космос, медицина, военка, атомные
применения. Подробнее\footnote{http://www.yaplakal.com/forum2/topic1490898.html} про усы.

\placefigure[here]{
  Катушки и прутки оловянно-свинцовых припоев. Проволока из припоя содержит
  центральный канал с флюсом, облегчающим процесс пайки.
}{\externalfigure[sn-solder.jpg][width=\textwidth]}


\startsubsubject[title={Факты об олове}]
\startitemize

  \item Чистое олово подвержено "оловяной чуме", когда при температурах ниже
  13,2°С олово меняет свою кристаллическую решетку превращаясь из блестящего
  металла в серый порошок (как при нагревании алмаз превращается в графит).
  Согласно байкам, оловянная чума - одна из причин поражения Наполеоновской
  армии в условиях суровых российских городов (представьте, как на морозе ваши
  пуговицы, ложки, вилки, кружки превращаются в серый порошок). И вполне
  состоявшийся факт, что оловянная чума стала одной из причин которая погубила
  экспедицию Скотта\footnote{https://ru.wikipedia.org/wiki/Терра_Нова_(экспедиция)} -
  консервные банки, емкости с топливом были пропаяны оловом и на морозе просто
  развалились.  Небольшая добавка висмута практически устраняет оловянную
  чуму\footnote{https://www.youtube.com/watch?v=gu4xpw-EM88}.

  \item Олово проводит электрический ток в 7 раз хуже меди.

  \item Олово используется как защитное покрытие консервных банок - луженая жесть при
  контакте с пищей не делает её опасной. (но так как олово правее железа в ряду
  напряженности металлов, лужение не защищает железо от коррозии гальванически,
  как цинк, который левее железа в ряду напряженности. Как работает
  гальваническая защита можно прочитать по ссылке ССЫЛКЕ)

  \item До широкого распространения алюминия, фольгу делали из олова, её называли
  "станиоль" (от stannum - латинское навание олова).

  \item Не пытайтесь отремонтировать ювелирные украшения при помощи мягких оловянных
  и оловянно-свинцовых припоев. Прочность соединения будет неприемлемой, а
  наличие легкоплавкого припоя на поверхности осложнит нормальную пайку твёрдыми
  припоями.

\stopitemize
\stopsubsubject

\startsubsubject[title={Легкоплавкие припои}]

На базе сплавов с содержанием олова были разработаны легкоплавкие припои. И
даже очень легкоплавкие припои, которые плавятся в горячей воде. Хороший
список\footnote{https://ru.wikipedia.org/wiki/Легкоплавкие_сплавы} сплавов есть в википедии.

\stopsubsubject


\startsubsubject[title={Основные припои для радиоаппаратуры}]
\startitemize

  \item ПОС61 — 61\% олова, остальное — свинец. Температура плавления (ликвидус)
  183°С. Есть множество сходных по составу и по свойствам импортных припоев, в
  которых пропорции компонентов отличаются на пару процентов, например Sn60Pb40
  или Sn63Pb37.

  \item ПОС-40 — 40\% олова. Остальное — свинец. Температура плавления (ликвидус)
  238°С Менее прочный, более тугоплавкий, неэвтектический (плавится не сразу,
  есть диапазон температур при котором припой больше походит на кашу). Но
  благодаря тому что чуть ли не в 2 раза дешевле (олово дорогое), применяется для
  неответственных соединений — пайка экранов, шин. Аналогичны припои ПОС-33
  (247 °C), ПОС-25 (260 °C), ПОС-15 (280 °C).

  \item Бессвинцовые припои. Хорошая статья.\footnote{http://go-radio.ru/lead-free-solder.html}

\stopitemize

Замыкают список совсем легкоплавкие припои:

\startitemize
\item Сплав Розе: 25\% Sn, 25\% Pb, 50\% Bi. Температура плавления +94°С

\item Сплав Вуда: 12,5\% Sn, 25\% Pb, 50\% Bi, 12.5\% Cd Температура плавления +68,5°С
\stopitemize

Применяются для лужения печатных плат любителями, так как плавятся в горячей
воде, и можно резиновым шпателем под слоем кипящей воды быстро покрыть припоем
медную фольгу печатной платы.
\stopsubsubject

Если спаять подпружиненные контакты легкоплавким припоем, то получится простой
и надежный термопредохранитель, при превышении температуры припой плавится и
контакты разрывают цепь. Правда предохранитель получится одноразовым.

\stopsubject

\stopchapter

%%%%%%%%%%%%%%%%%%%%%%%%%%%%%%%%%%%%%%%%%%%%%%%%%%%%%%%%%%%%%%%%%%%%%%%%%%%%%%%%%%%%%%%%%%%%%%%%%%%

\startchapter[title={Прочие проводники}]

\startsubject[title={Термопарные сплавы}]

Для изготовления термопар используют сплавы стойкие к высоким температурам,
но при этом обладающие высокой ТермоЭДС. Подробнее про
термопары\footnote{https://ru.wikipedia.org/wiki/Термопара}
можно прочитать в соответствующей литературе.

Сплавы:
\startitemize
  \item Хромель (90\% Ni, 10\% Cr)
  \item Копель (43\% Ni, 2-3\% Fe, 53\% Cu)
  \item Алюмель (93-96\% Ni, 1,8-2,5\% Al, 1,8-2,2\% Mn, 0,8-1,2\% Si)
  \item Платина (100\% Pt)
  \item Платина-родий (10-30\% Rh)
\stopitemize

Соединяя два проводника из двух разных металлов получают термопары, например
термопара типа K (ТХА — Термопара Хромель-Алюмель)

\stopsubject


\startsubject[title={Оксид Индия-Олова}]

\materialdesc{Оксид Индия - Oлова} (Indium tin oxide или сокращённо ITO) -
полупроводник, но обладает невысоким сопротивлением, а самое главное, пленка из
оксида индия-олова прозрачная. Это свойство используется при производстве ЖК
дисплеев, сетка электродов на поверхности нанесена именно из оксида
индия-олова. Также резистивные touch панели имеют прозрачное проводящее
покрытие.

Пленка ITO едва видна в отражении, что бы хоть как то она была заметна пришлось
разобрать ЖК дисплей:

\placefigure[here]{
  Стекла от ЖК индикатора электронных часов. Индикатор подключался к электронной
  схеме через токопроводящую резинку, гребенка контактов видна в нижней части
  стекла.
}{\externalfigure[in-sn-displays.jpg][width=\textwidth]}


\placefigure[here]{
  На просвет проводящая пленка не видна
}{\externalfigure[in-sn-transparency.jpg][width=\textwidth]}

\placefigure[here]{
  На удивление, сопротивление пленки довольно низкое.
}{\externalfigure[in-sn-resistance.jpg][width=\textwidth]}

\stopsubject

\stopchapter

%%%%%%%%%%%%%%%%%%%%%%%%%%%%%%%%%%%%%%%%%%%%%%%%%%%%%%%%%%%%%%%%%%%%%%%%%%%%%%%%%%%%%%%%%%%%%%%%%%%

\startchapter[title={Диэлектрики (Совсем не проводники)}]

Помимо проводников для производства электронной технике нужны диэлектрики. В
зависимости от условий и задач, могут быть важны разные свойства диэлектрика:
теплостойкость, тангенс угла потерь, гигроскопичность, механическая прочность и
т.~д.

\placefigure[here]{
  Картинка с айсбергом и его макушкой "что вы узнаете прочитав пособие"
}{\externalfigure[iceberg.jpg][width=\textwidth]}

Раздел с полимерами еще более поверхностный. Дело в том, что свойства
полимерного материала зависят от условий синтеза, введенных добавок,
термообработки, последующей обработки. Таким образом два образца полистирола
могут весьма значительно отличаться по свойствам. Производители пластиков идут
на различные ухищрения и манипуляции с составом, внося важные и не очень
изменения. Это как с книгами, разные издания одного и того же произведения, где
то на газетной бумаге с плохой версткой, а где то на качественной бумаге с
цветными иллюстрациями от модного художника. И та и другая книга - "Властелин
колец", но впечатления от использования могут отличаться. Поэтому приведены
некоторые общие свойства разных видов полимеров, за более точными
характеристиками нужно обращаться в справочник.


Материалы, которые применяются в электронной технике меняются по мере
прогресса. Так, ранее широко использовалось к примеру дерево, шелк, эбонит.
Сегодня же многие материалы вытеснены более дешевыми, технологичными
заменителями. В пособии есть описание в том числе исторических материалов,
данных для общего развития. Также добавлена информация, необходимая для полноты
раскрытия темы.




\stopchapter

\stoptext
