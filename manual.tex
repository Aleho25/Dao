%% This is ConTeXt
%% http://wiki.contextgarden.net/Main_Page
%% Compiling:
%$ context manual.tex



% - дефис
% – короткое тире Н-д
% — длинное тире М-д


% * Пофиксить отображение двух картинок в ряд, а не друг под другом

\def\homepage{\hyphenatedurl{https://serkov.su/blog/?page_id=2573}}
\def\manualver{1.4 }

\enableregime[utf]
\mainlanguage[russian]

\definepapersize[RIDERO-A5][width=145mm,height=205mm]
\setuppapersize[RIDERO-A5]
%\setuppapersize[A4]
\setuplayout [width=middle]
%convert to grayscale for RIDERO: gs -sOutputFile=manual_bw.pdf -sDEVICE=pdfwrite -sColorConversionStrategy=Gray -dProcessColorModel=/DeviceGray -dCompatibilityLevel=1.4 -dNOPAUSE -dAutoRotatePages=/None -dBATCH manual.pdf


\setuppagenumbering [location=footer]

%Стиль заголовков разделов
\setuphead[subject, subsubject][style={\bf}]		%Заголовки жирным
\setuphead[subject][style={\switchtobodyfont[14.4pt]}]	%Подзаголовки крупнее
\setuphead[subject][incrementnumber=yes, number=yes, page=yes, continue=no]	%Считать номера=да, показывать номер=да, начинать новую страницу=да, продолжать главу=нет (последний мараметр если дефолтный то page=yes игнорится)



\setupcombinedlist[content][list={chapter,subject}] %список того что попадает в список содержания


%стиль подписей к картинкам и таблицам
\setupcaptions [style=small]


\setupbodyfont[computer-modern-unicode,10pt,rm]

% for the document info/catalog (reported by 'pdfinfo', for example)
\setupinteraction[state=start,  % make hyperlinks active, etc.
  title={Руководство по материалам электротехники},
  subtitle={Простой путеводитель по миру проводников и диэлектриков},
  author={Серков Павел},
  keyword={Электроматериалы, материаловедение}]

\definedescription[materialdesc][alternative=serried, width=broad, headstyle=boldslanted]
\definedescription[usagedesc][alternative=serried, width=broad, headstyle=bold]
\definedescription[problemdesc][alternative=serried, width=broad, headstyle=bold]
\definedescription[featuredesc][alternative=serried, width=broad, headstyle=bold]

\setupexternalfigures[directory={images_LQ}]

% Документация: http://www.pragma-ade.com/general/manuals/units-mkiv.pdf
\setupprefixtext[
    femto=ф, pico=п, nano=н, micro=мк, milli=м, centi=с, deci=д,
    deca=да, hecto=г, kilo=к, mega=М, giga=Г, tera=Т, peta=П
]
\setupunittext[
    meter=м, second=с, gram=г,
    ampere=A, volt=В, ohm=Ом, watt=Вт,
    joule=Дж
]

%Оформление абзацев. Расстояие меж параграфов
\setupwhitespace [small]

%Абзацный отступ
\setupindenting[yes, first, medium]
\setupheads[indentnext=yes] 

%Шаманство с версткой

%\brokenpenalty        100     %This penalty is added after each line that ends with a hyphenated word. High values will discourage TEX in breaking a page there.
%\hyphenpenalty        50      %This penalty is added after each location in the paragraph where TEX can hyphenate and therefore this penalty determines the way TEX splits the paragraph into lines.
%\doublehyphendemerits 10000   %While TEX is breaking a paragraph into lines, it calculates demerits for potential linebreaks. This value is added to the demerits of a line if that line as well as the previous one both end with a hyphen.
%\finalhyphendemerits  5000    %When the pre-last line ends with a hyphen, TEX adds this value to the demerits of that line, thereby discouraging a line break at that point when the paragraph is split into lines.
%\widowpenalty         0       %This is the penalty added before the last line of a paragraph. This penalty determines how eager TEX will be in splitting a page before the last line.
%\clubpenalty          0       %This is the penalty added after the first line of a paragraph. This penalty determines TEX’s willingness to split a page after the first line.
%\adjdemerits          10000   %In the process of breaking a paragraph into lines, TEX tags each of the lines as very loose, loose, decent or tight. If two lines are tagged differently, TEX qualifies them as being visually incompatible. In that case the value of thisvariable is added to the demerits of the lines.




\starttext
\setuppagenumber [state=stop]


% Обложка. Не должно быть нумерации на этой странице
 \startTEXpage
    \externalfigure[cover_vertical.jpg][width=\paperwidth, height=\paperheight]
 \stopTEXpage



\page[yes]

\setuppagenumber [state=start]

%Требование ридеро - если не вставлять технические страницы то нумерация с 3х
\setcounter[userpage][3]

Версия этой книги \bold{\manualver} от \date[d=22,m=3,y=2018][day, month,year].

Книга распространяется свободно, проверить и скачать новые версии книги можно на страничке в блоге автора: \homepage.





%Данный документ сгенерирован: \currentdate

% visual debug
%\showlayout
%\showlayoutcomponents
%\showsetups
%\showmakeup
%\showframe
%\showgrid
%\showstruts
%http://wiki.contextgarden.net/Visual_Debugging

\page[yes]

% Генерация оглавления
\completecontent

\startchapter[title={Введение, которое обычно никто не читает}]

	Ковыряясь в поисках ответов на свои вопросы в разных учебниках по материаловедению, методичках, научпоп книгах
	я ужасался, насколько академический стиль изложения возводит стену между желающим узнать и знаниями. Насколько
	стремление авторов обойти острые грани, тёмные места превращает учебники в однородную бескрайнюю пустыню скуки и отчаяния.
	При этом запредельный уровень абстракции делает крайне сложным для неофита использование полученных знаний в практике.
	Поэтому я решил сделать свое руководство, с блекджеком и блудными девицами.

	Эта книга живая, по мере получения новых материалов, уточнений, комментариев от вас, дорогие читатели
	она будет дополняться, изменяться, становиться лучше. Всегда самая свежая версия книги лежит у меня на сайте в бложике
	по адресу: http://serkov.me/. Версия книги у вас в руках \manualver. Я обеими руками поддерживаю движение Open Source и Open Hardware, считаю, что
	обмен знаниями должен быть свободным, это принесет пользу для всех, поэтому
	пособие распространяется под лицензией Creative Commons 3.0 BY-NC-SA, что
	значит, вы можете делать с ним что угодно: выкладывать, распространять,
	модифицировать, соблюдая только три ограничения:

	\startitemize[intro]

		\item Ссылка на меня обязательна (в т.~ч. производных работах);

		\item Зарабатывать на моем пособии без договоренности со мной нельзя (запрет на
		использование в коммерческих целях);

		\item Все производные работы должны распространяться на тех же условиях;

	\stopitemize

	Плюшки данной книги:

	\startitemize[intro]

		\item Весь текст написан мной, и дополнен замечательными людьми, упомянутыми в разделе Благодарности. Я не включал
		информацию, в достоверности или актуальности которой я бы сомневался. Поэтому
		доля брехни по тексту в среднем ниже, чем в маркетинговых текстах
		перепродавцов-поставщиков, но выше, чем в хорошем советском учебнике.

		\item Большую часть материалов я хотя бы щупал, использовал в своих
		конструкциях, а не видел только на картинке. Все фотографии в тексте, если не указано иное, сделаны автором.

		\item Пособие полностью\footnote{Чтобы быть до конца честным — за исключением одной картинки,
		которую пришлось рисовать в чем умел.} подготовлено с использованием OpenSource продуктов
		(Linux, GIMP, LibreOffice, context). Просто из спортивного интереса.

		\item Некоторые разделы имеют пункт «Источники» - советы по поиску материалов:
		где купить, под какими названиями искать. Конечно, всё можно купить на
		Алиэкспресс и на Ebay, поэтому такой вариант не указывается. Пункт может быть
		полезен если материал нужен «здесь и сейчас».

	\stopitemize

	Для кого будет полезно это пособие прежде всего? Для всех желающих расширить
	кругозор, узнать что то новое для себя. Особенно рекомендую тем, кто хочет
	что-либо делать своими руками. Не смотря на страшное слово "Электротехника" в названии,
	руководство полезно даже тем, кто далек от электричества, точно-точно.

	Если вам понравилась книга, вы нашли ошибку, вам хочется добавить комментарий или дополнить её,
	даже просто аргументированно объяснить почему КГ/АМ  — обязательно
	напишите мне письмо на pavel.serkov@gmail.com.
	

\stopchapter

%%%%%%%%%%%%%%%%%%%%%%%%%%%%%%%%%%%%%%%%%%%%%%%%%%%%%%%%%%%%%%%%%%%%%%%%%%%%%%%%%%%%%%%%%%%%%%%%%%%

\startchapter[title={Проводники}]

	Двадцатый век — век пластмасс. До появления широкого спектра синтетических
	полимерных материалов, человек использовал в конструировании металлы и
	материалы природного происхождения — дерево, кожу и т.~д. Сегодня мы завалены
	пластмассовыми изделиями, начиная от одноразовой посуды, заканчивая
	тяжелонагруженными деталями двигателей автомобилей. Пластмассы во многом
	превосходят металлы, но никогда не вытеснят их полностью, поэтому рассказ
	начнется с металлов. Металлам посвящены сотни книг, дисциплина, посвященная им,
	называется "металловедение".

	Нас интересуют металлы с точки зрения электронной техники. Как проводники, как
	часть электронных приборов. Все остальные применения — например такие, как
	конструкционные материалы, в данное пособие пока не вошли.

	Главное для электронной техники свойство металлов — это способность хорошо
	проводить электрический ток. Посмотрим на таблицу удельного сопротивления
	\footnote{Понятие удельной величины позволяет сравнивать характеристики материалов вне
	зависимости от формы, а также рассчитывать значения для любой произвольной формы. Например
	удельное сопротивление измеряется в \unit{ohm square mm per meter}, что означает сопротивление 
	в \unit{ohm} проводника сечением 1 \unit{square mm} длинной 1 \unit{meter}. Теперь, если нам нужно найти
	сопротивление кабеля длинной 45 метров и сечением 2,5 \unit{square mm}, мы должны умножить удельное 
	сопротивление на 45 метров, и разделить на 2,5 \unit{square mm}}
	различных чистых металлов:

	%https://www.gpo.gov/fdsys/pkg/GOVPUB-C13-19188f38de43abe836138951e119bfd4/pdf/GOVPUB-C13-19188f38de43abe836138951e119bfd4.pdf
	%http://www.tibtech.com/conductivity.php

	\starttabulate[|l|l|]
		\HL
		\NC Металл     \NC \rho, \unit{ohm square mm per meter} \NC\NR
		\HL
		\NC Серебро     \NC 0,0159          \NC\NR
		\NC Медь        \NC 0,0157          \NC\NR
		\NC Золото      \NC 0,023           \NC\NR
		\NC Алюминий    \NC 0,0244          \NC\NR
		\NC Иридий      \NC 0,0474          \NC\NR
		\NC Вольфрам    \NC 0,053           \NC\NR
		\NC Молибден    \NC 0,054           \NC\NR
		\NC Цинк        \NC 0,059           \NC\NR
		\NC Никель      \NC 0,087           \NC\NR
		\NC Железо      \NC 0,098           \NC\NR
		\NC Платина     \NC 0,107           \NC\NR
		\NC Олово       \NC 0,12            \NC\NR
		\NC Свинец      \NC 0,192           \NC\NR
		\NC Титан       \NC 0,417           \NC\NR
		\NC Висмут      \NC 1,2             \NC\NR
		\HL
	\stoptabulate

	Видим лидеров нашего списка: Ag, Cu, Au, Al.\footnote{Обозначения химических элементов — металлов, названия сокращенные от латинских: Ag — Argentum (Серебро), Cu — Cuprum (Медь), Au — Aurum  (Золото), Al — Aluminium (Алюминий)}

	\startsubject[title={Серебро}]

		\materialdesc{Ag — Серебро.} Драгоценный металл.\footnote{Понятие «драгоценный
		металл» означает в том числе особые условия по работе с металлом,
		устанавливаемые законодательством.}
		Серебро — самый дешевый из драгоценных металлов, но, тем не менее, слишком дорог,
		чтобы массово делать из него провода. На 5\% лучшая электропроводность по сравнению с
		медью, при разнице в цене почти в 100 раз.

		\startsubsubject[title={Примеры применения}]

			\usagedesc{В виде покрытий проводников в СВЧ\footnote{СВЧ — Сверхвысокочастотной, 
			от 300 МГц до 300 ГГц (длинны волн 1\unit{meter} – 1\unit{mm})} технике.} Ток высокой частоты, из-за
			скин-эффекта\footnote{https://ru.wikipedia.org/wiki/Скин-эффект} в большей части течет по
			поверхности проводника, а не в его толще, поэтому тонкое покрытие высокочастотного волновода
			серебром дает бОльший прирост проводимости, чем покрытие
			серебром проводника для постоянного тока.

			\placefigure[here]{Волновод для СВЧ излучения, покрытый изнутри слоем серебра.
			}{\externalfigure[ag-wave.jpg][width=\textwidth]}

			\usagedesc{В сплавах контактных групп.} Контакты силовых, сигнальных реле, рубильников,
			выключателей чаще всего изготовлены из сплава с содержанием серебра. Переходное
			сопротивление такого контакта получается ниже медного, он меньше подвержен
			окислению. Так как контакт обычно миниатюрен, вклад этой малой добавки
			серебра в стоимость всего изделия незначителен. Хотя при утилизации большого количества реле,
			стоимость серебра делает
			целесообразным\footnote{http://www.weeerecycling.ru/2016/10/04/контакты-от-пускателей/}
			работу по отделению контактов в кучку для последующего аффинажа.

			\placefigure[here]{
			Контакты силового реле на 16 Ампер.
			Согласно документации производителя контакты содержат серебро и кадмий.
			}{\externalfigure[ag-relay.jpg][width=\textwidth]}

			\placefigure[here]{
			Различные реле. Верхнее реле имеет даже посеребренный корпус с
			характерной патиной.  Содержание драгметаллов в изделиях, выпущенных в СССР
			было указано в паспортах на изделия.
			}{\externalfigure[ag-relais.jpg][width=\textwidth]}

			\usagedesc{В качестве присадки в припоях.} Качественные припои (как твёрдые так и мягкие)
			часто содержат серебро.

			\usagedesc{Проводящие покрытия на диэлектриках.} Например, для получения
			контактной площадки на керамике, на неё наносится суспензия из серебряных
			частиц с последующим запеканием в печи (метод "вжигания").

			\usagedesc{Компонент электропроводящих клеев и красок.} Электропроводящие чернила
			часто содержат суспензию серебряных частиц. По мере высыхания таких чернил,
			растворитель испаряется, частицы в растворе оказываются всё ближе, слипаясь и
			создавая проводящие мостики, по которым может протекать ток. Хорошее
			видео\footnote{https://www.youtube.com/watch?v=dfNByi-rrO4}
			с рецептом по созданию таких чернил.

		\stopsubsubject

		\startsubsubject[title={Недостатки}]

			Несмотря на то, что серебро благородный металл, оно окисляется в среде с содержанием серы:
			\chemical{4Ag,+,2H_2S,+,O_2,->,2Ag_2S,+,2H_2O}

			Образуется темный налет — "патина". Также источником серы может служить резина,
			поэтому провод в резиновой изоляции и посеребренные контакты — плохое
			сочетание.

			Потемневшее серебро можно очистить
			химически.\footnote{http://and-ep.livejournal.com/6299.html}
			В отличии от чистки абразивными пастами (в том числе зубной пастой) это самый
			нежный способ чистки, не оставляющий царапин.

		\stopsubsubject

	\stopsubject


	\startsubject[title={Медь}]

		\materialdesc{Cu — медь.} Основной металл проводников тока. Обмотки
		электродвигателей, провода в изоляции, шины, гибкие проводники — чаще всего это
		именно медь. Медь нетрудно узнать по характерному красноватому цвету. Медь
		достаточно устойчива к коррозии.

		\startsubsubject[title={Примеры применения}]

			\usagedesc{Провода.} Основное применение меди в чистом виде. Любые добавки
			снижают электропроводность, поэтому сердцевина проводов обычно чистейшая
			медь.

			\placefigure[here]{
			Гибкие многожильные провода различного сечения.
			}{\externalfigure[cu-wires.jpg][width=\textwidth]}

			\usagedesc{Гибкие тоководы.} Если проводники для стационарных устройств можно
			в принципе изготовить из любого металла, то гибкие проводники делают почти
			всегда только из меди, алюминий для этих целей слишком ломкий. Содержат
			множество тоненьких медных жилок.

			\usagedesc{Теплоотводы.} Медь не только на 56\% лучше алюминия проводит ток,
			но ещё имеет почти вдвое лучшую теплопроводность. Из меди изготавливают тепловые
			трубки, радиаторы, теплораспределяющие пластины. Так как медь дороже алюминия,
			часто радиаторы делают составными, сердцевина из меди, а остальная часть из
			более дешевого алюминия.

			\placefigure[here]{
			Радиаторы охлаждения процессора. Центральный стержень изготовлен из меди,
			он хорошо отводит тепло от кристалла процессора, а алюминиевый радиатор с
			развитым оребрением уже охлаждает сам стержень.
			}{\externalfigure[cu-cooler.jpg][width=\textwidth]}

			\usagedesc{При изготовлении фольгированных печатных плат.} Печатные платы, в
			любом электронном устройстве изготовлены из пластины диэлектрика, на который
			наклеена медная фольга. Все соединения между элементами печатной платы
			выполнены дорожками из медной фольги.

			\usagedesc{Техника сверхвысокого вакуума.} Из металлов и сплавов только нержавеющая сталь
			и медь пригодны для камер сверхвысокого вакуума в таких приборах, как ускорители элементарных
			частиц или рентгеновские спектрометры. Все остальные металлы в вакууме слегка испаряются и портят
			вакуум.

			\usagedesc{Аноды рентгеновских трубок.} В рентгеноструктурном анализе требуется монохроматическое
			рентгеновское излучение. Его источником зачастую является облучаемая электронами медь
			(спектральная линия Cu~$K\alpha$), которая к тому же прекрасно отводит тепло. Если же требуется
			другое излучение (Co или Fe), его получают от маленького кусочка соответствующего металла на
			массивном медном теплоотводе. Такие аноды всегда охлаждаются проточной водой.

		\stopsubsubject

		\startsubsubject[title=Интересные факты о меди]
			\startitemize[intro]

				\item Медь — достаточно дорогой металл, поэтому недобросовестные производители стараются экономить
				на нем. Занижают сечение проводов (когда написано \unit{0,75 square mm}, а фактически 
				\unit{0,11 square mm}.)\footnote{http://serkov.su/blog/?p=2129} Окрашивают алюминий
				"под медь" в обмотках, внешне обмотка выглядит как медная, а стоит соскрести изоляцию —
				оказывается, что она сделана из алюминия. Этим грешат и китайские, и
				отечественные\footnote{http://www.owen.ru/forum/showthread.php?t=22703} производители,
				кабель сечением \unit{2,5 square mm} вполне может оказаться
				сечением \unit{2,3 square mm}, поэтому запас прочности и входной контроль не будут лишними.
				Разумеется, надежность контакта в электроарматуре жилы сечением \unit{2,3 square mm},
				рассчитанной на жилу \unit{2,5 square mm}, будет невысокой.

				\item Медь окрашивает пламя в зелёный цвет, это свойство использовали для
				обнаружения меди в руде, когда не был доступен химический анализ.
				Зеленый след в пламени — показатель наличия меди.\footnote{но не всегда, например в зеленый цвет пламя также окрашивают ионы бора}
				\placefigure[here]{
				Окрашивание пламени в зеленый цвет — показатель наличия меди.
				}{\externalfigure[cu-flame.jpg][width=\textwidth]}

				\item Медь — мягкий металл, но если добавить к меди хотя бы 10\% олова,
				получается твёрдый, упругий сплав — бронза. Именно освоение получения бронзы
				послужило названием к исторической эпохе — бронзовому веку\footnote{https://ru.wikipedia.org/wiki/Бронзовый_век}. 
				Добавка к меди бериллия дает бериллиевую бронзу — прочный упругий сплав, из которого изготавливают пружинящие контакты.
				 
				\item Медь — один из немногих мягких металлов с высокой температурой
				плавления, поэтому из меди изготавливают уплотнительные прокладки, например для
				высокотемпературной или вакуумной техники. Например, уплотнительная прокладка
				пробки картера двигателя автомобиля.

				\item При механической обработке (например волочении) медь уплотняется и
				становится жёсткой. Для восстановления исходной мягкости и пластичности медь
				"отжигают" в защитной атмосфере, нагревая до 500–\unit{700 celsius} и плавно охлаждая.
				Поэтому некоторые медные изделия твёрдые, а некоторые мягкие, например медные трубы. 
				Неотожженая медь имеет более высокую твердость, упругость, но на 3–5\% меньшую электропроводность.

				\item Медь не даёт искр. Для работы во взрывоопасных местах, например на газопроводе,
				используют искробезопасный инструмент, стальной инструмент покрытый слоем меди
				или инструмент изготовленный из сплавов меди — бронз. Если таким инструментом
				случайно чиркнуть по стальной поверхности он не даст опасных искр.

				\item Так как температурный коэффициент сопротивления для чистой меди
				известен, из меди изготавливают термометры сопротивления (тип ТСМ — Термометр
				Сопротивления Медный, есть еще ТСП — Термометр Сопротивления Платиновый).
				Термометр сопротивления — это точно изготовленный резистор, навитый из медной
				проволоки. Измерив его сопротивление, можно по таблице или по формуле
				определить его температуру достаточно точно.

				\item В каталогах производителей аудио кабелей можно увидеть упоминания "Бескислородной меди".
				(OFC — Oxygen-free copper). Практически всегда это всего лишь маркетинговый прием для обоснования
				неадекватно высокой цены продукции. Кислород ухудшает не только проводящие свойства меди, но и ее механические свойства,
				поэтому медь при производстве подвергают очистке от примесей путем электролиза, иначе из нее становится трудно получить
				проволоку волочением. Поэтому можете быть уверенны, любой провод и кабель, удовлетворяющий требованиям ГОСТ
				содержит медь достаточного качества.
				% Калинин Н Н Электрорадиоматериалы 1981, 51 стр
				% ГОСТ Р 53803-2010 - медная катанка электротехническая


			\stopitemize
		\stopsubsubject

	\stopsubject

	\startsubject[title=Алюминий]

		\materialdesc{Al — Алюминий.} "Крылатый металл" четвертый по проводимости после
		серебра, золота и меди. Алюминий хоть и проводит ток почти в полтора раза хуже меди, но
		он легче в 3,4 раза и в три раза дешевле. А если посчитать проводимость, то
		эквивалентный медному проводник из алюминия будет дешевле в 6,5 раз! Алюминий
		бы вытеснил медь как проводник везде, если бы не пара его противных свойств,
		но об этом в недостатках.

		Чистый алюминий, как и чистое железо, в технике практически не применяется. 
		Любой "алюминиевый" предмет состоит из какого-нибудь сплава алюминия. Сплавы могут
		содержать кремний, магний, медь, цинк и другие металлы. Их свойства отличаются очень сильно, и это
		необходимо учитывать при обработке. Ниже перечислены несколько самых распространенных марок алюминия:
		\footnote{Даны марки сплавов согласно номенклатуре Американской Алюминиевой Ассоциации (АА), Первая цифра - серия марок сплава, в зависимости от того, какой легирующей добавки больше, остальные цифры обычно не соотносятся с концентрацией и необходимо обращение к справочнику.}

		\startitemize
			\item 1199. Чистый 99,99\% алюминий. Бывает почти исключительно в виде фольги.
			\item 1050 и 1060. Чистый  алюминий 99,5\% и 99,6\% соответственно. Из-за высокой теплопроводности иногда
			  используется как материал для радиаторов. Мягок, легко гнется. Провода, пищевая фольга, посуда.
			\item 6061 и 6082. Сплавы: 6061: Si 0,6\%, Mg 1,0\%, Cu 0,28\%, 6082: Si, Mg, Mn. Первый более
			  распространен в США, второй — в Европе. Легко точить, фрезеровать. Наилучший материал для
			  самоделок. Прочен. Легко поддается сварке, паяется твердыми припоями. Легко анодируется.
			  Плохо гнется. Не годится для литья.
			\item 6060. Состав: Mg, Si. Более мягок, чем 6061 и 6082, при обработке резанием слегка
			  "пластилиновый", за что его не любят токари. Распространен и дешев, других особых преимуществ
			  не имеет. Дешевый алюминиевый профиль из непонятного сплава имеет хорошие шансы оказаться им.
			\item 5083. Сплав с магнием (>4\% Mg). Отличная коррозионная стойкость, устойчив в морской воде.
			  Один из лучших вариантов для деталей, работающих под дождем. Тоже может встретиться в магазине
			  стройматериалов, наряду с другими подобными марками.
			\item 44400, он же "силумин". Сплав с большим процентом кремния (Si >8\%). Литейный.
			  Низкая температура плавления, при пайке твердыми припоями риск расплавить саму деталь. Хрупок,
			  при изгибе ломается. На изломе видны характерные кристаллы.
			\item 7075. \NC 2,1–2,9\% Mg, 5,1–6,1\% Zn, 1,2–1,6\% Cu. Очень своеобразный сплав, отличается даже
			  цветом (пленка окислов слегка золотистая). Неожиданно твердый для алюминия, по твердости сравним
			  с мягкой сталью. Плохо анодируется. Не паяется вообще. Не предназначен для сварки. Не гнется и не
			  куется вообще. Не годится для литья. Резанием обрабатывается отлично, прекрасно полируется.
			  Хорош для ответственных деталей. Используется для винтов в велосипедах, в оружии (материал многих
			  деталей винтовки M16).
			% http://www.thefabricator.com/article/aluminumwelding/aluminum-workshop-what-s-so-bad-about-welding-7075-2024-
			% про сварку 7075


		\stopitemize


		Относительно невысокая температура плавления (\unit{660 celsius} у чистого, меньше
		\unit{600 celsius} у литейных сплавов) алюминия делает возможным отливку деталей в песочные формы
		в условиях гаража/мастерской. Однако многие марки алюминия не годятся для литья.

		\startsubsubject[title={Примеры применения}]

			\usagedesc{Провода.} Алюминий дешев, поэтому толстые силовые кабели,
			СИП\footnote{СИП — Самонесущий Изолированный Провод},
			ЛЭП\footnote{ЛЭП — Линия ЭлектроПередач} выгодно делать из алюминия. В старых
			домах квартирная проводка сделана алюминиевым проводом (с 2001 года ПУЭ
			запрещает в квартирах использовать алюминиевый провод, только медный,
			см ниже\footnote{Правила устройства электроустановок, 7-е издание, п.~7.1.34}).
			Также алюминий не используется как проводник в ответственных применениях.

			\placefigure[here]{
			Слева старый алюминиевый провод. Справа алюминиевые кабели различного сечения,
			пригодные для укладки в грунт. В частности, кабелем справа был подключен к
			электроэнергии целый этаж здания. Кабель помимо наружной резиновой оболочки
			имеет бронирующую стальную ленту для защиты нижележащей изоляции от повреждений,
			к примеру, лопатой при раскопке.
			}{\externalfigure[al-wires.jpg][width=\textwidth]}

			\usagedesc{Теплоотводы.} Не только домашние батареи делают из алюминия, но и
			радиаторы у микросхем, процессоров.

			\placefigure[here]{
			Различные алюминиевые радиаторы.
			}{\externalfigure[al-cooler.jpg][width=\textwidth]}

			\usagedesc{Корпуса приборов.} Корпус жёсткого диска в вашем компьютере отлит
			из алюминиевого сплава. Небольшая добавка кремния улучшает прочностные качества
			алюминия, сплав силумин:\footnote{https://ru.wikipedia.org/wiki/Силумин}
			это корпуса жёстких дисков, бытовых приборов, редукторов и т.~д.
			Анодированный алюминий (алюминий, у которого электрохимическим путем окисная
			пленка на поверхности сделана потолще и прочнее) хорошо окрашивается и просто красив.
			Окисная пленка (\chemical{Al_2O_3} — из того  же вещества состоят драгоценные
			камни рубины и сапфиры) достаточно твёрдая и износостойкая, но, к сожалению,
			алюминий под ней мягок, и при сильном воздействии ломается как лёд на воде.

			\usagedesc{Экраны.} Электромагнитное экранирование часто делается из алюминиевой
			фольги или тонкой алюминиевой жести. Можете провести простой эксперимент, мобильный
			телефон завернутый в фольгу потеряет сеть — он будет заэкранирован.

			\usagedesc{Отражающее покрытие у зеркал.} Тонкая пленка алюминия на стекле
			отражает 89\%\footnote{значения примерные, точное значение зависит от длины
			волны и типа покрытия} падающего света (Серебро 98\%, но на воздухе темнеет
			из-за сернистых соединений). Любой лазерный принтер содержит вращающееся зеркало,
			покрытое тонким слоем алюминия.

			\placefigure[here]{
			Зеркала от оптической системы планшетного сканера. Обратите внимание,
			оптические зеркала имеют металлизацию стекла снаружи, в отличии от привычных
			бытовых зеркал, где отражающее покрытие для защиты за стеклом. Бытовые зеркала
			дают двойное отражение — от поверхности стекла и от отражающего покрытия,
			что не так критично в быту, как защищенность отражающего покрытия.
			}{\externalfigure[al-mirror.jpg][width=\textwidth]}

			\usagedesc{Электроды обкладок конденсаторов.} Алюминиевая фольга, разделенная
			слоем диэлектрика и туго свернутая в цилиндр — часть электрических конденсаторов
			(впрочем, для уменьшения габаритов конденсаторов фольгу заменяют алюминиевым напылением).
			Тот факт, что пленка оксида алюминия тонкая, прочная и не проводит ток, используется
			в электролитических конденсаторах, обладающими огромными для своих габаритов
			электрическими емкостями.

			\usagedesc{Микропровод.} Тончайшей проволокой из алюминия подключают кристаллы микросхемы к
			выводам корпуса. Также может использоваться медная и золотая проволока.



		\stopsubsubject

		\startsubsubject[title={Недостатки}]

			\problemdesc{Алюминий — металл активный,} но на воздухе покрывается оксидной пленкой,
			которая предохраняет металл от разрушения и скрывает его активную натуру. Если
			не дать алюминию формировать стабильную защитную пленку, например капелькой
			ртути, алюминий активно реагирует\footnote{https://www.youtube.com/watch?v=Z7Ilxsu-JlY}
			с водой. В щелочной среде алюминий  растворяется, попробуйте залить
			алюминиевую фольгу средством для прочистки труб — реакция будет бурная, с выделением
			взрывоопасного водорода.  Химическая активность алюминия, в паре с большой
			разницей в электроотрицательности с медью делает невозможным прямое соединение
			проводов из этих двух металлов. В  присутствии влаги (а она в воздухе есть почти всегда)
			начинает протекать гальваническая коррозия\footnote{http://lab115.com/?p=8}
			с разрушением алюминия.

			\placefigure[here]{Два идентичных трансформатора от микроволновых печей. Левый вышел из строя по причине алюминиевых обмоток - отгорел провод от контакта - алюминий плохо паяется мягкими припоями, попытка обеспечить контакт также как и у медного провода привела к поломке.
			}\startcombination[2*1]
			{\externalfigure[MOTs.jpg][width=0.45\textwidth]}
			{\externalfigure[MOT_CUT.jpg][width=0.45\textwidth]}
			\stopcombination

			\problemdesc{Алюминий ползуч.} Если алюминиевый провод очень сильно сжать, он деформируется и
			сохранит новую форму — это называется "пластическая деформация". Если сжать его
			не так сильно, чтобы он не деформировался, но оставить под нагрузкой надолго —
			алюминий начнет "ползти" меняя форму постепенно. Это пакостное свойство ведет к
			тому, что хорошо затянутая клемма с алюминиевым проводом спустя 5–10–20 лет
			постепенно ослабнет и будет болтаться, не обеспечивая былого электрического
			контакта. Это одна из причин, почему ПУЭ запрещает тонкий алюминиевый провод
			для разводки электроэнергии по конечным потребителям в зданиях.
			\footnote{См п. 7.1.34 ПУЭ 7 издания} В промышленности не
			сложно обеспечить регламент — так называемая "протяжка" щитка, когда электрик
			периодически (1–2 раза в год) проверяет затяжку всех клемм в щитке. В домашних же условиях,
			обычно пока розетка с дымом не сгорит — никто и не озаботится качеством
			контакта. А плохой контакт — причина пожаров.

			\problemdesc{Алюминий, по сравнению с медью, менее пластичный,}
			риска от ножа на жиле, при сьёме изоляции с провода быстрее приведет к
			сломавшейся жиле, чем у меди, поэтому изоляцию с алюминиевых проводов надо
			счищать как с карандаша, под углом, а не в торец.

		\stopsubsubject

		\startsubsubject[title=Интересные факты об алюминии]
			\startitemize[intro]

				\item Алюминий — хороший восстановитель, что используется для восстановления
				других металлов, например титана из состояния диоксида.
				Теодор Грей\footnote{Настоятельно рекомендую книги Теодора Грея «Элементы.
				Путеводитель по периодической таблице», «Научные опыты с периодической
				таблицей», «Эксперименты. Опыты с периодической таблицей». Они очень хорошо
				сделаны визуально, и опыты в них не приторно безопасные, как в большинстве
				современных пособий, могут и бабахнуть.}
				в домашних условиях проводил\footnote{http://graysci.com/chapter-five/homemade-titanium/}
				такой опыт. В смеси с окислом железа алюминиевая пудра
				образует термит\footnote{https://ru.wikipedia.org/wiki/Термитная_смесь}
				— адскую смесь, которая горит разогреваясь до 2400°С при этом
				восстанавливается железо и весело стекает вниз, что используется для сварки
				рельсов, иным способом такой кусок железа качественно и быстро не прогреть.
				Термитные карандаши позволяют в полевых условиях сваривать провода, а бравый
				спецназовец термитной горелкой 
				пережжет\footnote{http://www.empi-inc.com/tec_torch.html} дужку самого крепкого замка.

				\item Чтобы сделать бисквит нежным и воздушным используется пекарский
				порошок\footnote{https://ru.wikipedia.org/wiki/Пекарский_порошок}.
				Такой же порошок есть для того, что бы сделать пористым бетон — Алюминий + щелочь.
				\footnote{\hyphenatedurl{http://litebeton.ru/statya/alyuminievye-pudry-pasty-dlya-proizvodstva-gazobetona-gidrofilnye-dobavki-dlya-pudr-obzor}}

				\item Алюминий — активный металл, но он быстро покрывается окисной пленкой,
				которая защищает его от разрушения. Рубин, сапфир, корунд — это всё названия
				одного и того же вещества — оксида алюминия \chemical{Al_2O_3} Белые точильные
				круги и бруски состоят из электрокорунда — оксида алюминия.

				\item Можно убедиться в активности алюминия простым опытом. Нарежьте
				алюминиевую фольгу в стакан, добавьте медный купорос и поваренную соль, залейте
				холодной водой. Спустя некоторое время смесь закипит, алюминий будет
				окисляться, восстанавливая медь, с выделением тепла.

				\item Алюминий неплохо поддается экструзии. Корпуса приборов из нарезанного и
				обработанного экструдированного профиля значительно дешевле литых.

				\placefigure[here]{
				Алюминиевый корпус внешнего аккумулятора для телефона. Экструдированный
				анодированный окрашенный профиль.
				}{\externalfigure[al-extrusion.jpg][width=\textwidth]}

				\item Алюминий весьма посредственно
				паяется\footnote{http://aluminium-guide.ru/myagkie-pripoi-dlya-alyuminiya/}
				мягкими (оловянно-свинцовыми) припоями, неплохо паяется цинковыми припоями.
				При конструировании приборов это стоит помнить, соединить провод с алюминиевым
				шасси проще прикрутив винтом к запрессованной стойке, чем припаять. В твердых марках
				алюминия (6061, 6082, 7075) можно нарезать резьбу для винта непосредственно.

				\item Алюминий можно сваривать аргоновой сваркой, но качественный шов получается только
				при TIG-сварке на {\it переменном} токе. Непрерывная смена полярности измельчает пленку
				окислов, которая в противном случае может попасть в шов. Учитывайте это при выборе сварочного
				аппарата для мастерской, если вам может потребоваться варить и алюминий.

			\stopitemize
		\stopsubsubject

		Еще раз важное замечание. {\bf Алюминиевые и медные проводники напрямую соединять нельзя!
		Для соединения проводников из меди и алюминия используйте промежуточный металл,
		например, стальную клемму.}

		\startsubsubject[title={Источники}]

			В крупных строительных магазинах (OBI, Leroy Merlin, Castorama) обычно есть в
			наличии алюминиевый профиль разных размеров и форм. Неплохим источником может
			послужить штампованная алюминиевая посуда — она очень дешева и существует
			разных форм. Но обратите внимание на марки. Если нужен 6061 и тем более 7075,
			придется покупать его у фирмы, специализирующейся на продажах металлов.

		\stopsubsubject

	\stopsubject


	\startsubject[title={Железо}]

		\materialdesc{Fe — железо.} Основной конструкционный материал в промышленности
		используется также и в электротехнике. Плохая, по сравнению с медью,
		электропроводность компенсируется очень низкой ценой. И, что важнее в России,
		меньшей привлекательностью для охотников за металлом, заземление из толстой
		ржавой трубы простоит без охраны дольше красивой медной шины.

		В технике железо применяется почти исключительно в виде сплавов с углеродом — чугуна и сталей.
		Свойства сталей разных марок весьма различны: от мягких до твердых инструментальных.

		\startsubsubject[title={Примеры применения}]

			\usagedesc{Метизы.} Винты, шайбы, гайки из стали изготавливаются огромными
			количествами на специально разработанном для этого оборудовании. Метизы из
			других металлов встречаются очень редко и значительно дороже. Поэтому, в
			большинстве случаев, медный наконечник медного провода будет притянут к медной
			же шине стальным болтом (или омеднённым). Также важным является высокая прочность стали, медный
			болт не затянуть с усилием стального. Обратите внимание на цифры на головке болта:
			они обозначают его прочность. Чем больше число, тем сильнее можно затягивать болт.

			\usagedesc{Клеммные колодки, соединители.} Соединители типа "орех" содержат
			стальные пластинки с защитным покрытием от коррозии. Также, применение стали
			необходимо для предотвращения гальванической коррозии при соединении медных
			и алюминиевых проводов.

			\placefigure[here]{
			Соединитель «орех». Внутри пластиковой оболочки комплект стальных пластин с
			винтами, позволяет сделать ответвление от жилы кабеля не разрезая саму жилу.
			Также позволяет перейти от алюминиевой жилы на медную.
			}{\externalfigure[fe-connector.jpg][width=\textwidth]}

			\usagedesc{Контуры заземления.} Требования электробезопасности обязывают
			предусматривать заземление. Часто, в промышленных условиях, заземляющую шину
			изготавливают из стального проката, закрепленного по периметру стены. Плохая
			электропроводность стали компенсируется большим сечением проводника. Во многих
			случаях правила безопасности и стандарты предписывают делать детали заземления
			именно из стали по соображениям механической прочности.

			\placefigure[here]{
			Стальная полоса, огибающая колонну — шина заземления.
			}{\externalfigure[fe-grounding.jpg][width=\textwidth]}


			\usagedesc{Широко используются магнитные свойства стали}
			— из стальных пластин собирают сердечники трансформаторов, дросселей.

		\stopsubsubject

		\startsubsubject[title={Недостатки}]

			\problemdesc{Коррозия.} Железо ржавеет, при этом плотность ржавчины ниже
			плотности исходного железа, из-за этого конструкция
			распухает\footnote{https://en.wikipedia.org/wiki/Rust\#/media/File:Rust_wedge.jpg}. Поэтому железо
			покрывают защитными покрытиями — оцинковка, лужение, хромирование,
			окраска и т.д. Разные марки стали подвержены коррозии в разной степени, причем по
			закону подлости сильнее всего ржавеют именно те, которые легче всего обрабатываются на станках.

		\stopsubsubject

	\stopsubject


	\startsubject[title={Золото}]

		\materialdesc{Au — Золото.} Самый бестолковый драгоценный металл. Имеет меньше
		всего применений в технике из всех драгоценных металлов, но является символом
		богатства. На удивление дороже платины (2017 г.), что лишено здравого смысла и
		является лишь результатом спекуляций.

		\startsubsubject[title={Примеры применения}]

			\usagedesc{Покрытия контактов.} Благодаря тому, что золото на воздухе не
			окисляется, контакты покрывают очень тонким слоем золота. В силу мягкости золота 
			покрытие не подходит для контактов много работающих на истирание, в таких случаях
			подбирают более твердые покрытия (например родиевые), или легируют золото.

			\placefigure[here]{
			Золотое покрытие на различных электронных компонентах: покрытие на
			контактах платы для установки в слот, покрытие на контактах мембранных кнопок
			мобильного телефона, покрытие на штырьках процессора.
			}{\externalfigure[au-coating.jpg][width=\textwidth]}

			\usagedesc{Защита от коррозии.} В некоторых ответственных применениях
			используется золотое покрытие для защиты проводников от коррозии (в основном —
			военка). Когда-то покрытие золотом являлось единственным способом защитить детали
			электроники от коррозии в условиях джунглей, поэтому у многих старых радиодеталей
			позолочены даже корпуса. А сейчас обычно просто заливают плату компаундом в "кирпич".

		\stopsubsubject

		\startsubsubject[title={Интересные факты о золоте}]

			Золото — один из четырех металлов, имеющий оттенок в не окислившемся
			состоянии. Все остальные металлы белые (желтоватый цвет имеют золото и цезий,
			медь - красноватая и в сплавах золотистая, осмий имеет голубой отлив).

			Плотность золота отличается от плотности вольфрама незначительно (\unit{19,32 gram / cubic cm}
			у золота, \unit{19,25 gram / cubic cm}), этим пользуются для
			подделки\footnote{\hyphenatedurl{http://www.zerohedge.com/news/tungsten-filled-10-oz-gold-bar-found-middle-manhattans-jewelry-district}}
			золотых слитков, покрывая вольфрамовый слиток слоем золота. Некоторые теории заговора утверждают, что возможно это одна из
			причин, почему США никому не дают проверить подлинность их золотого
			запаса. И, возможно, поэтому они отдали Германии их золото не
			сразу\footnote{https://utro.ru/articles/2016/03/21/1275137.shtml}.

			Можно извлечь\footnote{https://geektimes.ru/post/258242/} золото химически из горы старой
			электроники, но это не всегда экономически целесообразно и
			преследуется по закону (ст. 191, 192 УК РФ).

			Бестолковость золота требует пояснений. Представим добычу благородных металлов в 2016 году.
			Из всей добытой платины 64\% потребила промышленность. \footnote{Здесь и далее цифры примерные, усредненные по нескольким источникам}. 
			Из всего добытого серебра 68\% потребила промышленность. 
			Из всего добытого палладия 96\% потребила промышленность.
			Из всего добытого золота всего 10\% потребила промышленность. Остальное ушло на украшения и на слитки в сейфах.

			%https://www.gold.org/research/gold-demand-trends/gold-demand-trends-full-year-2016#package
			%https://en.wikipedia.org/wiki/List_of_countries_by_gold_production
			%https://www.gold.org/research/gold-demand-trends/gold-demand-trends-full-year-2016/technology
			%https://www.gold.org/research/gold-demand-trends/gold-demand-trends-full-year-2016/jewellery
			%http://www.ereport.ru/articles/commod/gold.htm
			%http://www.platinum.matthey.com/documents/new-item/pgm%20market%20reports/pgm_market_report_november_2016.pdf
			%https://minerals.usgs.gov/minerals/pubs/commodity/platinum/550400.pdf
			%https://www.platinuminvestment.com/files/Platinum_Fundamentals_forecast_Glaux_2016.pdf
			%http://www.platinum.matthey.com/documents/new-item/pgm%20market%20reports/pgm-market-report-may-2016.pdf

			%http://or2014.nornik.ru/en/apps/53-application.pdf
			%https://www.statista.com/statistics/253345/global-silver-demand-by-purpose/
			%http://www.scotiamocatta.com/scpt/scotiamocatta/prec/Silver_Forecast_2016.pdf
			%https://www.monex.com/cpm/2016CPMpgm.pdf



		\stopsubsubject

	\stopsubject


	\startsubject[title={Никель}]

		\materialdesc{Ni — Никель.} Замечательный металл, но в электронной технике
		основное применение в виде покрытий, как в чистом виде, так и в паре с хромом.

		\startsubsubject[title={Примеры применения}]

			\usagedesc{Покрытие контактов.} Наносится на медь, пластик для надежного
			контакта и для декоративных целей. Жадные китайцы иногда вообще делают контакты
			из пластмассы, покрывая сверху слоем никеля и хрома, внешне  выглядит
			нормальным, даже как то работает, но ни о какой надежности речи не идет.

			\placefigure[here]{
			Различные разъемы, покрытые никелем для надежного контакта.
			}{\externalfigure[ni-coating.jpg][width=\textwidth]}

			\placefigure[here]{
			У разъема справа для экономии металла сердцевина штыря сделана
			полой с заливкой пластиком. Латунная никелированная трубочка, из которой сделан
			штырь, не самый худший вариант.
			}{\externalfigure[ni-connectors.jpg][width=\textwidth]}

			\usagedesc{Тоководы у ламп.} Сплав Платинит (46\% Ni, 0,15\% C, остальное — Fe)
			не содержит платины, но имеет очень близкое к платине значение линейного
			температурного расширения (и близкое к стеклу), что позволяет делать из него надежные электроды,
			проходящие через стекло. Для аналогичных целей используют сплав Ковар (29\% Ni, 17\% Co, 54\% Fe).
			Такие электроды при изменении температуры не вызывают растрескивания стекла и потерю герметичности.
			Место сплавления стекла с этими сплавами имеет красноватый оттенок что ошибочно воспринимается за медь.

			\usagedesc{Промежуточные защитные слои.} Для защиты от коррозии, взаимной диффузии
			металлов при создании покрытий, могут формироваться промежуточные слои из никеля.
			Например при покрытии меди слоем золота, если не предусмотрен разделительный слой из никеля,
			золотое покрытие со временем из-за диффузии растворится в меди и потеряет целостность.
			Жала современных паяльников защищены слоем никеля, так как жало из голой меди медленно
			растворяется в олове, теряя форму.

		\stopsubsubject

	\stopsubject


	\startsubject[title={Вольфрам}]

		\materialdesc{W — Вольфрам.} Тугоплавкий металл, температура плавления \unit{3422 celsius},
		что определяет основное его использование — нити накала и электроды.

		\startsubsubject[title={Примеры применения}]

			\usagedesc{Нити накала.} В лампах накаливания, в галогеновых лампах спираль
			изготовлена из вольфрама, нагревается электрическим током до белого каления,
			при этом сохраняя свою форму. Также катоды в радиолампах изготавливаются из
			вольфрама, но раскаливаются не до таких высоких температур, как осветительные
			лампы, специальное покрытие на катоде позволяет протекать термоэлектронной
			эмиссии при невысоких температурах.

			\placefigure[here]{
			Нить накаливания этой галогеновой лампы изготовлена из вольфрама. Галоген, обычно пары иода,
			химически связывает испаряющийся с нити вольфрам и возвращает его на нить, что
			позволяет повысить температуру накала спирали и уменьшить габарит лампы без страха,
			что вольфрам постепенно осядет на стенках колбы.
			}{\externalfigure[w-filament.jpg][width=\textwidth]}

			\placefigure[here]{
			Мощная лампа накаливания от проектора. Даже тугоплавкий вольфрам со временем
			испаряется и оседает на стенках колбы в виде темного налета. Данного недостатка
			лишены галогеновые лампы.
			}{\externalfigure[w-lamp.jpg][width=\textwidth]}

			\usagedesc{Электроды дуговых ламп и сварочные электроды.} В ксеноновых дуговых лампах,
			ртутных дуговых лампах, электроды должны выдерживать температуру электрической
			дуги, при этом не расплавляясь и не изменяя своей формы, что под силу только
			вольфраму. Также электроды для сварки неплавящимся электродом изготовлены из
			вольфрама (TIG сварка).

			\usagedesc{Аноды рентгеновских трубок.} Поток электронов от катода в рентгеновской
			трубке, разогнанный высоким напряжением тормозится бомбардируя анод, очень сильно
			нагревая его, поэтому такие аноды, особенно если они не имеют водяного охлаждения,
			зачастую изготавливаются из вольфрама. Однако в физических лабораториях часто применяют
			и аноды из меди или кобальта в связи с особенностями спектра рентгеновского излучения от
			таких анодов.

		\stopsubsubject

		\startsubsubject[title={Источники}]

			Вольфрам — не очень пластичный материал, поэтому спиральку из лампы накаливания
			вряд ли удастся выпрямить и использовать по своему разумению. Если вдруг понадобится
			вольфрамовый стержень — вам пригодится любой магазин по сварочному делу, электрод для
			TIG-горелки без содержания лантана и других присадок. Проволоку для нитей накала самодельной
			техники нетрудно купить на eBay.

			Цветовая маркировка электродов:
			\startitemize[intro]
				\item Зеленый — чистый вольфрам.
				\item Красный, оранжевый — вольфрам + торий ({\it Радиоактивно! Не шлифовать, не резать - пыль опасна!}).
				\item Голубой — вольфрам + сложная смесь.
				\item Черный, желтый, синий — вольфрам + лантан.
				\item Серый — вольфрам + церий.
				\item Белый — вольфрам + цирконий.
			\stopitemize

		\stopsubsubject

	\stopsubject


	\startsubject[title={Ртуть}]

		\materialdesc{Hg — Ртуть.} При комнатной температуре — блестящий, собирающийся в шарики
		жидкий металл. По экологическим соображениям использование ртути сокращается, но она
		широко использовалась в старых приборах, поэтому заслуживает упоминания.

		Как и большинство металлов, ртуть образует сплавы. Но ртуть, будучи жидкой при комнатной температуре,
		способна сплавляться с металлами без дополнительного нагревания, растворять их. Растворенный в
		ртути металл, сплав металла с ртутью называется {\it "амальгама"}.

		\startsubsubject[title={Примеры применения}]

			\usagedesc{Жидкий контакт} в датчиках положения, ртутных электроконтактных термометрах.

			\placefigure[here]{
			Различные ртутные приборы. Слева — мощный ртутный переключатель, замыкающий/размыкающий
			цепь при наклоне. Ниже на чёрных платках — аналогичные китайские ртутные
			переключатели — датчики положения из детского набора с Arduino. Сверху — колба
			ртутного электроконтактного термометра. В стекло вплавлены проволочки так, что
			при температуре 70°С столбик ртути в капилляре замыкает цепь (температура
			указана на корпусе).
			}{\externalfigure[hg-devices.jpg][width=\textwidth]}

			\usagedesc{В термометрах.} Низкая температура замерзания, высокая температура кипения и большой
			коэффициент теплового расширения делают ртуть одним из самых удобных веществ для лабораторных
			и медицинских термометров. В бытовых термометрах ртуть уже очень давно не используется.

			\usagedesc{В манометрах и барометрах.} Ртуть тяжелая, поэтому для уравновешивания атмосферного
			давления достаточно 70–\unit{80 cm} высоты столбика ртути. Хотя ртутные барометры в основном
			вышли из употребления, единицы измерения давления "миллиметр ртутного столба", а в вакуумной
			технике — "микрон ртутного столба" и "торр" (округленный вариант мм.~рт.~ст.) используются
			и по сей день. Нормальным атмосферным давлением считается 760~мм.~рт.~ст.

			\usagedesc{В нормальных элементах.} Батарейка\footnote{Попытка запитать от такой батарейки самоделку обернется провалом - 
			батарейка имеет большое внутренее сопротивление (порядка единиц \unit{kilo ohm}) и не предназначена отдавать токи больше 
			сотых долей микроампера, да и то с перерывами.}  с электродами из жидкой ртути, в которой растворены
			сульфаты ртути и кадмия, имеет ЭДС, известную и стабильную до единиц микровольт
			(теоретически \unit{1,018636 volt} при \unit{20 celsius}). Такие элементы до сих пор
			используются в метрологии в качестве опорных источников напряжения, хотя и вытесняются полупроводниковыми
			схемами. Сосуд с ртутью в нормальном элементе запаян, однако он стеклянный, и ртути в нем много.
			Поэтому будьте осторожны, если найдете где-нибудь круглую железную банку с бакелитовой крышкой,
			клеммами и надписью "нормальный элемент" на бакелите. Внутри у нее — стеклянная колба с весьма
			опасными веществами.

			\placefigure[here]{Элемент нормальный насыщенный, НЭ-65, класс точности 0,005. Внешний вид корпуса нормальных элементов может различаться. Справа - содержимое корпуса, видна ртуть в нижней части колб. Такие элементы должны утилизироваться специализированной организацией.
			}\startcombination[1*2]
			{\externalfigure[Weston_cell.jpg][width=0.45\textwidth]}
			{\externalfigure[Weston_cell_open.jpg][width=0.45\textwidth]}
			\stopcombination

			\usagedesc{В диффузионных вакуумных насосах.} Струя ртутного пара, выходящая из сопла с большой
			скоростью, захватывает молекулы воздуха и вытягивает их из откачиваемого объема. Затем ртутный
			пар конденсируется за счет охлаждения жидким азотом и используется снова. Насосы такого типа
			когда-то использовались для откачки радиоламп. Сейчас вместо ртути используются нетоксичные
			и не требующие жидкого азота силиконовые масла, но в некоторых лабораториях до сих пор можно
			найти старые ртутные системы.

			\usagedesc{Пары ртути — рабочий газ люминесцентных ламп.} Несмотря на то, что люминесцентная
			лампа должна содержать небольшое количество ртути, в некоторых лампах ртути добавлено от души,
			и видно, как в колбе перекатывается шарик ртути. Пары ртути при возбуждении их электрическим
			током излучают яркий свет, преимущественно в синей и ультрафиолетовой области. Помимо них в
			спектре ртути есть яркие желтый и зеленый дублеты, по наличию которых ртутную лампу легко
			отличить от любой другой, посмотрев на нее через призму или отражение в компакт-диске.
			Специальная ртутная лампа в лабораториях используется как источник зеленого света с известной
			длиной волны.

			\usagedesc{В мощных тиратронах и ртутных выпрямителях.} Используется так же, как и в ртутных
			лампах. Мощные ртутные вентили широко использовались для питания локомотивов на железных дорогах
			и в других подобных задачах до появления полупроводниковых приборов.

			\usagedesc{Как растворитель для металлов} при выделении золота и платины из сырья амальгамацией
			и в производстве зеркал. Ртуть выпаривается, металл остается. Иногда этот процесс неправильно
			называют "аффинаж", путая его с совершенно другим способом очистки драгметаллов.

			\usagedesc{В ртутных счетчиках времени наработки.} В старой технике ртутный капиллярный
			кулономер использовался как счетчик часов, которые проработал прибор. Гениальная по простоте
			и надёжности конструкция. 

			\placefigure[here]{
			Ртутный счетчик времени наработки от осцилографа. В углу показан крупным планом разрыв столбика ртути в капилляре каплей электролита.
			Ртуть под действием тока растворяется на одном конце капли и восстанавливается на другом, в результате этот разрыв движется по капилляру на расстояние,
			пропорциональное пропущенному через капилляр количеству электричества. Благодарю Александра @Talion_amur за предоставленный образец.
			}{\externalfigure[hg-timer.jpg][width=\textwidth]}

			\usagedesc{В амальгамных зубных пломбах.} Встречаются и по сей день, особенно в США.

		\stopsubsubject

		\startsubsubject[title={Токсичность}]
			Все изделия, содержащие ртуть, должны утилизироваться специализированной службой.
			Недопустимо выбрасывать их с бытовым мусором во избежание скопления ртути на свалке.

			Все разливы ртути должны быть собраны, а поверхности демеркуризованы\footnote{от англ. Меркьюри - ртуть (в честь бога Меркурия), 
			приставка де- означает удаление, дословно демеркуризация - "очистка от ртути"}.	
			Ртуть хорошо испаряется\footnote{https://www.youtube.com/watch?v=TZprgqZh4IE} при комнатной температуре,
			поэтому закатившийся в щель шарик ртути долгое время будет отравлять воздух.
			  
			%TODO разлить ртуть и попробовать все методы, проверить серу, свинец, перманганат, хлорное железо, медную губку.
			  
			Если у вас разбилось изделие с ртутью, то предпринимайте следующие действия:

			\startitemize [intro, n]

				\item Откройте форточки и обеспечьте проветривание.

				\item Вызовите специализированную службу демеркуризации в вашем городе. Профессионалы не только
				грамотно уберут ртуть, но также и произведут замеры концентрации паров ртути в помещении.

				\item Если вдруг в вашем городе не оказалось службы демеркуризации, вы находитесь вдали от цивилизации
				то процесс демеркуризации придется продолжить самостоятельно. 

				\item Соберите видимые шарики ртути в герметичную тару. Их удобно собирать вместе
				при помощи двух хорошо обрезанных листов бумаги, сливая шарики в подготовленную тару. Мельчайшие шарики ртути из 
				щелей можно вытянуть при помощи спиринцовки, или щетки из металла, которые смачивает ртуть (например медь).
				Разумеется после использования такой "инструмент" окажется загрязнен ртутью и подлежит утилизации.

				Затем при помощи химических средств оставшаяся, не видимая глазу ртуть переводится в нелетучие {\bf но по прежнему ядовитые} 
				соли, которые спокойно можно удалить с поверхности моющими средствами. Для этого используются
				0,2\% водный раствор перманганата натрия ("марганцовка") подкисленный добавлением 0,5\% соляной кислоты или 20\% 
				раствор хлорного железа (того, которым платы травят). Вопреки указаниям в старых книгах, засыпание 
				места разлива порошком серы не эффективно.

				\item Тщательно промыть обработанные площади водой с моющим средством.

				\item Всю собранную ртуть и загрязненные предметы герметично упаковать и сдать в специализированную организацию.

			\stopitemize

			Что однозначно не стоит делать при разливе ртути:

			\startitemize [intro, n]

				\item Паниковать и спешить. Иногда, при небольших авариях больше вреда наносит паника и спешка, чем сама авария.
			Вспоминается байка, записанная Ю.А.Золотовым:

			\startnarrower[3*left,2*right]
				Однажды, когда профессор МГУ Алексей Николаевич Кост вел практикум по органической 
				химии, у одного из студентов разбилась колба с эфиром и его пары вспыхнули. 
				Началась паника, кто-то прибежал с углекислотным огнетушителем и с трудом погасил 
				пожар. Все это время Кост совершенно невозмутимо сидел за своим столом и с 
				кем-то разговаривал. Потом, когда все успокоились, подошел к месту происшествия и приказал:

				— Спички!

				Ему дали коробок, он чиркнул спичкой и бросил ее в еще не просохшую эфирную 
				лужу. Огонь вспыхнул вновь, все оторопели. А Кост, не суетясь, взял противопожарное 
				одеяло, ловко накрыл им пламя и изрек:

				— Гореть надо умеючи!
			\stopnarrower
	
				\item Пытаться собрать ртуть пылесосом, пылесос только в турборежиме раздробит 
				и испарит шарики ртути, в итоге все помещение и сам пылесос окажутся загрязнены ртутью.
				Аналогично не стоит использовать для сбора ртути веники, щетки — они только раскидывают и дробят шарики ртути.

				\item Сливать ртуть в раковину или унитаз. Ртуть значительно тяжелее воды, поэтому 
				навсегда осядет в первом попавшемся изгибе трубы — в гидрозатворе или колене.

			\stopitemize
			

			Некоторые в детстве играли шариками ртути, и "с ними ничего не было". 
			Действительно, вопреки распространенному мнению металлическая ртуть при 
			кратковременном контакте малоопасна. Причина малой токсичности металлической 
			ртути — в ее плохой {\it биодоступности}. Нерастворимая в воде и химически 
			инертная, почти как благородные металлы, она не может быстро попасть в организм.

			Опасно вдыхание паров ртути, и это практически единственный путь поступления ее в организм.
			Касание ртути пальцами никакой {\it дополнительной} опасности не добавляет. Более того, даже
			проглатывание ртути обычно проходит без последствий для здоровья. Ртуть химически достаточно
			инертна и выходит из организма естественным путем. Поэтому она является причиной не острых
			отравлений, а {\it вялотекущих хронических}, проявляющихся в медленном постепенном ухудшении
			здоровья и не всегда вовремя диагностируемых врачами. Именно этим ртуть и коварна: маленький
			шарик, закатившийся под плинтус, будет годами испаряться и отравлять воздух в квартире, а
			жильцы не будут понимать, чем и почему они болеют. Порча здоровья от контакта со ртутью в течение
			нескольких дней может быть необратима.

		 	Растворимые соединения ртути намного опаснее, и именно они образуются, когда ртуть
			так или иначе попадает в организм человека, животных или в растений. Рекорд по токсичности принадлежит 
			диметилртути — это ужасно токсичное из известных человечеству веществ, настолько токсичное, что 
			при первой возможности ищут менее опасную альтернативу если предстоит работа с ней.
			Капля диметилртути способна убить человека {\it сквозь резиновые перчатки}, причем первые симптомы
			отравления могут появиться только на следующий день.

			Если вы выкинув ртуть подальше от дома думаете, что проблема устранена — то 
			вы серьезно ошибаетесь. Ртуть — яд кумулятивный, способный к накоплению в живых организмах
			и передаче дальше по пищевой цепочке. Примером отравления человека ртутью является
			болезнь Минамата. Ртуть из выброшенной люминесцентной лампы отравит если не вас, то ваших потомков. 

		  
		\stopsubsubject

		\startsubsubject[title={Дополнительные сведения}]
			Если вы нашли где-нибудь ртуть, не пытайтесь ее продать. Ртуть и ее соли считаются сильнодействующими
			ядовитыми веществами (ст. 234 УК РФ). На содержащие ртуть приборы заводского производства,
			соответствующие официальным стандартам, запрет не распространяется. Найденную ртуть
			и неисправные ртутьсодержащие приборы, следует сдавать на переработку в
			специализированные службы в вашем городе. Единственный широко доступный источник ртути
			(если вдруг понадобится в научной работе) — медицинские термометры.
		\stopsubsubject

	\stopsubject

\stopchapter

%%%%%%%%%%%%%%%%%%%%%%%%%%%%%%%%%%%%%%%%%%%%%%%%%%%%%%%%%%%%%%%%%%%%%%%%%%%%%%%%%%%%%%%%%%%%%%%%%%%

\startchapter[title={Так себе проводники}]
	
	Иногда очень хорошая проводимость не требуется, и даже наоборот, нужны проводники с заметным сопротивлением, например для изготовления резисторов.
	Если взять для них медную проволоку, то ее понадобилось бы слишком много. А еще иногда требуются  материалы для скользящих контактов, сопротивлений 
	стабильных при изменении температуры и т.~п.  Выбор материала проводника осуществляется исходя из требований решаемой задачи, и в этом разделе проводники,
	которые обрадают не такой высокой проводимостью как расмотренные ранее, но обладают рядом свойств. Хоть логично было бы включить припои в этот раздел,
	но припои выделены в отдельную главу.


	\startsubject[title={Углерод}]

		\materialdesc{С — углерод.} Не совсем металл, но тоже проводник. Графит,
		угольная пыль — не такие хорошие проводники как металлы, но зато очень дешевые,
		не подвержены коррозии.

		\startsubsubject[title={Примеры применения}]

			\usagedesc{Компонент резисторов.} В виде пленок, в виде объемных брусков в диэлектрической
			оболочке.

			\usagedesc{Добавка в полимеры для придания электропроводности.} Для защиты от образования
			статического электричества достаточно ввести в состав полимера мелкодисперсный
			графит, и пластик из диэлектрика становится очень плохим проводником,
			достаточным, что бы статический заряд с него стекал. При работе с изделиями из
			такого пластика они не будут прилипать и искрить, что важно при пожароопасности
			или работе с электроникой.

			\placefigure[here]{
			Токопроводящий лак на базе мелкодисперсного графита. Покрыв пластиковую деталь
			таким лаком её электропроводность становится достаточной для выращивания слоя
			металла методом гальванопластики.
			}{\externalfigure[c-spray.jpg][height=.7\textheight]}

			\usagedesc{На базе полимеров, заполненных мелкодисперсным графитом, основаны различные
			нагреватели} — пленочные электронагреватели теплых полов, греющие кабели для
			систем водоснабжения, нагреватели для одежды и т~.д. Высокий коэффициент
			расширения правильно подобранного состава полимеров при нагреве приводит к отрицательной обратной связи, что
			делает такие нагреватели саморегулирующимися и потому безопасными.\footnote{Но это актуально далеко не для всех нагревателей!} При
			пропускании тока через такой полимер, он нагревается, от нагрева расширяется,
			контакт между частичками углерода в матрице из полимера ухудшается, от этого
			увеличивается сопротивление — уменьшается протекаемый ток, уменьшается нагрев.
			В итоге, устанавливается некоторая температура полимера, стабильно
			поддерживающаяся этим механизмом обратной связи без каких либо внешних
			устройств. 

			\placefigure[here]{
			Нагреватель от печки лазерного принтера. Основа — фарфор, проводники — серебро.
			Нагреватель — углеродная композиция, покрыта для защиты слоем глазури.
			}{\externalfigure[c-heater.jpg][width=\textwidth]}

			\usagedesc{Аналогично устроены полимерные самовосстанавливающиеся предохранители.} Если ток
			через такой предохранитель превысит номинальный, от нагрева полимер в составе
			расширяется, и резко увеличившееся сопротивление прерывает ток через
			предохранитель до некоторого небольшого значения. Такие предохранители
			обеспечивают медленную защиту, но не требуют замены предохранителя после каждой
			аварии.

			\usagedesc{Угольный сварочный электрод} — используется для сварки, когда от электрода
			требуется только поддерживать дугу не плавясь. Уголь значительно дешевле
			вольфрама, но менее прочен и постепенно сгорает на воздухе.

			\placefigure[here]{
			Электроды от дуговой лампы, использовавшейся для киносъемок. Марка электродов
			КСБ — Уголь КиноСьемочный Белопламенный неомеднённый.
			}{\externalfigure[c-electrode.jpg][width=\textwidth]}

			\usagedesc{Медно-графитовые материалы.}
			Получают спеканием порошка меди и графита в разных пропорциях. В зависимости от
			состава могут быть от чёрных как уголь до темно красных с медным блеском.
			Используется как материал скользящих контактов — щеток электрических приборов. 
			Такие щетки обеспечивают низкое сопротивление вращению — хорошо скользят по
			контактам коллектора. Кроме того их твёрдость заметно ниже твёрдости металла
			коллектора, так что в процессе работы истираются и подлежат замене дешевые
			щетки а не дорогой ротор.

			\placefigure[here]{
			Изношенные щетки от двигателя стиральной машины. Плохой контакт щеток с
			коллектором — причина повышенного искрения.
			}{\externalfigure[c-brush.jpg][width=\textwidth]}

			\usagedesc{Материал проводников на печатных палатах.} Иногда перемычки на односторонних печатных
			платах выполняются не проволокой, а нанесением графитовой пасты. Такие проводники не пригодны
			для работы с силовыми цепями но вполне сносно подходят для сигнальных цепей, удешевляя и упрощая производство.

			\placefigure[here]{
			Пульт дистанционного управления. Контактные площадки на печатной плате, проводники, проводящее покрытие на кнопке изготовлено из графитового покрытия.
			}{\externalfigure[c-remote.jpg][width=\textwidth]}

		\stopsubsubject

		\startsubsubject[title={Источники}]

			Если вдруг понадобился срочно угольный электрод, например сварить термопару,
			самый доступный способ — вытащить центральный электрод из солевой батарейки
			(маркировка которой начинается с R а не LR, щелочные («алкалиновые») не подойдут).
			Угольный стержень из батарейки содержит в себе следы электролита, поэтому перед 
			применением не лишнем будет промыть и прокипятить его в воде для удаления остатков электролита.

		\stopsubsubject
		\startsubsubject[title={Дополнительные сведения}]

			Графит — компонент смазок для механизмов работающих при высоких температурах или под большими нагрузками с малыми скоростями.
			Малая прочность кристаллической решетки графита обеспечивает скольжение слоев графита меж собой. Но при этом важно не забывать,
			графит в гальванической паре ведет себя близко к благородным металлам, поэтому не рекомендуется использовать графитные смазки в
			паре с алюминиевыми, магниевыми сплавами.
			% http://www.bmpcoe.org/library/books/navmat%20p-4855-2/63.html
			% http://www.npl.co.uk/upload/pdf/bimetallic_20071105114556.pdf
		\stopsubsubject

	\stopsubject


	\startsubject[title={Нихромы}]

		Для изготовления нагревателей, мощных сопротивлений требуются сплавы со
		следующими требованиями:

		\startitemize[intro]
			\item Относительно высокое удельное сопротивление — иначе
			нагреватель придется делать длинным и тонким, что отрицательно скажется на
			долговечности.

			\item Устойчивость к окислению на воздухе. Если в колбу лампы накаливания
			попадет воздух, то спираль очень быстро сгорит. При высоких температурах
			скорости химических реакций растут, и кислород воздуха начинает окислять даже
			стойкие при комнатной температуре металлы.

			\item Иметь приемлемые механические
			характеристики. Низкая пластичность и повышенная хрупкость негативно скажется
			на надежности изделия.
		\stopitemize

		Нагреватели обычно изготавливают из следующих сплавов:

		\materialdesc{Нихром} (55–78\% никеля, 15–23\% хрома) рабочая температура до \unit{1100 celsius} хотя нихромы — это целый класс
		сплавов с небольшой разницей в составе.

		\materialdesc{Фехраль,} название образовано от состава FeCrAl (12–27\% Cr, 3.5–5.5\% Al,
		1\% Si, 0.7\% Mn, остальное Fe) рабочая температура до \unit{1350 celsius} (Иногда называют канталом — kanthal, это не марка
		сплава, а торговая марка\footnote{Принадлежит компании Sandvik Materials Technology},
		которая стала нарицательной, как например «термос»).

		Добавка хрома обеспечивает образование защитной пленки на поверхности сплава,
		благодаря чему нагреватели из нихрома могут длительное время работать на
		воздухе с высокой температурой поверхности.

		Фехраль после нагрева становится ломким. Нихром после нагрева еще можно
		как-то гнуть. При этом фехраль дешевле нихрома, в рознице не так заметно, но
		ощутимо в оптовых партиях.

		Нихромовая спиралька с фитилем внутри — испаритель электронной сигареты.
		Нихромовой струной, подогреваемой электрическим током, режут пенополистирол.
		Также из нихрома изготавливают термосьемники изоляции — на сегодняшний день
		самый надежный способ снять изоляцию с провода и не повредить токопроводящую
		жилу.

		На удивление, достаточно трудно купить нихром в виде проволоки в небольших
		количествах, местные продавцы о количествах менее килограмма даже слышать не
		хотят. Так что, если понадобится изготовить нагревательный элемент — то проще
		перемотать нихром с какого-нибудь неисправного тепловентилятора. 

		Концы нагревательных элементов обычно приваривают к тоководам или зажимают
		механически — винтом или опрессовкой.

	\stopsubject


	\startsubject[title={Сплавы для изготовления термостабильных сопротивлений}]

		У всех материалов есть ТКС — температурный коэффициент сопротивления, мера
		того, насколько изменяется сопротивление с изменением температуры. Он может
		быть положительным — как у металлов, с  ростом температуры сопротивление
		растет, может быть отрицательным, как у полупроводников, с ростом температуры
		сопротивление падает. При изготовлении точных измерительных приборов необходимо
		иметь сопротивления с минимальным дрейфом номинала в зависимости от
		температуры. Для этого изобрели сплавы с минимальным ТКС:

		Константан (59\% Cu, 39–41\% Ni, 1–2\% Mn)

		Манганин (85\% Cu, 11.5–13.5\% Mn, 2.5–3.5\% Ni)

		Таблица, с указанием температурного коэффициента (обозначается как α) для различных металлов:

		\starttabulate[|l|l|]
			\HL
			\NC Материал      \NC Температурный коэффициент \alpha, \unit{ilinear celsius} \NC\NR
			\HL
			\NC Кремний       \NC -0,075      \NC\NR
			\NC Германий      \NC -0,048      \NC\NR
			\NC Манганин      \NC 0,00002     \NC\NR
			\NC Константан    \NC 0,00005     \NC\NR
			\NC Нихром        \NC 0,0004      \NC\NR
			\NC Ртуть         \NC 0,0009      \NC\NR
			\NC Сталь 0,5\% С \NC 0,003       \NC\NR
			\NC Цинк          \NC 0,0037      \NC\NR
			\NC Титан         \NC 0,0038      \NC\NR
			\NC Серебро       \NC 0,0038      \NC\NR
			\NC Медь          \NC 0,00386     \NC\NR
			\NC Свинец        \NC 0,0039      \NC\NR
			\NC Платина       \NC 0,003927    \NC\NR
			\NC Золото        \NC 0,004       \NC\NR
			\NC Алюминий      \NC 0,00429     \NC\NR
			\NC Олово         \NC 0,0045      \NC\NR
			\NC Вольфрам      \NC 0,0045      \NC\NR
			\NC Никель        \NC 0,006       \NC\NR
			\NC Железо        \NC 0,00651     \NC\NR
			\HL
		\stoptabulate

		Если упростить, то коэффициент \alpha{} говорит, во сколько раз изменится
		сопротивление проводника при изменении температуры на один градус Цельсия.

	\stopsubject
	\startsubject[title={Термопарные сплавы}]

		Для изготовления термопар используют сплавы стойкие к высоким температурам,
		но при этом обладающие высокой ТермоЭДС. Подробнее про
		термопары\footnote{https://ru.wikipedia.org/wiki/Термопара}
		можно прочитать в соответствующей литературе.

		Сплавы:
		\startitemize[intro]
			\item Хромель (90\% Ni, 10\% Cr)
			\item Копель (43\% Ni, 2–3\% Fe, 53\% Cu)
			\item Алюмель (93–96\% Ni, 1,8–2,5\% Al, 1,8–2,2\% Mn, 0,8–1,2\% Si)
			\item Платина (100\% Pt)
			\item Платина-родий (10–30\% Rh)
			\item Медь (100\% Cu)
			\item Константан (59\% Cu, 39–41\% Ni, 1–2\% Mn)
		\stopitemize

		Соединяя два проводника из двух разных металлов получают термопары, например
		термопара типа K (ТХА — Термопара Хромель-Алюмель). Самые распространенные пары:
		хромель-алюмель, хромель-копель, медь-константан (для низких температур),
		платина-платинородий (для точных измерений и для высоких температур).

	\stopsubject


	\startsubject[title={Оксид Индия-Олова}]

		\materialdesc{Оксид Индия - Олова} (Indium tin oxide или сокращённо ITO) —
		полупроводник, но обладает невысоким сопротивлением, а самое главное, пленка из
		оксида индия-олова прозрачна. Это свойство используется при производстве ЖК
		дисплеев, сетка электродов на поверхности стекла нанесена именно из оксида
		индия-олова. Также резистивные touch панели имеют прозрачное проводящее
		покрытие.

		Пленка ITO едва видна в отражении, чтобы хоть как то она была заметна пришлось
		разобрать ЖК дисплей:

		\placefigure[here]{
		  Стекла от ЖК индикатора электронных часов. Индикатор подключался к электронной
		  схеме через токопроводящую резинку, гребенка контактов видна в нижней части
		  стекла.
		}{\externalfigure[in-sn-displays.jpg][width=\textwidth]}


		\placefigure[here]{
		  На просвет проводящая пленка не видна
		}{\externalfigure[in-sn-transparency.jpg][width=\textwidth]}

		\placefigure[here]{
		  На удивление, сопротивление пленки довольно низкое.
		}{\externalfigure[in-sn-resistance.jpg][width=\textwidth]}

	\stopsubject


\stopchapter

%%%%%%%%%%%%%%%%%%%%%%%%%%%%%%%%%%%%%%%%%%%%%%%%%%%%%%%%%%%%%%%%%%%%%%%%%%%%%%%%%%%%%%%%%%%%%%%%%%%

\startchapter[title={Припои}]

	Пайка — это процесс соединения двух деталей при помощи припоя, материала с
	температурой плавления меньшей, чем у соединяемых деталей. Например, соединение
	двух медных проводников при помощи олова. Именно использование припоя —
	основное отличие от сварки, когда детали соединяются расплавом из самих себя,
	например стальной крюк к стальной двери приваривается при помощи стального
	плавящегося сварочного электрода.

	Припои чаще классифицируют на две группы — тугоплавкие (температура плавления
	\unit{400 celsius} и более) и легкоплавкие. Или, иногда, на твёрдые и мягкие. Учитывая, что
	мягкие припои обычно легкоплавкие, то часто твёрдые припои синоним тугоплавких,
	а мягкие припои — легкоплавких.

	В электронной технике припои используют для создания надежного электрического
	контакта. Основные припои в электронной технике — мягкие, на базе олова и
	оловянно-свинцовых сплавов. Все остальные экзотические припои рассматриваться
	не будут.


	\startsubject[title={Олово}]

		\materialdesc{Sn — Олово.} Основной компонент мягких припоев. Олово — относительно
		легкоплавкий металл, что позволяет использовать его для соединения проводников.
		В чистом виде не используется (см. факты). Из-за дороговизны олова (а также
		других причин, см. ниже), его в припоях разбавляют свинцом. Припой из 61\% олова
		и 39\% свинца образует эвтектику\footnote{https://ru.wikipedia.org/wiki/Эвтектика},
		такой смесью, ПОС-61 (Припой Оловянно-Свинцовый — 61\% олова) паяют радиодетали на
		платах, провода. В менее ответственных узлах (шасси, теплоотводы, экраны и т.п.)
		олово в припоях разбавляют сильнее, до 30\% олова, 70\% свинца.

		Электронные устройства долгое время паяли оловянно-свинцовыми припоями. Затем
		набежали экологи и заявили, что свинец — металл тяжелый, токсичный, и проблемы
		бы не было, если бы все эти ваши айфоны, компьютеры и прочие гаджеты не
		оказывались на свалке, откуда свинец попадает в окружающую среду. Поэтому
		придумали\footnote{https://ru.wikipedia.org/wiki/Restriction_of_Hazardous_Substances_Directive}
		серию бессвинцовых припоев, когда олово разбавлено висмутом, или
		вовсе используется в чистом виде, стабилизированное добавками, например, серебра.
		Но эти припои дороже, хуже по характеристикам, более тугоплавкие. Поэтому
		оловянно-свинцовые припои надолго останутся в ответственных изделиях военного,
		космического, медицинского применения.

		Кроме того, бессвинцовые припои склонны к образованию "усов".\footnote{Помимо олова, склонны создавать "усы" также покрытия из кадмия и цинка.} Оловянные усы —
		длинные тонкие кристаллы, вырастающие из оловянного припоя — причина отказов и
		сбоев аппаратуры. К сожалению, присадки в припои не позволяют на 100\% прекратить
		рост "усов", поэтому оловянно-свинцовые припои, как проверенные временем,
		используются в критичных системах — космос, медицина, военка, атомные
		применения. Подробнее\footnote{http://www.yaplakal.com/forum2/topic1490898.html} про усы.
		% Груев, Матвеев, Сергеева. Электрохимические покрытия РЭА. Стр 126
		% рост усов в оловянномп окрытии с содержанием менее 15% легирующего элемента неизбежно. Рост усов провоцирует механические напряжения.
		% усы также склонны образовывать кадмий, цинк, висмут

		\placefigure[here]{
		  Катушки и прутки оловянно-свинцовых припоев. Проволока из припоя содержит
		  центральный канал с флюсом, облегчающим процесс пайки.
		}{\externalfigure[sn-solder.jpg][width=\textwidth]}


		\startsubsubject[title={Факты об олове}]
			\startitemize[intro]

				\item Чистое олово подвержено "оловянной чуме", когда при температурах ниже
				\unit{13,2 celsius} олово меняет свою кристаллическую решетку, превращаясь из блестящего
				металла в серый порошок (как при нагревании алмаз превращается в графит).
				Согласно байкам, оловянная чума — одна из причин поражения Наполеоновской
				армии в условиях суровых российских городов (представьте, как на морозе ваши
				пуговицы, ложки, вилки, кружки превращаются в серый порошок). И вполне
				состоявшийся факт, что оловянная чума стала одной из причин которая погубила
				экспедицию Скотта\footnote{https://ru.wikipedia.org/wiki/Терра_Нова_(экспедиция)} —
				консервные банки, емкости с топливом были пропаяны оловом и на морозе просто
				развалились.  Небольшая добавка висмута практически устраняет оловянную
				чуму\footnote{https://www.youtube.com/watch?v=gu4xpw-EM88}.

				\item Олово проводит электрический ток в 7 раз хуже меди.

				\item Олово используется как защитное покрытие консервных банок — луженая жесть при
				контакте с пищей не делает её опасной. (но так как олово правее железа в ряду
				напряженности металлов, лужение не защищает железо от коррозии гальванически,
				как цинк, который левее железа в ряду напряженности. Как работает
				гальваническая защита можно прочитать по ссылке\footnote{http://lab115.com/?p=8}).

				\item До широкого распространения алюминия, фольгу делали из олова, её называли
				"станиоль" (от stannum — латинское название олова).

				\item Не пытайтесь отремонтировать ювелирные украшения при помощи мягких оловянных
				и оловянно-свинцовых припоев. Прочность соединения будет неприемлемой, а
				наличие легкоплавкого припоя на поверхности осложнит нормальную пайку твёрдыми
				припоями.

			\stopitemize
		\stopsubsubject

		\startsubject[title={Легкоплавкие припои}]

			На базе сплавов с содержанием олова были разработаны легкоплавкие припои. И
			даже очень легкоплавкие припои, которые плавятся в горячей воде. Хороший
			список\footnote{https://ru.wikipedia.org/wiki/Легкоплавкие_сплавы} сплавов есть в Википедии.


			Наиболее известные легкоплавкие припои:

			\startitemize[intro]
				\item Сплав Розе: 25\% Sn, 25\% Pb, 50\% Bi. Температура плавления \unit{+94 celsius}.

				\item Сплав Вуда: 12,5\% Sn, 25\% Pb, 50\% Bi, 12.5\% Cd Температура плавления \unit{+68,5 celsius}.
			\stopitemize

			Применяются для лужения печатных плат любителями, так как плавятся в горячей
			воде, и можно резиновым шпателем под слоем кипящей воды быстро покрыть припоем
			медную фольгу печатной платы.\footnote{Но такой способ лужения автор не рекомендует, при пайке
			печатных плат луженных таким способом возможно образование тонкого слоя легкоплавкого припоя под пайкой,
			что может провоцировать разрушение соединения при эксплуатации при повышенной температуре.}
			 В технике их используют для пайки деталей, не выдерживающих
			нагрева до обычной температуры припоев, или в тех случаях, когда зачем-то нужен очень
			легкоплавкий металл (например, для датчика температуры).

			Если спаять подпружиненные контакты легкоплавким припоем, то получится простой
			и надежный термопредохранитель, при превышении температуры припой плавится и
			контакты разрывают цепь. Правда, предохранитель получится одноразовым. Во многих
			советских телевизорах в блоке строчной развертки была защита из обычной стальной
			спиральной пружинки, припаянной на легкоплавкий припой. При перегреве, в том числе от
			большого тока через пружинку, она отпаивалась и отрывалась. Предохранители такого
			типа очень хороши как защита от пожара.

		\stopsubject


		\startsubject[title={Основные припои для радиоаппаратуры}]
			\startitemize[intro]

				\item ПОС-61 — 61\% олова, остальное — свинец. Температура плавления (ликвидус)
				\unit{183 celsius}. Есть множество сходных по составу и по свойствам импортных припоев, в
				которых пропорции компонентов отличаются на пару процентов, например Sn60Pb40
				или Sn63Pb37.

				\item ПОС-40 — 40\% олова. Остальное — свинец. Температура плавления (ликвидус)
				\unit{238 celsius} Менее прочный, более тугоплавкий, неэвтектический (плавится не сразу,
				есть диапазон температур при котором припой больше походит на кашу). Но
				благодаря тому, что чуть ли не в два раза дешевле (олово дорогое), применяется для
				неответственных соединений — пайка экранов, шин. Аналогичны припои ПОС-33
				(\unit{247 celsius}), ПОС-25 (\unit{260 celsuis}), ПОС-15 (\unit{280 celsius}).
				За счет постепенного затвердевания удобен для литья мелких вещей вроде оловянных солдатиков:
				в кашицеобразном состоянии его можно "подтолкнуть" в литник палочкой и создать дополнительное
				давление в форме. Получить такое же качество изделий из ПОС-61 заметно труднее.

				\item Бессвинцовые припои. Для пайки медных водопроводных труб горелкой чаще всего используют
				мягкий припой с 3\% меди (Sn97Cu3). Он не содержит свинца, потому пригоден для питьевой воды.
				По экологическим причинам современную электронику на заводах паяют в основном бессвинцовыми
				припоями. Хорошая статья.\footnote{http://go-radio.ru/lead-free-solder.html} В производстве
				электроники чаще всего используют бессвинцовые припои из композиции олова, серебра и меди:
				Sn 95,5 \% Ag 3,8 \% Cu 0,7\% (с небольшими вариациями в пропорциях, температура плавления \unit{217 celsius}),
				
				%https://www.sciencedirect.com/science/article/pii/S0927796X00000103?via%3Dihub
			\stopitemize

			
		\stopsubject

		
	\stopsubject

	
\stopchapter


%%%%%%%%%%%%%%%%%%%%%%%%%%%%%%%%%%%%%%%%%%%%%%%%%%%%%%%%%%%%%%%%%%%%%%%%%%%%%%%%%%%%%%%%%%%%%
\startchapter[title={Электрические соединения}]

	Популярная шутка говорит о том, что электротехника — это наука о контактах. И две основные неисправности —
	нет контакта там где он должен быть, и есть контакт там где его быть не должно. 

	На обложке этого руководства изображена скрутка двух проводов — медного и алюминиевого. Некоторых читателей
	такое зрелище возмутило, и не без оснований — так делать нельзя. Если попытаться разобраться в причинах этого "нельзя",
	то можно найти множество дискуссий на эту тему, практически в каждой из которых можно найти довод "всегда так делал,
	на даче такая скрутка работает уже 100500 лет". К сожалению, понимания причин запрета такой подход не привносит.

	В чем же проблема соединить в контакт два произвольных металла? Дело в том, что в силу некоторых причин (о которых ниже)
	некоторые металлы образуют надежный контакт и работают практически безотказно, а некоторые образуют контакт, который тоже работает, но
	менее надежен и чаще приносит проблемы. Нужно понимать, что "чаще" не означает, что если вы сделали такое соединение, 
	то оно откажет завтра с вероятностью 100\%. Нет, вероятность отказа станет не 0,0001\%, а к примеру 0,01\%. Все такая же малая,
	но вас бы не устроила в 100 раз большая вероятность пожара?

	Опыт эксплуатации различной техники привел инженеров к выводу, что определенные комбинации металлов обеспечивают приемлемую надежность контакта,
	а некоторые слишком низкую. Еще раз стоит отметить, что на надежность контакта сильно влияют условия эксплуатации, если соединение
	находится при постоянной температуре в сухом месте, то оно может быть вполне надежным, даже если пара металлов нежелательная.



	\startsubject[title={Ряд электрохимической активности металлов}]

		Первая причина нарушения контакта которую мы рассмотрим — электрохимическая коррозия.

		Некоторые из вас помнят со школы ряд активности металлов (неполный):

		Li K Ba Sr Ca Na Mg Al Mn Cr Zn Fe Cd Co Ni Sn Pb H Sb Bi Cu Hg Ag Pd Pt Au

		Значения электрохимического потенциала этих металлов приведены в таблице \in[tab:electrochem].\footnote{https://ru.wikipedia.org/wiki/Электрохимический_ряд_активности_металлов}


		\placetable[here][tab:electrochem]{Таблица электрохимических потенциалов распространенных металлов}
		\starttabulate[|l|l|]
			\HL
			\NC Металл \NC Электрохимический потенциал, \unit{volt} \NR
			\HL
			\NC Литий (Li)         \NC -3,0401  \NC\NR
			\NC Калий (K)          \NC -2,931   \NC\NR
			\NC Барий (Ba)         \NC -2,905   \NC\NR
			\NC Стронций (Sr)      \NC -2,899   \NC\NR
			\NC Кальций (Ca)       \NC -2,868   \NC\NR
			\NC Натрий (Na)        \NC -2,71    \NC\NR
			\NC Магний (Mg)        \NC -2,372   \NC\NR
			\NC Алюминий (Al)      \NC -1,700   \NC\NR
			\NC Марганец (Mn)      \NC -1,185   \NC\NR
			\NC Хром (Cr)          \NC -0,852   \NC\NR
			\NC Цинк (Zn)          \NC -0,763   \NC\NR
			\NC Железо (Fe)        \NC -0,441   \NC\NR
			\NC Кадмий (Cd)        \NC -0,404   \NC\NR
			\NC Кобальт (Co)       \NC -0,28    \NC\NR
			\NC Никель (Ni)        \NC -0,234   \NC\NR
			\NC Олово (Sn)         \NC -0,141   \NC\NR
			\NC Свинец (Pb)        \NC -0,126   \NC\NR
			\NC Водород (H)        \NC 0        \NC\NR
			\NC Сурьма (Sb)        \NC +0,240   \NC\NR
			\NC Висмут (Bi)        \NC +0,317   \NC\NR
			\NC Медь (Cu)          \NC +0,338   \NC\NR
			\NC Ртуть (Hg)         \NC +0,7973  \NC\NR
			\NC Серебро (Ag)       \NC +0,799   \NC\NR
			\NC Палладий (Pd)      \NC +0,98    \NC\NR
			\NC Платина (Pt)       \NC +0,963   \NC\NR
			\NC Золото (Au)        \NC +1,691   \NC\NR
			\HL
		\stoptabulate

		Для инженера этот ряд говорит следующее: В присутствии электролита (вода, влажность воздуха) 
		в паре металлов будет разрушаться тот металл, что в ряду напряженности {\it левее}.
		Чем дальше друг от друга металлы в ряду, тем интенсивнее будет протекать коррозия. На базе 
		этого явления построена электрохимическая защита металлов, например оцинковка стали.
		При наличии воды, первым делом разрушается цинковое покрытие, и только после того 
		как оно разрушилось начинается коррозия стали.

		В случае электрических контактов, нам важнее не то, какой металл разрушится в паре, 
		они нужны оба, а то, насколько интенсивно будет протекать процесс коррозии. И в этом плане
		потенциал создаваемый парой алюминий-медь 2,038 \unit{volt} очень большой, его достаточно 
		чтобы разорвать\footnote{Минимально необходимое напряжение для этого 1,23 \unit{volt}} 
		молекулу воды в процессе электролиза! Но если разделить эти два металла стальной 
		оцинкованной \footnote{Считаем, что слой цинка сплошной, поэтому не принимаем во внимание 
		пару железо-цинк} пластинкой, то образуется две пары: цинк-алюминий с потенциалом 0,937 \unit{volt},
		и цинк-медь, с потенциалом 1,101 \unit{volt}. Это уже не такие большие потенциалы,
		поэтому процесс коррозии будет протекать медленнее.

		Принимая во внимание, что основными металлами для изготовления проводников являются медь 
		и алюминий, то заучивать таблицу и считать потенциалы не требуется, важно только помнить, 
		что непосредственно соединять медь и алюминий в электрический контакт работающий на воздухе нельзя.

		Наиболее изобретательные читатели зададут вопрос: "А если соединение меди и алюминия покрыть
		непроницаемым для воды слоем (лак, герметик, краска, густая смазка и т.д.), то тогда
		в месте контакта не будет воды, а значит не будет электролиза и коррозии?". Это верное замечание,
		и позволяет решить проблему электрохимической коррозии, но есть еще одна проблема, которая 
		не позволяет соединять любые два металла просто так.

		% MIL-STD-1250A Table III. Galvanic couples. хорошая таблица


	\stopsubject

	\startsubject[title={Тепловое расширение}]

		Все тела при нагревании расширяются, и металлы не исключение. Для любого материала есть характеристика,
		такая как "коэффициент теплового расширения тел", который показывает, во сколько раз
		увеличится размер тела, при нагреве на \unit{1 celsius}. \footnote{В различных диапазонах температур 
		значение теплового коэффициента расширения может различаться, кроме того для некоторых анизотропных
		материалов коэффициент может различаться в разных плоскостях. Для упрощения не будем учитывать эту разницу,
		воспользовавшись усредненными значениями} Вот небольшая табличка:\footnote{https://en.wikipedia.org/wiki/Thermal_expansion}

		\starttabulate[|lp|rp|]
		\HL
		\NC Материал \NC Тепловой коэффициент расширения $\alpha$  при \unit{20 celsius}, \unit{K^-1}\NR
		\HL
		\NC Алюминий                              \NC $23,1  \cdot 10^{-6}$  \NC\AR
		\NC Медь                                  \NC $17    \cdot 10^{-6}$  \NC\AR
		\NC Сталь                                 \NC $10,8  \cdot 10^{-6}$  \NC\AR
		\NC Стекло                                \NC $8,5   \cdot 10^{-6}$  \NC\AR
		\NC Стекло термостойкое (боросиликатное)  \NC $3,3   \cdot 10^{-6}$  \NC\AR
		\NC Стекло кварцевое                      \NC $0,59  \cdot 10^{-6}$  \NC\AR
		\NC Инвар (сплав)                         \NC $1,2   \cdot 10^{-6}$  \NC\AR
		\NC Платина                               \NC $9     \cdot 10^{-6}$  \NC\AR
		\HL
		\stoptabulate

		Из этой таблички видно, что соединение из двух материалов при нагревании будет расширяться по разному, провоцируя
		внутренние напряжения и деформации. Иногда это полезное свойство — оно используется в биметаллических пластинках в терморегуляторах, 
		такие пластинки при нагреве изгибаются и разрывают контакт. Но в деле создания надежного электрического соединения
		такая разница в величине теплового расширения может ослабить контакт. Если соединение не обладает упругими свойствами,
		то спустя нескольких циклов нагрева и охлаждения, можно обнаружить что вместо плотного тугого контакта проводник болтается.

		Если соединения разных материалов не избежать, то нужно помнить, что такое соединение потенциально может ослабнуть
		при изменениях температуры, и должно быть обслуживаемым и контролируемым. Замуровать соединение медного и алюминиевого
		проводника в стенке под слоем штукатурки — плохая идея.

	\stopsubject

	\startsubject[title={Ползучесть}]

		Некоторые материалы склонны проявлять явление "ползучести", когда к примеру проводник под небольшой механической нагрузкой, не достаточной
		для пластической деформации, тем не менее деформируется со временем. Величина этого явления зависит от нагрузки и от температуры,
		характеризуясь очень малой величиной. Пройдут тысячи часов, прежде чем размер тела изменится на доли процента. Тем не менее
		это явление достаточно важно в обеспечении надежного контакта. Ползучесть, наряду с тепловым расширением вносит вклад в то, 
		что затянутая клемма спустя годы ослабевает и провод в ней болтается.

		% https://w3.siemens.com/powerdistribution/global/EN/tip/focus-markets/Documents/Siemens-Data-Center-Whitepaper-Aluminium-versus-Copper.pdf
		% https://help.leonardo-energy.org/hc/en-us/articles/207900985-Copper-wires-resist-mechanical-creeping

		К сожалению алюминий (чистый) обладает значительно более интенсивной ползучестью, чем медь, что делает электрические контакты с его участием менее надежными
		и требующими регулярного обслуживания. Это стоит помнить при ремонте и обслуживании проводки из алюминиевого кабеля времен СССР.
		Производители современных алюминиевых кабелей легируют алюминий в токопроводящей жиле, добиваясь уменьшения ползучести до значений, сопоставимых с медью,
		пускай и ценой небольшого снижения электропроводности.


	\stopsubject

	\startsubject[title={Так как же все-таки соединять провода?}]

		Вопрос сложный тем, что ответ зависит от условий работы соединения и однозначно универсального способа нет.

		Но про пару алюминий-медь было сказано столько плохого, что я просто обязан дать ответ на вопрос "как их соединять?".

		Первый вариант — классический, при помощи стальной пластинки исключая непосредственный контакт меди и алюминия. Стальная пластинка
		предотвратит интенсивную электрохимическую коррозию (но не избавит от нее совсем), обеспечит приемлемое усилие на площади контакта проводников.
		Но такое соединение требует регламентных работ по обслуживанию: 1–2 раза в год необходимо проверять усилие затяжки проводников.

		Второй вариант. Специализированные пружинные клеммы для алюминиевого проводника. (например клеммники WAGO серии 2273 с пастой).
		В такой клемме зачищенный проводник всё время прижимается пружинным контактом, предотвращая его ослабление вследствие ползучести.
		Паста внутри клеммника предотвращает доступ влаги и воздуха к поверхности алюминия, препятствуя окислению проводника. \footnote{Важно отметить,
		клеммы должны быть качественные, а сечение проводника номинальным. Самолично наблюдал сгоревшие соединения выполненные клеммами, купленными в 
		ближайшем киоске (вероятно поддельными).}

		Третий вариант — Медно-алюминиевые гильзы. Этот вид соединения актуален для силовых линий на большие токи с сечением от 10 кв. мм.
		Медно-алюминиевые гильзы предназначены под опрессовку специальным инструментом. Соединенные в толще металлы обеспечивают надежный контакт большой площади,
		влага и электрохимическая коррозия могут лишь повредить нежную поверхность гильзы, не нарушив контакт в толще.

		И помните, любое силовое электрическое соединение (тем более из разных металлов) должно быть доступно для обслуживания! Замурованная в стену скрутка - залог
		того, что вас будет вспоминать ремонтная бригада в различных матерных выражениях.

	\stopsubject

\stopchapter

%%%%%%%%%%%%%%%%%%%%%%%%%%%%%%%%%%%%%%%%%%%%%%%%%%%%%%%%%%%%%%%%%%%%%%%%%%%%%%%%%%%%%%%%%%%%%%%%%%%


\startchapter[title={Неорганические диэлектрики}]

	Помимо проводников для производства электронной техники нужны диэлектрики. В
	зависимости от условий и задач, могут быть важны разные свойства диэлектрика:
	теплостойкость, тангенс угла потерь, гигроскопичность, механическая прочность и
	т.~д.

	Диэлектрики в этом руководстве разбиты на группы по происхождению - Неорганические,
	Органические полусинтетические (тоесть основанные на природных материалах но улучшенные и обработанные каким-либо образом)
	и органические синтетические - полученные полностью искуственно. В последнем разделе в основном перечисленны полимеры (пластики).
	Надеюсь после прочтения читатель будет различать пластики по виду, и они перестанут быть абстрактной категорией где всё свалено подряд.
	
	К сожалению свойства полимеров очень сильно различаются, в зависимости от условий синтеза, введеных добавок, последующей обработки.
	Производители пластиков идут на различные ухищрения и манипуляции с составом, внося важные и не очень
	изменения. Это как с книгами, разные издания одного и того же произведения, где
	то на газетной бумаге с плохой версткой, а где то на качественной бумаге с
	цветными иллюстрациями от модного художника. И та и другая книга — "Властелин
	колец", но впечатления от использования могут отличаться. Поэтому приведены
	некоторые общие свойства разных видов полимеров, за более точными
	характеристиками нужно обращаться к справочнику.


	Материалы, которые применяются в электронной технике меняются по мере
	прогресса. Так, ранее широко использовалось, к примеру, дерево, шелк, эбонит.
	Сегодня же многие материалы вытеснены более дешевыми, технологичными
	заменителями. В пособии есть описание в том числе исторических материалов,
	данных для общего развития. Также добавлена информация, необходимая для полноты
	раскрытия темы.


	\startsubject[title={Фарфор}]

		\materialdesc{Фарфор} — плотная прочная керамика, получаемая обжигом смеси 
		каолина, кварца, полевого шпата и глины. Аналогичен фарфоровой чашке у вас на 
		кухне, только при техническом применении реже покрывается глазурью.

		\startsubsubject[title={Примеры применения}]

			\usagedesc{Высокотемпературные изоляторы.} В виде фарфоровых бус для изоляции 
			концов нагревательных спиралей. Чешуеподобная конструкция бусин позволяет изгибаться 
			не обнажая проводник. Иногда нагревательную спираль прячут защитные фарфоровые бусины.

			\placefigure[here]{
			Корпус ртутной дуговой лампы от светолучевого осциллографа. Рама из алюминиевого 
			сплава, чёрный корпус — карболит, фарфоровые бусы изолируют проводники, которыми 
			подключается лампа. Лампа очень сильно нагревается во время работы. Рядом кучка цилиндрических
			фарфоровых бус от различных нагревателей.
			}{\externalfigure[Ceramic-beads.jpg][width=\textwidth]}

			\placefigure[here]{Проводники в изоляции из фарфоровых бус для работы рядом с мощной 
			дуговой ксеноновой лампой кинопроектора
			}{\externalfigure[ceramic_beards_wires.jpg][width=\textwidth]}


			\usagedesc{Детали электроизделий.} Если заглянуть внутрь патрона для лампы, то 
			часть, которая содержит ламели подключения скорее всего сделана из фарфора, он 
			может длительное время работать при повышенной температуре лампы накаливания без 
			потери свойств. Корпуса предохранителей, розеток, держатели контактов ламп — 
			везде, где есть опасность нагрева, фарфор вне конкуренции.

			\placefigure[here]{
			Держатели ламелей розетки, патрона изготовлены из фарфора. Чёрный корпус 
			патронов — карболит. 
			}{\externalfigure[Ceramic.jpg][width=\textwidth]}

			\usagedesc{Изоляторы на столбах.} На фото изолятор со столба, ликвидированного в 
			ходе реконструкции линии. Тридцать лет солнца, ветра, птичьего помета, дождей, морозов 
			нисколько не повлияли на фарфор, он по прежнему выглядит как новенький, 
			достаточно было помыть изолятор с мылом.\footnote{Срок службы фарфоровых изделий ограничен из-за появления микротрещин в процессе эксплуатации.}

			\placefigure[here]{Фарфоровые изоляторы линий электропередач. Между фарфоровым 
			изолятором и стальным крюком втулка из полиэтилена, для защиты фарфора от трещин. 
			Дисковая форма изоляторов позволяет воде стекать не образуя сплошного слоя, 
			замыкающего проводник на опору. Фарфоровые изоляторы, в отличии от стеклянных,
			непрозрачны, что затрудняет визуальную проверку изолятора на наличие трещин.
			}{\externalfigure[Ceramic-powerlines.jpg][width=\textwidth]}

			\placefigure[here]{Мощные резисторы имеют основу из фарфоровой трубки. У зеленого 
			резистора обмотка скрыта под эмалью.
			}{\externalfigure[Ceramic-resistor.jpg][width=\textwidth]}

			\placefigure[here]{Свечи зажигания от двигателя внутреннего сгорания. Центральный 
			электрод изолирован фарфором. Ни один другой диэлектрик не способен выдержать 
			длительное воздействие температуры, давления, горючего внутри камеры сгорания.
			}{\externalfigure[Ceramic-sparkplug.jpg][width=\textwidth]}

		\stopsubsubject


		\startsubsubject[title={Недостатки}]

			\problemdesc{Хрупкий,} как и все керамики. Перетянутый винт, удар — и фарфор 
			осыпается.

		\stopsubsubject

	\stopsubject


	%%%%%%%%%%%%%%%%%%%%%%%%%%%%%%%%%%%%%%%%%%%%%%%%%%%%%%%%%%%%%%%%%%%%%%%%%%%%%%%
	\startsubject[title={Стекло}]

		В зависимости от требований могут использоваться разные сорта стекол, от 
		легкоплавких натриевых до тугоплавких кварцевых. Основной плюс стекла, помимо его 
		термостойкости — прозрачность для видимого света (а кварцевое прозрачно еще и 
		для ультрафиолета). Также немаловажный плюс — возможность визуально оценить 
		целостность, трещины в стекле обычно видны.

		\startsubsubject[title={Примеры применения}]

			Корпуса радиоламп, осветительных ламп, предохранителей.

			\placefigure[here]{Стеклянный и фарфоровый изолятор линий электропередач 
			проработавший на улице более 30 лет.
			}{\externalfigure[glass_and_ceramic_insulator.jpg][width=\textwidth]}


			Кварцевые трубки — корпуса нагревателей, электрогрилей
			\placefigure[here]{Кусочек технического кварцевого стекла. Видно большое количество пузырьков в стекле.
			}{\externalfigure[quartz_defect.jpg][width=\textwidth]}

			Типичный признак (но не обязательный!) технического кварцевого стекла — большое количество 
			пузырьков в направлении экструзии стекла. Более дорогое оптическое кварцевое стекло
			абсолютно прозрачно. Торец такого стекла белый, без зеленого оттенка.


			Корпуса маломощных полупроводниковых диодов, изоляторы выводов радиоэлементов.
			\placefigure[here]{Корпуса этих полупроводниковых диодов изготовлены из стекла.
			}{\externalfigure[glass_diode.jpg][width=\textwidth]}

			Недостатки: Хрупкое, не выносит ударов. Некоторые сорта стекла растрескиваются 
			при резком неравномерном нагреве.

	
	
		\stopsubsubject

		\startsubsubject[title={Интересные факты о стекле}]

			Здесь стоит дополнительно сказать про сапфировое стекло, закаленное стекло и 
			химически закаленное стекло. В рекламных описаниях множества электронных устройств 
			для массового потребления можно встретить упоминания этих видов стекол.

			\bold{Сапфировое стекло} формально стеклом не является (оно не аморфное, как стекла, а 
			кристаллическое), но, в силу внешнего сходства, так именуется. Сапфировое стекло — 
			это тонкие пластинки лейкосапфира (чистый \chemical{Al_2O_3} — оксид алюминия). 
			Лейкосапфир тверже обычных стекол, поэтому используется для защиты оптики от
			абразивного истирания песчинками пыли в военной технике,  в дорогих устройствах 
			бытового назначения. Стекло наручных часов из сапфира дольше останется 
			нецарапанным. При этом, получение сапфировых стекол большого размера по вменяемой 
			цене затруднительно, поэтому планшеты с сапфировым стеклом мы увидим нескоро.

			\bold{Закаленное стекло}. Стекло хорошо сопротивляется сжатию и плохо — растяжению. 
			Повысить механическую прочность стекла можно его закалкой — стекло разогревают 
			до высоких температур и резко и равномерно охлаждают. В результате в стекле 
			образуются механические напряжения, которые увеличивают механическую прочность. 
			Чаще всего закалку стекла делают для безопасности. Обычное стекло, если в него 
			кинуть камнем, разбивается на несколько довольно крупных осколков, которые могут 
			нанести серьезную травму. Закаленное стекло при разрушении дает много мелких 
			осколков, которые значительно безопаснее. Поэтому все\footnote{Кроме лобового,
			иначе оно разрушалось от первого прилетевшего из под колес камушка. Лобовое стекло для
			безопасности трехслойное — средний слой из полимерной пленки с клеем. При ударе все
			осколки оказываются приклеенными к пленке.} стекла в автомобиле, в 
			торговых центрах, стеклянные полки мебели — закалены. Изделие из закаленного 
			стекла обработке не подлежит, если попытаетесь стеклянную полочку для ванной 
			подрезать, она с хлопком рассыпется в крошку, поэтому закалка производится после 
			обработки. Классической демонстрацией свойств закаленного стекла являются батавские слёзки.

			\bold{Химически закаленное стекло}. Например, часто упоминаемое Gorilla glass. Для 
			тонких пластинок стекла термический способ закалки не подходит, поэтому пластинки 
			стекла обрабатывают в растворе, который, к примеру, замещает ион натрия на ион 
			калия. Так как ион калия крупнее, то поверхностные слои стекла как бы "распирает" 
			более крупными атомами в решетке, создавая как раз требуемые механические 
			напряжения. Как итог — такое стекло прочнее, лучше сопротивляется царапинам. 

			\bold{Термостойкое стекло}. Обычное оконное стекло при нагревании сильно расширяется. Если нагрев
			неравномерный, то части стекла из-за разного расширения создадут механические напряжения,
			что может привести к растрескиванию. Введением добавок коэффициент теплового расширения
			стекла уменьшают, получая термостойкие сорта. Такие стекла при {\bf неравномерном}
			нагреве не образуют трещин. Наиболее крутое в этом отношении кварцевое стекло,
			поэтому из него делают корпуса нагревателей в электрогрилях.

		\stopsubsubject

	\stopsubject

	%%%%%%%%%%%%%%%%%%%%%%%%%%%%%%%%%%%%%%%%%%%%%%%%%%%%%%%%%%%%%%%%%%%%%%%%

	\startsubject[title={Слюда}]

		\materialdesc{Слюда} Природный слоистый материал, обладает термостойкостью, 
		прочностью, прекрасный диэлектрик. Слюды — большой класс слоистых минералов, из них
		в технике используется в основном мусковит и иногда биотит и флогопит. 

		По английски слюда — Mica, отсюда производные 
		названия материалов на базе слюд — миканиты, микалента, микафолий, микалекс и т.д.

		Слюда, добытая в руднике, разбирается, сортируется. Крупные куски вручную 
		расщепляются на пластинки — так получается {\bf щипаная слюда} — прозрачные 
		однородные пластинки. Такая слюда обладает самым высоким качеством и идет на 
		ответственные применения — в вакуумной технике, окна ввода/вывода излучения и т.д. 
		К сожалению, крупные однородные куски слюды без дефектов — редкость, поэтому 
		пластинки из слюды разной формы склеивают воедино, так получается {\bf миканит}. 
		Если в качестве подложки для наклеивания пластинок слюды использовать ткань 
		(стеклоткань, бумагу) получается {\bf микалента, микафолий, стекломиканит}. 
		Совсем мелкие отходы слюды размалываются, и в виде водной пульпы отливаются на 
		сетку, также как бумага.  После удаления воды частички слюды слипаются в единое 
		полотно — получается {\bf слюдяная бумага (слюдинит, слюдопласт)}. Получившееся 
		полотно для прочности может пропитываться органическим связующим. Гибкость 
		слюдяной бумаги позволяет наматывать её в качестве изоляции. Также намоткой 
		можно получить стержни, трубки. Если пропитать слюду расплавленным стеклом, то 
		получившийся прочный материал называется {\bf микалекс}.

		Перемолотая в пыль слюда — компонент пигментов, благодаря своей "чешуйчастости" 
		дает перламутровый эффект. В пигментах используется в основном биотит.

		Синтетический материал — фторфлогопит (synthetic mica) — это слюда (флогопит) 
		где -OH группы заменены фтором. Фторфлогопит более прочен и термически стоек, 
		выглядит также как слюда, тоже слоистый но абсолютно прозрачный/белый, а не 
		желтоватого оттенка, как природная слюда. Увы, пока с этим материалом живьем не 
		сталкивался.

		\startsubsubject[title={Примеры применения}]

			Конструктивные элементы для удержания нагревательных элементов в фенах, 
			калориферах, тепловентиляторах, паяльниках и т.~д.

			\placefigure[here]{Нагреватели бытовых тепловентиляторов. Конструкция слева 
			менее материалоемкая, но значительно менее надежная, особенно в условиях 
			механических нагрузок.
			}{\externalfigure[Mica_heaters.jpg][width=\textwidth]}

			Как защитное окошко выхода микроволнового излучения от магнетрона в микроволновках. 
			(обычно попадая на слюду еда обугливается, и становясь проводником, начинает бурно 
			искрить, от чего владельцы микроволновки со страху микроволновку выбрасывают, 
			хотя достаточно вырезать пластинку из листа слюды и заменить окошко.) 

			\placefigure[here]{Слюдяное окошко в микроволновке. Иногда встречаются 
			пластиковые, но только у моделей без гриля.
			}{\externalfigure[Mica_microwave.jpg][width=\textwidth]}

			Благодаря тому, что тонкие пластинки слюды не пропускают газы, но пропускают 
			энергичные заряженные частицы — слюдяные окошки используются в конструкциях 
			счетчиков альфа и бета частиц.

			Используется в конструкциях радиоламп — удерживает электроды на своих местах.

			\placefigure[here]{Восьмигранная пластинка изготовлена из слюды.
			}{\externalfigure[Mica_tube.jpg][width=\textwidth]}

			Используется как материал слюдяных конденсаторов. Слюда выступает диэлектриком, 
			а электродами — проводящее напыление металла на пластинках слюды. Данный вид 
			конденсаторов встречается всё реже и реже, вытесненный конденсаторами на базе 
			полимерных пленок. Слюдяные конденсаторы могут работать при высокой температуре.

			\placefigure[here]{Слюдяные конденсаторы производства СССР полувековой давности.
			}{\externalfigure[Mica_cap.jpg][width=\textwidth]}

			\placefigure[here]{Пластинки слюды в конденсаторе. Металлизация на пластинках 
			формирует обкладки.
			}{\externalfigure[Mica_cap_open.jpg][width=\textwidth]}


			До появления и широкого распространения теплопроводящих изолирующих прокладок 
			из полимерных материалов, вроде Номакон, слюдяные пластинки использовались для 
			электрической изоляции компонентов при сохранении теплового контакта, например, 
			когда необходимо на один радиатор закрепить несколько транзисторов, корпуса 
			которых под разными напряжениями.


			\placefigure[here]{Пластинки природной щипаной слюды.
			}{\externalfigure[Mica_sheets.jpg][width=\textwidth]}

			\placefigure[here]{Природная слюда прозрачна. Слюдоматериалы полученные 
			переработкой природной слюды как правило непрозрачны.
			}{\externalfigure[Mica_transparent.jpg][width=\textwidth]}

			\stopsubsubject

			\startsubsubject[title={Интересные факты о слюде}]

			Раньше, несколько веков назад, когда не умели делать  тонкие оконные стекла, 
			светопрозрачные конструкции делали расщепляя природную слюду. Так как большие 
			куски слюды без дефектов были редкостью, то и окна принимали причудливую форму.

			\placefigure[here]{Слюда вместо стекла в оконной раме. Из экспозиции 
			красноярского краеведческого музея.
			}{\externalfigure[Mica_window.jpg][height=.7\textheight]}

		\stopsubsubject

		Слюда — достаточно мягкий материал, слюдяная пластинка (как и большинство 
		материалов на её базе) легко режется ножницами. В силу своей слоистой природы, 
		склеивание слюды — занятие малонадежное, сила сцепления меж слоев невысокая, 
		поэтому при производстве детали из слюды скрепляют механически — заклепки, 
		люверсы, винты и т.~д.

		\placefigure[here]{Электрические соединения с нагревательным элементом выполнены 
		полыми заклепками.
		}{\externalfigure[Mica_electrical_connections.jpg][width=\textwidth]}

	\stopsubject

	%%%%%%%%%%%%%%%%%%%%%%%%%%%%%%%%%%%%%%%%%%%%%%%%%%%%%%%%%%%%%%%%%%%%%%%%%%%%%%%%%%%%%%

	\startsubject[title={Алюмооксидные керамики}]

		Очень похожи по внешнему виду на фарфор, только лучше. Содержат практически 
		чистый \chemical{Al_2O_3}. Более подробно неплохо описано в этой\footnote{http://www.test-expert.ru/news/detail.php?ID=436} статье.

		Твёрдая, прочная керамика, из которой изготавливают:

		\startsubsubject[title={Примеры применения}]

			\usagedesc{Корпуса микросхем,} обычно ответственного применения.

			\placefigure[here]{Корпуса процессоров раньше делали керамическими, но рост 
			тепловыделения и конкуренция по цене вынудили отказаться от этого материала. 
			Именно с керамическим корпусом процессоров был связан анекдот про нового русского 
			и плитку в ванной от Intel.
			}{\externalfigure[ceramic_CPU.jpg][width=\textwidth]}

			\usagedesc{Корпуса электровакуумных приборов.} 

			\placefigure[here]{Корпус вакуумной колбы магнетрона изготовлен из меди и алюмооксидной керамики. Керамика видна на фото, фиолетовый поясок между колпачком и корпусом.
			}{\externalfigure[magnetron.jpg][width=\textwidth]}


		\stopsubsubject

		Алюмооксидная керамика очень твёрдая, обрабатывается как и многие керамики 
		алмазным инструментом. Обломок керамического корпуса микросхемы — отличное 
		орудие для написания посланий на лобовом стекле автомобиля, оставляет четкие 
		ровные царапины не хуже стеклореза.

		Данный вид керамики плотный, не впитывает влагу, удерживает вакуум, не трескается 
		при резком перепаде температур и тепловом ударе. При этом сцепление металлических пленок с поверхностью высокое, 
		позволяет делать на керамике дорожки, герметично приваривать металлические детали.

		Внешне очень похожа бериллиевая керамика — она превосходит алюмооксидную керамику по предельной рабочей температуре, 
		по теплопроводности (сопоставимую с металлами!), но в силу дороговизны и токсичности пыли из нее применяется редко.

	\stopsubject

	%%%%%%%%%%%%%%%%%%%%%%%%%%%%%%%%%%%%%%%%%%%%%%%%%%%%%%%%%%%%%%%%%%%%%%%%%%%%%%%%%%%%%%


	\startsubject[title={Асбест}]

		Уникальный, непревзойденный класс материалов. Природное волокно, "горный лен". 
		Является огнестойким диэлектриком. Использовалось во множестве применений, начиная 
		от армирующей добавки в полимеры, заканчивая изоляцией нагревательных приборов.
		Выпускается в виде листов (асбестокартон), нити, пряжи. Чаще всего используется именно как теплоизолятор, 
		как диэлектрик только в установках невысокого (до \unit{1 kilovolt}) напряжения.

		Широко применялся в строительстве. Шифер — это цемент, упрочненный волокнами асбеста, 
		практически вечный материал. Высоко ценилась его дешевизна и огнестойкость. Но 
		есть одно но:

		{\bf Асбест — канцероген.} Причем канцероген 1-го класса (от МАИР), наравне с 
		мышьяком, формальдегидом. \footnote{Степень опасности различных видов асбеста — 
		вопрос дискусионный, и нет единодушного мнения на этот счет.} 
		Длительное наблюдение показало, что изделия из асбеста 
		пылят волокном, которое при вдыхании может провоцировать заболевание легких — асбестоз.
		Прежде всего в группе риска работники предприятий по добыче и переработке асбеста. 
		В меньшей степени подвержены опасности те, кто ежедневно эксплуатируют изделия из 
		асбеста. В остальных случаях нет причин для паники, если у вас на даче крыша 
		покрыта шифером, а печь в бане прикрыта асбестокартоном, то вы скорее всего умрете 
		не от асбеста, а от заболеваний сердечно-сосудистой системы (статистика\footnote{http://www.medvestnik.ru/content/news/Rosstat-obnarodoval-dannye-o-prichinah-smerti-rossiyan-v-2015-godu.html} смертности).


		\placefigure[here]{Кусок асбестокартона и старый грязный асбестовый шнур. Асбест 
		на ощупь очень мягкий и не колется как стеклоткани.
		}{\externalfigure[asbestos.jpg][width=\textwidth]}

		Асбест и изделия из асбеста до сих пор широко производятся, поскольку в некоторых 
		задачах заменить асбест без потери свойств попросту нечем (или слишком дорого). 
		Асбест отличный материал при конструировании экспериментальных устройств, 
		содержащих нагреватели или раскаленные части. На куске асбестокартона можно 
		спокойно газовой горелкой греть детали до \unit{1000 celsius}, при этом он сохранит свою форму. 
		Асбестовая нить удобна для стягивания нихрома в нагревателях.

		\placefigure[here]{Магнитный усилитель и токовый шунт от блока питания 50-ВУК-120-1 на плате из
		материала на базе асбеста.
		}{\externalfigure[asbestos-plate.jpg][width=\textwidth]}


		Байка (из Википедии):\quote{
		Давно существует легенда о том, как Акинфий Демидов привёз Петру I прекрасную 
		белоснежную скатерть со своего уральского завода. Во время трапезы он 
		демонстративно опрокинул на скатерть тарелку супа, вылил бокал красного вина, а 
		затем скомкал скатерть и бросил её в камин. Затем, достав из огня, показал царю: 
		на ней не осталось ни одного пятнышка. Эта скатерть была сделана из уральского 
		хризотил-асбеста. И в самом деле, демидовские крепостные рабочие достигли 
		совершенства в изготовлении асбестовых тканей. Из них делали ажурные дамские 
		шляпки, перчатки, кошельки, сумочки и кружева. Они не требовали стирки, их кидали 
		в огонь, и через несколько минут после охлаждения их можно было снова носить.} 


	\stopsubject

	%%%%%%%%%%%%%%%%%%%%%%%%%%%%%%%%%%%%%%%%%%%%%%%%%%%%%%%%%%%%%%%%%%%%%%%%%%%%%%%%%%%%%%


	\startsubject[title={Вода}]

		Это абсолютно контринтуитивно, но этот пункт включен сюда, чтобы взорвать вам мозг. 
		Вода не проводит ток! Везде учат, что вода хороший проводник электричества, и 
		обычно это так. Но очень чистая деионизированная вода, которая не содержит ничего 
		кроме \chemical{H_2O} ток не проводит — её удельное сопротивление 18 \unit{Mega ohm centimeter}.
		Та вода, которая проводит ток — недостаточно чистая. Измерение 
		электрической проводимости — довольно простой способ оценки качества и чистоты воды.\footnote{Актуально для постоянного тока и для переменного тока низкой частоты.}

		Имея сильно полярные и подвижные молекулы, вода не только изолятор, но и имеет очень высокую
		диэлектрическую проницаемость — около 81 при комнатной температуре (у большинства обычных
		диэлектриков она не превышает 20–30). На этом основаны емкостные измерители влажности: небольшое
		количество воды между обкладками конденсатора резко повышает его емкость.

		К сожалению, вода — прекрасный растворитель, а растворенные в ней вещества обычно образуют
		электролиты. Стоит постоять дистиллированной воде на воздухе, и она растворяет в себе
		углекислый газ, образуя электролит — слабый раствор угольной кислоты. Вода способна
		растворять и стенки сосуда, в котором находится. Малейшая примесь солей, особенно хлоридов
		и сульфидов натрия, калия, кальция, резко повышает проводимость воды. Поэтому на практике
		в роли диэлектрика вода никуда не годится.


		\placefigure[here]{Бутылка деионизированной воды из радиомагазина. Печатные платы 
		электронных устройств стоит промывать только дистилированной или деионизированной 
		водой, иначе соли, содержащиеся в воде, могут наделать бед.
		}{\externalfigure[water.jpg][width=.7\textwidth]}

	\stopsubject

	%%%%%%%%%%%%%%%%%%%%%%%%%%%%%%%%%%%%%%%%%%%%%%%%%%%%%%%%%%%%%%%%%%%%%%%%%%%%%%%%%%%%%%

	\startsubject[title={Элегаз}]

		Диэлектрики могут быть газообразными. Сухой воздух — хороший диэлектрик, но в 
		некоторых задачах его электроизоляционные свойства недостаточны. Пример газообразного 
		диэлектрика — гексафторид серы или "элегаз", он тяжелее воздуха и имеет пробивное 
		напряжение в несколько раз выше, чем у воздуха, что позволяет сделать электрическую машину компактнее.
		Кроме того, элегаз обладает дугогасящими свойствами, и при контакте с дугой практически не  деградирует, рекомбинируя обратно.

		Довольно забавный опыт, когда вдохнув гелия голос человека становится выше с 
		элегазом выглядит иначе — голос становится ниже\footnote{https://www.youtube.com/watch?v=winZR7JqxBs}. Другое видео: Пара гелий — гексафторид серы\footnote{https://www.youtube.com/watch?v=FvvSIAqOkIw}.
		Так как элегаз тяжелее воздуха, в  нем может плавать легкая лодка\footnote{https://www.youtube.com/watch?v=1PJTq2xQiQ0}

		
	\stopsubject

\stopchapter

%%%%%%%%%%%%%%%%%%%%%%%%%%%%%%%%%%%%%%%%%%%%%%%%%%%%%%%%%%%%%%%%%%%%%%%%%%%%%%%%%%%%%%

\startchapter[title={Органические диэлектрики полусинтетические}]

	Достаточно сложно провести грань между природными и синтетическими материалами, условно назовем полусинтетическими назовем такие материалы,
	для которых природное сырье было переработано для устранения каких-либо недостатков или изменения формы.

	Для большинства природных материалов можно подобрать замену, порой гораздо лучше оригинала, если бы не стоимость. Инженеры при проектировании
	руководствуются в том числе и стоимостью конечного изделия. Зачем использовать дорогую импортную пленку если бумага с пропиткой тоже сработает,
	но будет дешевле? Снижение цены на синтетические материалы стимулирует к сокращению использования полусинтетических материалов.

	\startsubject[title={Бумага, картон}]

		Различные сорта\footnote{Например: Бумага конденсаторная, Бумага кабельная, 
		Бумага телефонная, Бумага крепированная, различных марок.} бумаги широко 
		использовались в электротехнике, начиная от тонкой конденсаторной бумаги (толщина 
		такой бумаги может быть 1 мкм, ГОСТ 1908-88), в качестве диэлектрика обкладок 
		конденсаторов, заканчивая толстым электрокартоном, из которого изготавливались 
		корпуса катушек у трансформаторов.
		Появление дешевых полимерных пленок практически полностью вытесняет бумагу из 
		основных применений.

		\placefigure[here]{Металло-бумажные конденсаторы и их содержимое — толстый 
		цилиндр из туго свитой бумаги с металлизацией.
		}\startcombination[2*1]
		{\externalfigure[paper_cap.jpg][width=0.45\textwidth]}
		{\externalfigure[paper_cap_open.jpg][width=0.45\textwidth]}
		\stopcombination


		Бумага обладает большим недостатком — гигроскопичностью, втягивая из воздуха 
		воду, электроизоляционные свойства снижаются, поэтому чаще всего её пропитывают 
		воском, трансформаторным маслом и т.~п. На текущий момент вытеснена из 
		множества применений пластиковыми пленками и листами.


		\placefigure[here]{Содержимое другого бумажного конденсатора. Видна 
		парафиново-масляная пропитка. Сам конденсатор изолирован от стенок толстым электрокартоном.
		}{\externalfigure[paper_cap2.jpg][width=\textwidth]}


		Бумага за счет своей волокнистой структуры хорошо поддается пропитке и хорошо 
		удерживает жидкие диэлектрики. Относительным преимуществом бумаги является 
		термостойкость, с ростом температуры бумага не расплавится и не потечет, а только обуглится.


		\placefigure[here]{Трансформатор от микроволновой печи, изоляция обмоток от 
		сердечника сделана из бумаги с последующей пропиткой.
		}{\externalfigure[MOT.jpg][width=\textwidth]}

	\stopsubject
	%%%%%%%%%%%%%%%%%%%%%%%%%%%%%%%%%%%%%%%%%%%%%%%%%%%%%%%%%%%%%%%%%%%%%%%%%%%%%%%%%%%%%%

	\startsubject[title={Шёлк}]

		Под шёлком обычно подразумевается  синтетическое волокно. Чаще всего применяется 
		как дополнительная к основной изоляция.

		\placefigure[here]{Провод марки МГШВ\footnote{МГШВ — Монтажный Гибкий (многопроволочная 
		жила) Шёлковая изоляция, Виниловая изоляция.} имеет две изоляции — из полиэфирной нити и из ПВХ 
		}{\externalfigure[silk_wire_PVC.jpg][width=\textwidth]}

		Также в декоративных целях провод может иметь наружную оплетку из искусственной 
		нити — из нейлона, капрона.

		\placefigure[here]{Катушка обмоточного провода марки ПЭЛШО\footnote{ПЭЛШО — Провод с Эмалевым Лакостойким покрытием и Однослойной Шёлковой обмоткой.}.
		}{\externalfigure[silk_wire.jpg][width=\textwidth]}

	\stopsubject

	%%%%%%%%%%%%%%%%%%%%%%%%%%%%%%%%%%%%%%%%%%%%%%%%%%%%%%%%%%%%%%%%%%%%%%%%%%%%%%%%%%%%%%

	\startsubject[title={Воск, парафин}]

		Используется для пропитки трансформаторов, особенно с бумажной изоляцией. Такой 
		трансформатор при перегрузке начинает "плакать" парафином — это верный знак что 
		его стоит выключить, так как дальше будет хуже.

		Общая тенденция по уменьшению габаритов техники и ростом рабочей температуры 
		заставляет производителей заменять легкоплавкий парафин на синтетические компаунды.

		При отсутствии доступа к полимерным заливочным составам, пропитка изделия 
		парафином — самый доступный способ повышения стойкости изделия к влаге среди 
		самодельщиков. Тепловая и механическая стойкость такой пропитки близки к нулю. Чтобы при пропитке
		внутри не осталось пузырей воздуха, по возможности это лучше делать в вакууме.

	\stopsubject


	%%%%%%%%%%%%%%%%%%%%%%%%%%%%%%%%%%%%%%%%%%%%%%%%%%%%%%%%%%%%%%%%%%%%%%%%%%%%%%%%%%%%%%

	\startsubject[title={Трансформаторное масло}]

		Жидкий диэлектрик. Трансформаторное масло — это масло высокой степени чистоты и 
		низкой вязкости. Используется как диэлектрик и теплоноситель в электрических машинах, трансформаторах.

		Внимание! Трансформаторное масло может содержать достаточно {\bf токсичные 
		присадки}\footnote{В том числе в виде загрязнений, достаточно плохо промытой тары.}, 
		в т.~ч. крайне токсичный полихлорбифенил. Поэтому самое глупое, что может прийти в голову — это 
		использование трансформаторного масла не по назначению — в качестве топлива, 
		смазочного масла и т.д. Старый трансформатор может содержать {\it чистый} полихлорированный бифенил
		(выпускался, например, под названием "совол" или
		"совтол"\footnote{http://www.otrabotka.net/dop/chto_v_transformatore_maslo_ili_yad.html}).
		В настоящее время применение полихлорированных бифенилов в новых трансформаторах
		запрещено, но старые трансформаторы и масляные конденсаторы встречаются повсеместно.

		"Утопить" в трансформаторном масле трансформаторы и прочие компоненты — самый 
		простой способ наладить охлаждение и повысить электрическую прочность при 
		самостоятельной сборке высоковольтных устройств (для питания трансформатора 
		Теслы, рентгеновских трубок и т.д.) Основное преимущество этого способа — масло заполняет
		все промежутки, даже сложной формы, и избавиться от пузырей воздуха значительно легче,
		чем при заливке парафином или эпоксидной смолой.

		Прозрачность масла, его лучшая, по сравнению с воздухом теплопроводность, 
		используется иногда при моддинге ПК — все компоненты компьютера — материнская 
		плата, процессор, блок питания и т.~д. кроме HDD и носителей информацией, 
		устанавливаются в аквариуме и заливаются прозрачным маслом\footnote{Гуглить картинки на "моддинг в масле"}. 

	\stopsubject


	%%%%%%%%%%%%%%%%%%%%%%%%%%%%%%%%%%%%%%%%%%%%%%%%%%%%%%%%%%%%%%%%%%%%%%%%%%%%%%%%%%%%%%

	\startsubject[title={Фанера, ДСП}]
		Широко используемый материал, правда почти не применяемый в электронной технике. 
		Представляет собой шпон дерева (фанера) или опилки (ДСП — древесно-стружечная плита) 
		склеенные фенолформальдегидной смолой, и спрессованные в плиты. Фанера — более 
		прочный материал, чем ДСП. Благодаря взаимо перпендикулярной ориентации направления 
		волокон в слоях, фанера обладает равной прочностью во всех направлениях, что 
		делает фанеру достаточно прочным материалом.
		Раньше из фанеры изготавливали корпуса приборов (старые ламповые телевизоры имели 
		корпус из фанеры/ДСП). Но горючесть, набухание от влаги сыграли свою роль и 
		данный материал более массово не используется. Фанера, МДФ (древесное волокно проклеенное 
		карбамидформальдегидными смолами) до сих пор используются для изготовления 
		корпусов аудио аппаратуры — колонок, усилителей, где важны акустические свойства материала,
		а также, вместе со шпоном ценных пород дерева, для украшения дорогостоящей техники.

		Тем не менее фанера — удобный материал в прототипировании низковольтных устройств, 
		и многие коммерчески успешные на рынке устройства когда-то были кучкой железок,
		собранных на фанерке. С появлением лазерных резаков фанеру стали широко использовать любители
		для изготовления корпусов устройств и несложных механических конструкций, так как
		достаточно просто сделать на компьютере чертеж, по которому резак вырежет все детали
		автоматически.

		В продаже встречается фанера тополя, бука и березы разных сортов. Сорт фанеры обозначается
		двумя цифрами через дробь, например 2/4, что значит, что лист фанеры 2-го сорта с одной стороны
		и 4-го сорта с другой. Высшие сорта — гладкие и без сучков, пригодны для изготовления мебели,
		наклейки шпона. 3-й сорт может иметь крупные сучки, а 4-й — вообще дырки от выпавших сучков,
		что делает его пригодным разве что для заборов. Однако из большого листа плохой фанеры можно
		попытаться выбрать хороший кусок. Для лазерной резки и выпиливания лобзиком используют в основном
		тополь, мягкий и дающий очень гладкую поверхность при шлифовании. Более твердая и дорогая береза
		годится для прочных рам, толстых корпусов, мебели. Листы такой фанеры бывают до \unit{40 mm}
		толщиной и представляют собой по сути готовую столешницу для верстака, надо только скруглить
		кромки. Существует водоупорная фанера (обычно ламинирована в коричневый цвет и с одной стороны часто
		с рифлением, чтобы стоящие на ней предметы не скользили), гибкая фанера и фанера, покрытая
		прямо на заводе шпоном ценных пород дерева. Купить все это можно на специализированных фирмах,
		в обычных магазинах такое продают крайне редко.

	\stopsubject

	\startsubject[title={Резина}]

		Эластичный материал, получаемый вулканизацией каучука. Вообще часто резиной 
		называют любой эластичный материал, не акцентируя на разницу в составе, хотя 
		силиконовая резина от изопреновой отличается довольно сильно.

		До изобретения вулканизации природный каучук был специфическим материалом — 
		липким на жаре, ломким на холоде, непрочным. Открытие вулканизации Гудиером 
		позволило лишить резину природных недостатков. Если в исходное сырьё ввести 
		1–2\% серы, то при нагревании между молекулами каучука образуются мостики через 
		атомы серы, в результате чего резина становится упругой, эластичной. Если 
		ввести много серы (30\%), то мостиков будет так много, что резина станет 
		твёрдой, получится материал под названием эбонит. Регулируя степень 
		вулканизации можно регулировать свойства материала в широких пределах.

		Удивительно, до сих пор около 40\% рынка резинового сырья занимает натуральный каучук,
		и большая часть натурального сырья идет на производство покрышек.

		% http://spendmatters.com/2012/11/19/rubber-both-natural-and-synthetic-prices-falling/
		% Всё еще 40-45% резины на рынке производится из натурального латекса

		% https://www.thomasnet.com/articles/plastics-rubber/rubber-industry-solutions/
		% 68% натурального каучука уходит на покрышки

		% http://www.therubbereconomist.com/NR_SR_substitution.html
		% таки 60% синтетической резины в 2007

		% http://digital.ipcprintservices.com/article/The_Rubber_Economist/2765540/401726/article.html
		% таки 60 но может опуститься до 50%



		\placefigure[here]{Кабель в резиновой изоляции, поликлиновой приводной 
		ремень, уплотнительные кольца — изделия из резины.
		}{\externalfigure[rubber.jpg][width=\textwidth]}

		\startsubsubject[title={Примеры применения}]

		\usagedesc{Изоляция проводов.} Повод в резиновой изоляции обладает рядом 
		преимуществ перед собратом с изоляцией из ПВХ, резина на морозе дубеет не 
		так сильно, не плавится. Нагревательные приборы, не исключающие контакт с 
		питающим проводом, имеют провод из термостойкой изоляции, например, утюги. 
		Удлинитель из провода в резиновой изоляции — хорошее решение для тяжелых 
		условий эксплуатации. К сожалению медные жилы такого провода очень часто 
		окисляются, что осложняет монтаж. Резиновая изоляция обладает невысокой 
		атмосферостойкостью, изоляция кабеля пролежавшего под открытым небом покрывается 
		трещинами и может вызвать короткое замыкание.

		\usagedesc{Уплотнения.} Резиновые колечки, прокладки, манжеты, часто не 
		только обеспечивают герметичность, но и устраняют неприятные люфты и 
		вибрации в изделиях.

		\usagedesc{Приводные ремни.} Гибкие ремни круглого (пассики), квадратного, 
		плоского, клинового, поликлинового сечений, зубчатые... и множество других 
		форм. Предназначены для передачи вращения в разных механизмах, например от 
		вала электродвигателя на реечный привод выдвижения лотка у DVD дисковода.

		\usagedesc{Средства защиты от напряжения.} Резиновые перчатки, боты, 
		коврики — всё для защиты электрика от удара электрическим током.

	\stopsubject

	%%%%%%%%%%%%%%%%%%%%%%%%%%%%%%%%%%%%%%%%%%%%%%%%%%%%%%%%%%%%%%%%%%%%%%%%%%%%%%%%%

	\startsubject[title={Эбонит}]

		\materialdesc{Эбонит} — представляет собой высоко-вулканизированный каучук с 
		большим (до 30\%) содержанием серы, за счет чего, в отличии от привычной 
		резины, обладает твёрдостью. Есть ГОСТ 2748-77 на эбонит.


		\placefigure[here]{Эбонитовый пруток. Я сточил часть прутка для видимости 
		самого материала.
		}{\externalfigure[ebonite.jpg][width=\textwidth]}

		\startsubsubject[title={Примеры применения}]

			\usagedesc{Электроизоляционный материал.} До широкого распространения 
			синтетических пластиков был в ходу, сейчас полностью вытеснен пластиками, 
			превосходящими его по свойствам.

			\usagedesc{Поделочный материал} — рукоятки ножей, мундштуки трубок, декоративные элементы.

			\usagedesc{Демонстрация явления трибоэлектричества} — школьный опыт, когда эбонитовая палочка от натирания
			мехом электризуется и начинает притягивать к себе нарезанные кусочки бумаги считается классическим,
			и описан во многих учебниках. В данном опыте эбонит не обладает какими-либо исключительными свойствами,
			просто исторически так сложилось, что до открытия других полимерных материалов эбонит был
			одним из самых доступных прочных диэлектриков. При демонстрации опыта можно смело заменить эбонитовую палочку
			на палочку из любого другого диэлектрика.

		\stopsubsubject

	\stopsubject

\stopchapter

%%%%%%%%%%%%%%%%%%%%%%%%%%%%%%%%%%%%%%%%%%%%%%%%%%%%%%%%%%%%%%%%%%%%%%%%%%%%%%%%%%%%%%

\startchapter[title={Органические диэлектрики синтетические}]

	Доступные природные материалы использовались широко, но, с развитием техники 
	становилось всё более очевидным, что природные материалы порой полное дерьмо. 
	Большой разброс свойств, подверженность гниению, трудности в добыче — поэтому 
	поиски искусственных заменителей велись и ведутся всё время. Появление 
	синтетических материалов — это революция не только техническая, но и экономическая, 
	политическая. Вам больше не нужны колонии чтобы покрыть свои потребности в резине. 
	Экипировка вашего солдата стала легче в несколько раз. В этом разделе — материалы, 
	полученные «с нуля», а не попытка улучшить природные, как в предыдущем разделе.

	Многие из приведенных материалов являются полимерами — материалами с длинными 
	молекулами, состоящими из простых однотипных кирпичиков — мономеров. Полимеры 
	можно разделить на две большие группы по их поведению при нагреве, это термопласты 
	и реактопласты. Термопласты при нагревании плавятся, реактопласты при нагревании 
	разлагаются. Соответственно гору старых пластиковых игрушек из термопластов можно 
	переплавить в новое изделие, а гору старых изделий из реактопластов так переработать не выйдет.

	Полимер может состоять из чистого мономера, а может также содержать со-полимер, 
	который встраивается в структуру молекулы. Например есть два мономера: А и Б. 
	Молекула полимера из чистого А будет выглядеть так:

	...-А-А-А-А-А-А-А-А-А-А-А-...

	Молекула полимера из сополимеров А и Б может выглядеть так:

	...-А-Б-А-Б-А-Б-А-Б-А-Б-А-Б-...

	Или даже так:

	...-А-А-Б-Б-А-Б-Б-Б-А-А-Б-Б-...

	Введение сополимера позволяет изменить свойства пластмассы. Пример — полистирол 
	и АБС пластик. Полистирол прозрачный хрупкий пластик, введение сополимера 
	акрилонитрила и введение добавки из полибутадиена дает на выходе ударопрочный пластик.

	Иногда, может дополнительно указываться стереорегулярность полимера. Допустим у нас есть 
	мономер -Г-, который может вставать в цепочку полимера "вверх ногами" -L-. Полимер,
	с цепочке которого мономеры стоят как попало называется атактическим:

	...-L-Г-Г-L-Г-L-L-L-Г-Г-L-Г-...

	Если в полимере все несимментричные звенья смотрят в одну сторону, такой полимер называется изотактическим:

	...-L-L-L-L-L-L-L-L-L-L-L-L-...

	Если в полимере они чередуются, то такой полимер называется синдиотактическим:

	...-L-Г-L-Г-L-Г-L-Г-L-Г-L-Г-...

	Обычно, стереорегулярность незначительно влияет на важные для электроники свойства материала, 
	поэтому не указывается.




	%%%%%%%%%%%%%%%%%%%%%%%%%%%%%%%%%%%%%%%%%%%%%%%%%%%%%%%%%%%%%%%%%%%%%%%%%%%%%%%%%

	\startsubsubject[title={Общие свойства полимеров}]

		Полимеры, благодаря своей структуре из длинных молекул, обладают некоторыми общими свойствами,
		которые стоит рассмотреть внимательнее.


		{\bold 1. Полимеры не имеют четкой температуры фазового перехода}, как например металлы.
		Они словно карамель, с ростом температуры размягчаются, превращаясь в вязкую тягучую жидкость.
		Поэтому для полимеров "температура плавления" — это температура, при которой вязкость 
		полимера уже позволяет ему течь, но это не означает, что до этой температуры он
		твёрдый.

		Температура стеклования — это температура, ниже которой полимер из высокоэластичного 
		состояния переходит в стеклообразное состояние, с ростом твердости и хрупкости. Представьте
		себе жевательный мармелад — при комнатной температуре он находится в высокоэластичном состоянии.
		Если его охладить ниже температуры стеклования в морозильной камере, то мармелад
		можно будет разбить, и осколки будут как от стекла.

		Максимальная рабочая температура — температура при которой полимер может работать длительное время,
		без существенных изменений своих свойств. Часто с ростом температуры растет ползучесть полимера,
		поэтому при максимальной рабочей температуре прочностные свойства снижаются.

		Указанные температуры могут отличаться при определении даже для одного и того же образца,
		при различии методик определения.

		{\bold 2. Полимеры подвержены старению и разрушению.} Факторами, ускоряющими процесс старения полимера
		являются радиация, ультрафиолетовое излучение, высокая температура, агрессивная среда.
		Разные полимеры в разной степени подвержены старению, кроме того, различными добавками можно снизить 
		скорость разрушения полимера. Так, нейлоновая стяжка на силиконовом шланге с горячей водой за
		пару лет потеряет эластичность и станет хрупкой, в то время как силиконовый шланг по прежнему
		будет мягким и гибким.

		Лишь очень малое количество пластиков терпят длительный нагрев свыше 100°С — 
		фторопласт-4, каптон, peek, силиконы. Во всех остальных случаях чем выше 
		температура эксплуатации — тем быстрее протекают процессы старения и деструкции 
		в полимере.


		{\bold 3. Полимеры проницаемы для газов и некоторых растворителей.} Молекулы 
		газа очень маленькие\footnote{чем меньше атомная масса, тем меньше размер 
		атома, самый мерзкий в этом плане водород, он даже сквозь металлы 
		протискивается.} поэтому могут постепенно проникать сквозь разветвленную 
		молекулярную сеть пластика. Для предотвращения этого процесса поверхность 
		полимера покрывают слоем металла. Обратите на это внимание при вскрытии 
		упаковки продуктов питания. Металлизация в упаковке служит этой цели — не
		пропустить к продукту кислород. Металлопластиковые трубы содержат слой 
		алюминия с той же целью — помимо армирования не допустить проникновение кислорода в теплоноситель, 
		это вызывает коррозию оборудования (котлов, теплообменников и т.~д.).






	\stopsubsubject



	%%%%%%%%%%%%%%%%%%%%%%%%%%%%%%%%%%%%%%%%%%%%%%%%%%%%%%%%%%%%%%%%%%%%%%%%%%%%%%%%%

	\startsubject[title={Материалы на базе фенолформальдегидных смол}]

		Фенол-формальдегидные смолы, как нетрудно понять из названия — продукт 
		поликонденсации фенола и формальдегида. Молекулы полимера образуют разветвленную 
		трехмерную структуру, что обуславливает механические свойства — твёрдость.

		Ниже рассмотрим только фенол-формальдегидные пластмассы — {\bf фенопласты}. 
		Карбамид-формальдегидные, меламин-формальдегидные пластмассы — {\bf аминопласты}, 
		рассматривать не будем, их базовые свойства идентичны, методы обработки одинаковые, 
		разница лишь в прочности, цвете.

		\placefigure[here]{Химическая структура бакелита (кусочек для примера) Полимеры 
		с такой разветвленной беспорядочной структурой обычно твёрдые и хрупкие.\footnote{Автор рисунка — Dirk Hünniger, взято из Википедии}
		}{\externalfigure[bakelite_chem.png][width=\textwidth]}

		Открыл процесс поликонденсации Лео Бакеланд — американский химик бельгийского 
		происхождения. Он и назвал новый материал, полученный при отверждении смолы — бакелитом. 
		В СССР аналогичный материал назывался "карболит" — от карболовой кислоты, 
		старого названия фенола.

		Примеры использования фенолформальдегидных смол:
		Как самостоятельный материал в чистом виде в качестве клеев, лаков.
		С порошковыми наполнителями (придающими прочность или разбавляющими материал 
		просто для экономии) и без — карболит/бакелит

		С наполнением из стекловолокна в хаотичном порядке — волокниты, например прессматериал АГ-4В

		С наполнением из слоев хлопчатобумажной ткани — Текстолиты

		С наполнением из стеклоткани — Стеклотекстолиты

		С наполнением из слоев проклееной бумаги — Гетинакс

	\stopsubject

	%%%%%%%%%%%%%%%%%%%%%%%%%%%%%%%%%%%%%%%%%%%%%%%%%%%%%%%%%%%%%%%%%%%%%%%%%%%%%%%%%

	\startsubject[title={Карболит (бакелит)}]

		Представляет собой твёрдый термостойкий пластик. Если вы возьмете какое-либо 
		устройство, собранное до 1950 года, то практически все пластиковые детали в нем — это карболит.

		\placefigure[here]{Различные изделия из карболита — коробочка, розетка. Вилка, 
		корпус вольтметра, гнезда, ручки регулировки.
		}{\externalfigure[karbolit.jpg][width=\textwidth]}

		Изделия получают как заливкой в формы, так и (чаще) прессованием порошка смолы с 
		наполнителем в металлические формы с нагревом. При нагревании процесс полимеризации, 
		уже частично начавшейся при производстве порошка, заканчивается, но, так как 
		порошок в этот момент зажат под давлением в форме — то и вид конечного изделия 
		повторяет форму. Серьезный недостаток такого метода в том, что нужно время, 
		которое должно провести изделие в форме, чтобы набрать прочность, достаточную 
		для раскрытия формы без разрушения, поэтому во многих задачах бакелит вытеснен 
		термопластичными материалами, термопластавтомат может производить изделия 
		заданной формы значительно быстрее.

		\placefigure[here]{Корпус электросчетчика сделан из карболита.
		}{\externalfigure[karbolit_case.jpg][width=\textwidth]}

		Немного о процессе расскажет это американское рекламное видео\footnote{https://www.youtube.com/watch?v=GlCFBexBWGU} прошлого века, оцените энтузиазм, с которым говорят о новом материале.

		На сегодняшний день изделия из карболита производятся массово, но он уже не так 
		популярен как раньше, хотя есть задачи, где его заменить чем-либо трудно.



		\startsubsubject[title={Плюшки}]

			\featuredesc{Термостойкий пластик.} Может длительное время работать при 
			температуре до +150°С Является реактопластом — не плавится, а разрушается от 
			нагрева. Так карболитовый патрон для лампы накаливания при перегреве рассыпется, 
			а не стечет к вам на голову.

			\featuredesc{Стойкий к растворителям,} ГСМ\footnote{Горюче-смазочным материалам}. Карболитовые детали без труда 
			работают вблизи двигателя автомобиля, в условиях нагрева, контакта с маслом, бензином.

			\featuredesc{Твёрдый.} Обычно карболитовые детали можно распознать по блестящей 
			поверхности и по твёрдости, ноготь такой пластик не царапает и даже не 
			цепляется. Большие плоские детали почти не гнутся, а при превышении усилия со 
			звуком "хрум" ломаются.

			\featuredesc{Хорошо обрабатывается.} В отличии от многих других пластиков 
			хорошо шлифуется. Если попробовать шлифовать, например, полипропилен, то быстро 
			от нагрева начнет образовываться "борода" из пластика. Карболит же отлично 
			шлифуется и часто можно видеть следы шлифовки по периметру детали — удаление облоя.

			\featuredesc{Отличный внешний вид.} Способность образовывать твёрдую глянцевую 
			поверхность особенно заметна на внешнем виде ретроаппаратуры. 
			Даже в магазине на полке ручки для резисторов из карболита смотрятся солиднее 
			таких же, но из термопластиков. 

		\stopsubsubject

		\startsubsubject[title={Недостатки}]

			\problemdesc{Дороговизна.} Особенность производства в виде прессовки из порошка 
			определяет довольно высокую себестоимость изделий из-за низкой скорости 
			процесса и наличия ручного труда. Изготовление деталей из термопластиков 
			порой в разы дешевле.


			\problemdesc{Хрупкость.} Оборотная сторона твёрдости, при ударах трескается, 
			из него не сделать гибкий шланг, сильфон и т.д.

			\problemdesc{Практически не  подлежит вторичной переработке.} Есть способы, но 
			они не получили широкого распространения.

			\problemdesc{Ограниченная цветовая гамма.} Фенолформальдегидная смола сама по 
			себе коричневого цвета, что затрудняет получение изделий светлых цветов. Этого 
			недостатка лишены, например, меламинформальдегидные смолы из которых делают 
			изделия белого цвета.

		\stopsubsubject

		Замечательный фильм\footnote {https://www.youtube.com/watch?v=NU2OJTmoj6g} 
		40х годов, в котором видно производство фенолформальдегидной смолы, формовка 
		деталей прессованием, получение гетинакса, текстолита, галалита и многое другое.

		
	\stopsubject

	%%%%%%%%%%%%%%%%%%%%%%%%%%%%%%%%%%%%%%%%%%%%%%%%%%%%%%%%%%%%%%%%%%%%%%%%%%%%%%%%%

	\startsubject[title={Гетинакс}]

		\materialdesc{Гетинакс} — это слоистый пластик, получаемый путем прессования 
		бумаги, пропитанной фенольной или эпоксидной смолой. В англоязычной литературе 
		имеет название FR-2.\footnote{FR — Flame Resistant — огнестойкий} (FR-1, FR-2, FR-3 это всё гетинаксы\footnote {http://www.smtservice.ru/platyi/laminates_pcb.php}, разница только в материале связующего) У нас есть ГОСТ 2718-74 на гетинакс.
		Имеет низкую прочность, но при этом достаточно низкую цену. Является 
		электроизоляционным материалом, изделия из гетинакса можно получать 
		штамповкой, поэтому панели с ламелями, вставки, изоляционные шайбы, 
		держатели контактов иногда изготавливают из гетинакса.

		\startsubsubject[title={Примеры применения}]

			\usagedesc{Материал дешевых односторонних печатных плат.} В задачах, 
			где не требуется высокая надежность и есть возможность обойтись одним 
			проводящим слоем, печатные платы изготавливают из гетинакса. В дешевых 
			электронных китайских игрушках чаще всего гетинаксовые платы. Гетинакс 
			недостаточно прочен для создания надежных переходных отверстий, поэтому 
			двухсторонние и многослойные печатные платы из гетинакса не изготавливаются.


			\placefigure[here]{Различные изделия из гетинакса. Пластина специально 
			была сломана, чтобы показать характерный рисунок на изломе. Гетинаксовый 
			брусок слегка распух справа — результат расщепления слоев при резке.
			}{\externalfigure[FR-2.jpg][width=\textwidth]}

			\placefigure[here]{Ламели подключения обмоток трансформатора сделаны из 
			гетинакса, изолирующая ламели от сердечника подкладка, боковины оправки 
			обмотки — гетинакс.
			}{\externalfigure[FR-2_transformer.jpg][width=\textwidth]}

			\usagedesc{Ламинированный гетинакс (слопласт, слоистый пластик)} — гетинакс 
			с наклеенной декоративной пленкой — материал внутренней отделки автобусов, 
			вагонов поезда, столешниц. Прочный износостойкий трудногорючий материал.

		\stopsubsubject

		\startsubsubject[title={Примечание}]
			Материал непрочный и склонен давать трещины при обработке, требуется особая 
			осторожность при обработке резанием пилами с большим зубом. В силу низкой 
			прочности мало пригоден в качестве конструкционного материала.
		\stopsubsubject

		\startsubsubject[title={Источники}]
			Продается многими компаниями, специализирующимися на электротехнических 
			материалов. Гуглить по «Гетинакс ГОСТ2718-74».
		\stopsubsubject

	\stopsubject

	%%%%%%%%%%%%%%%%%%%%%%%%%%%%%%%%%%%%%%%%%%%%%%%%%%%%%%%%%%%%%%%%%%%%%%%%%%%%%%%%%

	\startsubject[title={Текстолит}]

		Текстолиты — это целый класс композиционных материалов, состоят из 
		прессованной ткани со связующим. Например, хлопчатобумажная ткань 
		пропитанная фенолформальдегидной смолой. Имеет характерный вид — 
		на плоскостях и срезах видно плетение ткани. Обычно коричневого и 
		темно-коричневого цвета. Зарубежом известны под торговыми марками 
		Novotext, Turbax, Resitex, Cerolon, Textolit, Micarta. Материал 
		известен с 30х годов 20 века.

		\startsubsubject[title={Примеры применения}]

			\usagedesc{Как конструкционный материал.} Текстолит прочен и не 
			проводит ток, поэтому используется как материал прокладок, шайб, 
			перегородок, вставок, шестерен и т.~д. При нагревании он не ползет, 
			это выгодно отличает его от термопластичных материалов.

			\usagedesc{Поделочный материал.} Из текстолита часто изготавливают 
			рукоятки ножей, приспособления и оснастку в условиях небольших 
			мастерских. Текстолит хорошо обрабатывается, при этом не впитывает 
			воду, стоек к воздействию горюче-смазочных материалов.

		\stopsubsubject

		\placefigure[here]{Текстолит различных форм — пластины, прутки. 
		Расположение ткани в материале различается — у прутков ткань намотана, 
		а не уложена слоями.
		}{\externalfigure[textolit.jpg][width=\textwidth]}

		В зависимости от использованной в производстве ткани, наблюдаемая 
		текстура может различаться.


		\placefigure[here]{Текстолит из тканей с разным шагом плетения. 
		Текстолит всегда можно узнать по характерной текстуре и виду.
		}{\externalfigure[textolit_fabric.jpg][width=\textwidth]}

		\placefigure[here]{Шестерня изготовленная из текстолита.
		}{\externalfigure[gear.jpg][width=\textwidth]}

		Материал доступен в продаже в России, но постепенно вытесняется другими материалами. 

	\stopsubject

	%%%%%%%%%%%%%%%%%%%%%%%%%%%%%%%%%%%%%%%%%%%%%%%%%%%%%%%%%%%%%%%%%%%%%%%%%%%%%%%%%

	\startsubject[title={Стеклотекстолит}]
		Разновидность текстолита, в которой используется стеклоткань и 
		чаще всего эпоксидная смола. Обычно светло желтого цвета. Широко 
		распространенный композиционный материал, сочетает в себе легкость, 
		прочность, упругость, не гниет, трудногорюч.

		В виде листов — основной материал печатных плат, имеет за рубежом 
		название FR-4. Достаточно прочный и стабильный для изготовления 
		многослойных печатных плат. Вне формы листов часто имеет в назвнии слово 
		fiberglass — стеклопластик.\footnote{Дословно fiberglass — "стекловолокно", но часто подразумевается именно пластик со стекловолокном.}

		\startsubsubject[title={Примеры применения}]

			\usagedesc{Основной материал для изготовления печатных плат.} 
			Выпускается уже с заранее наклеенной медной фольгой с одной или с 
			двух сторон, именуется "Стеклотекстолит фольгированный ГОСТ 10316-78".

			\placefigure[here]{Заготовки печатных плат из 1,5 мм фольгированного стеклотекстолита.
			}{\externalfigure[FR-4_cutout.jpg][width=\textwidth]}

			\usagedesc{Конструкционный материал.} В виде листов различной толщины 
			"Стеклотекстолит конструкционный ГОСТ 10292-74" Оправки катушек, 
			держатели электродов, корпусные элементы. 

			\placefigure[here]{Обрезки листового стеклотекстолита различной 
			толщины. Деталь на переднем плане специально была сломана — на изломе 
			виден текстильный материал.
			}{\externalfigure[FR-4_pieces.jpg][width=\textwidth]}

			\usagedesc{Электроизоляционный материал.} В качестве прокладок, сепараторов, 
			держателей, защитных пластин.
	
		\stopsubsubject

		\startsubsubject[title={Что стоит добавить}]

			Стеклотекстолит и стеклопластики вообще — очень интересные материалы.
			Стеклопластиковая арматура при падении на бетонный пол звенит, что 
			говорит о высокой упругости материала. Материал легче стали, при этом 
			во многом сопоставим с ней по прочности.

			Материал анизотропен. Так как он слоистый, при нагрузке перпендикулярно 
			слоям он значительно прочнее, чем при нагрузке направленной на разделение 
			слоев. При изготовлении изделий путем формования стеклоткани с эпоксидной 
			смолой это учитывают при выборе направлений укладки ткани.

			Материал обманчиво легко обрабатывается. Легко режется ножовкой, пилится 
			напильником, сверлится. Но стеклянные волокна в составе очень быстро 
			изнашивают рабочую кромку инструмента. Обычные сверла из быстрорежущей 
			стали тупятся уже после двух-трёх десятков отверстий, поэтому стеклотекстолит 
			на производстве печатных плат сверлится твёрдосплавными сверлами из карбида 
			вольфрама. Но сверла эти очень хрупкие и сверлить ими возможно только в станке.

			Материал разрушается необратимо, этим поведением (как и треском при разрыве 
			волокон) сильно напоминает дерево. Если стальная деталь при превышении 
			нагрузки погнется, то её можно при помощи молотка и какой-то там матери 
			выправить обратно, то стеклопластиковая деталь при превышении нагрузки с 
			треском теряет форму и ремонту не подлежит, только замене.

			Стеклотекстолит не гниет, стоек к атмосферным воздействиям (однако может расслаиваться
			из-за замерзания попавшей в случайные поры воды зимой) и прозрачен для 
			радиоволн. Поэтому из него делают различные обтекатели для антенн, кожухи, 
			корпуса. Оборотная сторона медали — материал не подлежит переработке. 
			Вообще, отправляется на свалку и будет лежать там столетиями. 

			Изготовление изделий сложной формы — неавтоматизируемый процесс с обилием 
			ручного труда. Правильно разложить ткань, подрезать, расправить складки, 
			нанести смолу, выгнать пузыри воздуха — всё вручную. Поэтому изделия из 
			стеклопластиков, из карбона (ткань из углеродного волокна) дорогие и 
			вряд ли когда-то станут дешевле. Это собственно и поставило крест на 
			стеклопластиковых кузовах массовых авто — долго, дорого их производить и не 
			переработать в металлолом потом, хотя такой кузов почти вечный и не 
			заржавеет. Некоторые изделия получают автоматизированной намоткой 
			стеклонити — газовые баллоны, арматуру для бетона.

			При обработке стеклотекстолита обязательно одевайте респиратор, частички стекловолокна
			в составе попадая в легкие наносят вред здоровью. Также они достаточно абразивные, поэтому при обработке
			стеклотекстолита на станках примите меры для защиты направляющих.
	
		\stopsubsubject

	\stopsubject

	%%%%%%%%%%%%%%%%%%%%%%%%%%%%%%%%%%%%%%%%%%%%%%%%%%%%%%%%%%%%%%%%%%%%%%%%%%%%%%%%%

	\startsubject[title={Лакоткань}]

		\materialdesc{Лакоткань} — гибкий электроизоляционный материал, состоит из 
		ткани (хлопчатобумажная, синтетическая, стеклоткань) пропитанной эластичным 
		связующим (лаки, смолы). Есть ГОСТ 28034-89 на лакоткани.


		\placefigure[here]{Кусочек лакоткани. Часто можно встретить в трансформаторах.
		}{\externalfigure[lakotkan.jpg][width=\textwidth]}

		Так как слой ткани всего один — лакоткань гибкая и прочная, и зачастую 
		полупрозрачная. Иногда связующее специально делают липким, слои такой ткани 
		хорошо слипаются образуя со временем почти монолитный слой.

		Применяется часто для изоляции слоев обмоток в трансформаторах, в обмотках электромоторов, генераторов.

	\stopsubject

	%%%%%%%%%%%%%%%%%%%%%%%%%%%%%%%%%%%%%%%%%%%%%%%%%%%%%%%%%%%%%%%%%%%%%%%%%%%%%%%%%

	
	\startsubject[title={Полиэтилен}]

		В зависимости от условий синтеза, у молекул полиэтилена может быть разной 
		структура, поэтому отличают:
		Полиэтилен низкой плотности (высокого давления — по условиям синтеза). 
		(LDPE — Low density Polyethylene) молекулярные цепочки имеют много ветвлений
		Полиэтилен высокой плотности (низкого давления — по условиям синтеза). 
		(HDPE — High density Polyethylene) молекулярные цепочки длинные и мало ветвятся.

		Есть и другие варианты, сверхвысокомолекулярный полиэтилен UHMWPE, полиэтилен 
		сверхнизкой плотности ULDPE и так далее.

		Полиэтилен еще может быть "сшитым" PE-X, когда химически или физически, например 
		радиацией, провоцируют создание поперечных химических связей между длинными 
		молекулами полиэтилена. Молекулы, связанные поперечными "мостиками", придают 
		изделиям дополнительную прочность и термостойкость.

		Физические свойства зависят от типа полиэтилена. LDPE более гибкий, сильнее 
		растягивается прежде чем порваться. HDPE жёсткий. Также важны различные 
		добавки и наполнители, которые вводят в полиэтилен, они могут радикально 
		изменить свойства полимера.

		Химические свойства практически одинаковые для всех типов полиэтилена. 

		\startsubsubject[title={Плюшки}]

			\featuredesc{Химически стоек}, кислоты, щелочи, растворители не оказывают на 
			него влияния (за редким исключением). Пластиковая канистра со злобной 
			химией — это полиэтилен. Канистра под топливо — полиэтилен.\footnote {Стойкость прямо зависит от температуры, в нагретых неполярных растворителях вполне себе набухает и растворяется.}

			\featuredesc{Гибкий} — позволяет изготавливать гибкие сильфоны, 
			дозаторы, емкости. Полиэтиленовая канистра вполне может выдержать падение 
			на пол с высоты в пару метров без разрушения. Корпуса автомобильных 
			аккумуляторов иногда делают из полиэтилена (стоек к кислоте и при 
			распухании банок не потрескается). Трубопровод из полиэтилена не 
			боится морозов, если вода в такой трубе замерзнет, то стенки трубы 
			просто растянутся, а не лопнут, как это бывает с металлами.

			\featuredesc{Вязкий.} Полиэтилен, особенно низкой плотности, мягкий и 
			тянется, при этом не склонен легко рваться (нет эффекта расстегивающейся 
			молнии), что позволяет использовать его в броне. Местами может заменять 
			резину, например, различные отбойники — амортизаторы. Строительные каски 
			изготавливают из полиэтилена.

			\featuredesc{Светостоек (только с добавками).} В отличии от других видов 
			полимеров сочетает гибкость с устойчивостью к УФ. Поэтому у проводов для 
			уличного применения ПВХ заменяют на полиэтилен. Разница особенно заметна 
			на морозе, ПВХ дубеет сильнее полиэтилена при низких температурах. Без 
			добавок, увы, разрушается, полиэтиленовая пленка оставленная на улице 
			на третий сезон превращается в труху.

			\featuredesc{Низкая адгезия} — следствие химической стойкости. Это 
			одновременно и плюс и минус. Полиэтилен крайне трудно окрашивать и 
			клеить, требуются специальные ухищрения, обработка поверхности, 
			создание промежуточных слоев. При этом адгезия всё равно крайне низка, 
			такая склейка не может держать высокую нагрузку. По этой же причине 
			окраска полиэтилена обычно производится в массе при изготовлении добавкой 
			красителя в сырьё, а не покраской поверхности.
			Это делает полиэтилен идеальным материалом для изготовления тюбиков
			клея, например застывший "супер клей" в таком тюбике легко счистить 
			с носика. Невозможность прочной склейки определяет основной способ 
			соединения полиэтиленовых деталей — сварка. 

		\stopsubsubject

		Как отличить полиэтилен от других пластиков? 
		При горении пахнет парафином (свечкой). При этом хорошо плавится. 
		Изделия из полиэтилена маркируются знаками\footnote{Знаки переработки 
		нарисованы пользователем Tomina и взяты из Википедии.}:

		\placefigure[here]{Знаки переработки для полиэтилена}
		\startcombination[2*1]
		{\externalfigure[PE_HD.jpg]}{Полиэтилен высокой плотности (низкого давления)}
		{\externalfigure[PE_LD.jpg]}{Полиэтилен низкой плотности (высокого давления)}
		\stopcombination


		Казалось бы — полиэтилен идеальный материал для труб, такие трубы никогда 
		не сгниют, на морозе не полопаются. Собственно из полиэтилена и делают 
		трубы для подачи воды, для канализации. Но только для холодной воды. 
		{\bf Полиэтилен размягчается при 80°С}, плавится уже при 135°С (Примерные 
		значения, зависит от сырья, добавок и модификаций, сшитый PE-X полиэтилен 
		более термостойкий). По этой причине труба из немодифицированного полиэтилена 
		для горячей воды под давлением может раздуться и порваться.


		\placefigure[here]{Дренажная труба из полиэтилена.
		}{\externalfigure[PE-pipe.jpg][width=\textwidth]}

		Тем не менее, трубы из сшитого полиэтилена используют для водопровода, в т.~ч. и 
		горячей воды (вода, поступая от котельной заметно может остыть и далеко не во 
		всех домах развивает хотя бы 65°С). Ограниченно может применяться для 
		отопления\footnote {\hyphenatedurl{http://valtec.ru/catalog/sistemy_metallopolimernyh_i_polimernyh_truboprovodov/truby_iz_sshitogo_polietilena/?yclid=830869231905282620}}.

		\startsubsubject[title={Примеры применения (в электронной технике)}]


			\usagedesc{Изоляция проводов и кабелей.} Полиэтилен с добавкой стабилизаторов 
			и красителей — материал изоляции провода СИП — способного работать на 
			открытом воздухе под солнцем десятки лет.

			\usagedesc{Изоляторы ВЧ разъемов}, материал изоляции внутренней жилы 
			коаксиального кабеля. При работе изоляции с переменным током высокой 
			частоты (более 1 МГц) на первый план выходит ряд специфических характеристик 
			материала, таких как, например, диэлектрическая абсорбция. В итоге то, что 
			хорошо работает на постоянном токе в высокочастотной технике начинает 
			разогреваться, вносить потери. 

			\usagedesc{Корпуса приборов и изделий}, сепараторы, держатели. Различные 
			емкости для жидкостей, трубочки.

			\usagedesc{Упаковочный материал.} Не только в виде пленки, но и в виде 
			листов вспененного полиэтилена.



			\placefigure[here]{Изоляция клеммных колодок сделана из полиэтилена. 
			Старайтесь не использовать дерьмовые\footnote{В них плохо всё, и 
			конструкция, и материал. Не верьте номинальным токам, указанным на 
			упаковке. Хороший клеммник давит на провод плоскостью пластинки, а  не острием 
			винта, и имеет термостойкую изоляцию.} клеммники как на фото, термостойкость 
			изоляции недостаточна и при нагреве изоляция стекает, к тому же горит. 
			Изоляция сердцевины коаксиального кабеля из полиэтилена, наружная чёрная 
			оболочка — из ПВХ.
			}{\externalfigure[PE-wire.jpg][width=\textwidth]}

		\stopsubsubject

	\stopsubject

	%%%%%%%%%%%%%%%%%%%%%%%%%%%%%%%%%%%%%%%%%%%%%%%%%%%%%%%%%%%%%%%%%%%%%%%%%%%%%%%%%

	\startsubject[title={Полипропилен}]

		Полимер похожий на полиэтилен (дополнительный боковой хвостик у молекулы 
		мономера) но с несколько отличными свойствами. Более термостойкий, более 
		жёсткий, менее химически стойкий.

		По прежнему плохо (но уже чуть лучше чем полиэтилен) склеивается и окрашивается.

		Из полипропилена изготавливают трубы для холодной и горячей воды (так 
		как температура плавления полипропилена порядка 170°С то горячую воду 
		такие трубы держат уверено, особенно если имеют армирующий слой, не 
		требуется дополнительных мер по сшивке как у полиэтилена). Трубы соединяют сваркой.

		К сожалению, полипропилен очень похож на полиэтилен высокой плотности 
		как по физическим, так и по химическим свойствам, поэтому надежного 
		способа различить эти два типа полимеров меж собой я не смог найти. 
		Они слегка отличаются по запаху при горении и по температуре размягчения.

		Огромное количество полипропилена расходуется на разного рода 
		упаковку — стаканчики, блистеры и т.~д.

		Прессованное полипропиленовое волокно — материал фильтров, стойких к 
		влаге и агрессивным химическим веществам.


		\placefigure[here]{Фильтрующий картридж из полипропилена для фильтров 
		воды. Полипропиленовые волокна навиты и спрессованы так, что задерживают 
		частицы крупнее 5 мкм.
		}{\externalfigure[PP-filter.jpg][width=\textwidth]}

		Нетканное полотно из полипропиленовых волокон — дешевый заменитель ткани.

		Нетканное полотно — ткань, полученная способом, аналогичным изготовлению 
		бумаги — волокном покрывают ровную поверхность и волокна слипаются между 
		собой в хаотичном порядке. Дополнительно полотно может "прошиваться" 
		спеканием в точках по сетке. Такое полотно менее прочно, чем плетенная 
		ткань, но ЗНАЧИТЕЛЬНО проще в производстве и дешевле. Одноразовая одежда, 
		фильтры, одноразовые влажные салфетки — это всё изделия из нетканного полотна.

		В электронной технике полипропилен используется в виде пленки — изолятора 
		в пленочных конденсаторах.

		Прочность и дешевизна полипропиленовых труб а также простота их соединения 
		позволяет создавать из них прочные объемные конструкции — от фотобокса до 
		двухъярусной детской кровати.

		\placefigure[here]{Различные пленочные конденсаторы. Белый конденсатор на 
		заднем фоне имеет полипропиленовую изоляцию.
		}{\externalfigure[PP-cap.jpg][width=\textwidth]}

	\stopsubject

	%%%%%%%%%%%%%%%%%%%%%%%%%%%%%%%%%%%%%%%%%%%%%%%%%%%%%%%%%%%%%%%%%%%%%%%%%%%%%%%%%

	\startsubject[title={Полистирол, АБС-пластик}]

		Полистирол в чистом виде прозрачный хрупкий пластик.

		Оптический полистирол — один из немногих полимеров, обладающий отличными 
		оптическими качествами и пригодный для изготовления линз, призм и других 
		оптических приборов. Многие другие полимеры, например полиэтилен, 
		полипропилен пропускают свет, но изготовленный из них блок на просвет 
		будет мутным. В сочетании с низким весом и меньшей хрупкостью, по сравнению 
		со стеклом, полностью вытеснил стекло из очков\footnote{Помимо полистирола, 
		очковые линзы изготавливаются из поликарбоната, CR-39 и других оптических полимеров.}. 
		Оптические компоненты бытовой электронной техники — объективы фотоприемников, фонарей, 
		светорассеиватели фотовспышек — изготовляются из оптического полистирола 
		с последующим нанесением покрытий, если требуется. Такая оптика дешевле стеклянной.

		С хрупкостью полистирола борются, вводя в него вязкие эластичные добавки — 
		эластификаторы, например полибутадиен. Модифицированный таким образом 
		полистирол значится как "ударопрочный полистирол" или high-impact polystyrene 
		— HIPS\footnote{В советской литературе АБС относится к разновидности 
		ударопрочных полистиролов, но на практике его выделяют отдельно}.

		Если при производстве к стиролу при полимеризации добавлен сопролимер 
		акрилонитрил, а также бутадиен, то получившийся прочный пластик называется 
		АБС-пластик (Акрилонитрил - Бутадиен - Стирол). Полибутадиен — это резина, в АБС 
		пластике он присутствует в виде мельчайших вкраплений, добавляя прочности и 
		упругости. 

		Вспененный полистирол, пенополистирол мы все помним как "пенопласт", 
		упаковочный материал, теплоизолятор в технике и в строительстве. Сильно 
		горюч, что ограничивает его применение в строительстве.

		Полистирольная пленка используется как диэлектрик в некоторых моделях конденсаторов. HIPS и ABS
		используются только как конструкционные материалы. Специально для активистов 3Д печати стоит отметить,
		хоть ABS и HIPS не проводят электрический ток, изготавливать из них изделия работающие при напряжении
		более 48 Вольт я бы не рекомендовал — слоистая структура отпечатка к сожалению способствует удержанию влаги
		в изделии, что может создавать ощутимые утечки при высоком напряжении.

		% https://www.hardiepolymers.com/knowledge/transparent-abs-can-be-a-clear-winner/
		% АБС не бывает прозрачным. А вот MABS или ABS-M бывает.

	\stopsubject

	%%%%%%%%%%%%%%%%%%%%%%%%%%%%%%%%%%%%%%%%%%%%%%%%%%%%%%%%%%%%%%%%%%%%%%%%%%%%%%%%%

	\startsubject[title={Фторопласт-4 (политетрафторэтилен PTFE)}]

		Уникальный по своим свойствам пластик. Чаще всего молочно белый скользкий 
		пластик. Чистый фторопласт-4 мягкий — царапается ногтем.

		"Клей для фторопласта" стоит на одной полке с философским камнем, святым 
		граалем и другими фантастическими артефактами. Фторопласт настолько химически 
		инертен, что ни в чем не растворяется, даже не набухает. Золото хоть в 
		царской водке растворяется, а фторопласту глубоко плевать на все эти 
		растворители. Как итог — ничем не красится, ничем не клеится.\footnote {Если честно, способ склейки фторопласта существует, но он явно не для каждой мастерской. Подробнее описано тут: http://www.anchem.ru/forum/read.asp?id=5385} 

		Фторопласт — термостойкий полимер, легко выдерживает температуру +250°С. 
		При температурах выше 415°С разлагается. При этом нагреванием фторопласта 
		его можно размягчить, но в вязкотекучее состояние он не переходит, начиная 
		разлагаться, поэтому изделия из фторопласта получают прессованием 
		мелкодисперсного порошка с последующим спеканием.

		В быту чаще всего вы сталкиваетесь с фторопластами под торговой маркой "тефлон" 
		покрытие сковородок антипригарным слоем — это всё фторопласт. (В силу химической 
		инертности фторопласта такие сковороды абсолютно безопасны... если их не 
		перегревать. При перегреве покрытие начинает разрушаться с выделением вредных 
		веществ. Вcе остальные страшилки про PFOA\footnote{PFOA — Perfluorooctanoic acid, 
		перфтороктановая кислота, едкая, токсичная, иногда используется в процессе нанесения покрытий
		из тефлона, разрушается при последующем отжиге изделий. Скандал был связан с отравлением 
		окружающей среды заводом, который сбрасывал PFOA в сточные воды. Следовые количества
		PFOA в готовых изделиях не наносят сколько-нибудь значимого вреда здоровью.} актуальны для работников производств, 
		а не потребителей продукции).

		Фторопласт имеет очень низкое сопротивление скольжения, поэтому фторопласт-4 — 
		хороший материал для подшипников скольжения. Но в чистом виде проявляет 
		склонность к ползучести — под нагрузкой постепенно течет, впрочем, этого недостатка 
		лишены другие фторполимеры.

		Отдельно хочется упомянуть монтажный провод во фторопластовой изоляции — 
		МГТФ\footnote{МГТФ — Монтажный Гибкий Теплостойкий изоляция из Фторопласта.}, 
		белый провод, который часто можно найти внутри военной аппаратуры. У нас его 
		несложно купить, стоит дешево. Если же поискать на ebay "teflon insulated wire" 
		то стоит раза в 3 дороже минимум. Он гибкий, сохраняет гибкость в широком 
		диапазоне температур, не боится кратковременных перегрузок — изоляция не 
		стекает. При пайке изоляция у него не "ползет" от нагрева, что позволяет 
		зачистить кончик в 0,5 мм и припаять к ножке микросхемы в TQFP\footnote{TQFP - Thin Quad Flat Pack, разновидность корпусов микросхем} корпусе без 
		лишних неудобств. К сожалению, в силу особенностей производства изоляции 
		(навивка тонкой пленки фторопласта на жилу) такой провод не подходит для работы во влажной среде.

		\startsubsubject[title={Примеры применения}]

			\usagedesc{Лента ФУМ} (Фторопластовый Уплотнительный Материал) в сантехнике для 
			герметизации резьбовых соединений. Также используется как уплотнительные 
			прокладки шара в шаровых кранах.

			\usagedesc{Диэлектрик в высокочастотных разъемах.} Фторопласт удерживает центральный 
			электрод разъема, в отличии от полиэтилена позволяет не беспокоиться при 
			пайке, что изолятор поплывет от нагрева.

			\placefigure[here]{Высокочастотные разъемы. Изолятор левого изготовлен из 
			полиэтилена, правого — из фторопласта. Корпуса разъемов посеребрены.
			}{\externalfigure[PTFE_CONN.jpg][width=\textwidth]}

			\usagedesc{Изоляция термостойких проводов.} Провод МГТФ — монтажный провод 
			в устройствах авиационного назначения. 

			\placefigure[here]{Моток провода МГТФ сечением 0,35 \unit{square mm}. Характерный 
			розоватый оттенок — медь просвечивает через фторопласт.
			}{\externalfigure[PTFE_WIRE.jpg][width=\textwidth]}

		\stopsubsubject

		\startsubsubject[title={Источники}]

			Фторопласт продается множеством фирм в виде прутков, трубочек 
			(электроизоляционных, поэтому тонкостенных), листов. В крепежных магазинах 
			бывает в виде втулок, шайб.

			Фторопластовая пневматическая трубка пригодна не только как трубка для 
			пневмоустройств в агрессивных средах, но и как вставка в экструдеры 3D 
			принтеров, термостойкость и скользкость фторопласта там подходит идеально.

			Стеклохолст пропитанный фторопластом — продается в хозяйственных магазинах 
			как мат для выпечки, выглядит как тонкий лист ткани желтоватого цвета.
			\footnote{Не путать с силиконовым матом который выглядит как тонкая резина. 
			В описании на коробке должен быть указан политетрафторэтилен (PTFE) или тефлон.} 
			Таким материалом закрыты например нагреватели у запайщиков пакетов — именно 
			благодаря ему пленка не прилипает.

		\stopsubsubject

	\stopsubject

	%%%%%%%%%%%%%%%%%%%%%%%%%%%%%%%%%%%%%%%%%%%%%%%%%%%%%%%%%%%%%%%%%%%%%%%%%%%%%%%%%

	\startsubject[title={Поливинилхлорид — ПВХ}]

		Сам по себе ПВХ жёсткий пластик, но введением в состав пластификатора можно 
		сделать его гибким. Часто в обиходе используется название "Винил". Винипласт - 
		название материала из ПВХ без пластификатора (жёсткий). Выпускается в том числе 
		в виде листов, пленок.

		\placefigure[here]{Тройник, уголок, крепежные скобы для гофроканала, 
		герметичный кабельный ввод — изготовлены из не пластифицированного ПВХ.
		}{\externalfigure[PVC_hard.jpg][width=\textwidth]}

		\startsubsubject[title={Примеры применения}]

			\usagedesc{Изоляция проводов} — достаточно трудно в быту найти провод с 
			изоляцией не из ПВХ.

			\usagedesc{Изолента} — всем известная синяя изолента это ПВХ
			Серая гофра для укладки проводов в строительстве — ПВХ. (чёрная гофра — полиэтиленовая)
			Различные надувные игрушки — ПВХ.

	
		\stopsubsubject

		\startsubsubject[title={Плюшки}]

			\featuredesc{Добавкой антипиретиков горючесть снижается до "не поддерживает 
			горение, самозатухает"}.\footnote{Сам по себе ПВХ без пластификатора не горит, 
			горючесть появляется из-за пластификатора, которую и снижают антипиретиками.} 
			Практически все провода общего назначения имеют изоляцию из ПВХ.

			\featuredesc{Неплохо склеивается}, как специальными клеями для ПВХ, так и 
			цианоакрилатными, полиуретановыми. (Свищ в надувной игрушке из ПВХ неплохо 
			заклеивается полиуретановым клеем).

		\stopsubsubject

		\startsubsubject[title={Минусы}]
			\problemdesc{Не морозостойкий.} При -15°С провода наушников из ПВХ позволяют 
			держать их горизонтально к земле. При -30°С вполне реально могут поломаться. 
			По этой причине кабельные заводы требуют перед размоткой катушек с проводами 
			дать им отлежаться в тепле.

			\problemdesc{Не светостойкий.} ПВХ на солнце разрушается, становится хрупким. 
			Поэтому на улице используются полиэтиленовые (чёрные) гофроканалы, а не ПВХ (серые)

			\placefigure[here]{Оболочка коаксиального кабеля с изоляцией из ПВХ. Кабель 
			для внутренней проводки провисел на улице несколько лет. Изоляция полностью 
			разрушилась.
			}{\externalfigure[PVC_sundestroyed.jpg][width=\textwidth]}

			\problemdesc{При нагревании выделяет едкий ядовитый дым,} содержащий в том 
			числе  HCl (соляную кислоту). Этот дым разъедает оптику, поэтому ПВХ практически не режут на 
			станках лазерного раскроя. Использование ПВХ панелей в отделке катастрофически 
			увеличивает токсичность дыма при пожаре.

			\problemdesc{Миграция пластификатора.} У пластифицированного (мягкого) ПВХ
			пластификатор не вступает в прочную химическую связь с полимером, поэтому со временем
			пластификатор может мигрировать, испаряться из изделия, особенно из 
			приповерхностных слоев. Нагрев, контакт с некоторыми горюче-смазочными веществами 
			и растворителями может ускорять этот процесс. Итогом такой метаморфозы
			является "дубение" изделия, появление трещин. Если планируется длительная работа
			изделия, и требуется эластичность, то стоит посмотреть в сторону эластомеров.
			\footnote{Относительно недавно был скандал как раз связанный с выделением пластификатора из
			кабеля. Спустя некоторое время кабель начинал плакать маслом, но это не чудо,
			а выделение пластификатора из заполнителя кабеля. Гуглить по ключевым словам "кабель NYM потёк".}

		\stopsubsubject

	\stopsubject

	%%%%%%%%%%%%%%%%%%%%%%%%%%%%%%%%%%%%%%%%%%%%%%%%%%%%%%%%%%%%%%%%%%%%%%%%%%%%%%%%%

	\startsubject[title={Полиэтилентерефталат (ПЭТФ)}]

		Другие название этого полимера — полиэстер, ПЭТ, майлар\footnote{Под майларом 
		чаще всего имеют ввиду ПЭТ пленку.}, лавсан\footnote{ЛАВСАН — Лаборатория 
		Высокомолекулярных Соединений Академии Наук} С этим полимером вы сталкиваетесь 
		каждый день — бутылки для воды и напитков получают из него. Волокно из 
		полиэтилентерефталата идет на изготовление флисовой ткани. Это удивительно, но 
		толстовка из флиса и бутылка из под газировки сделаны из одного и того же 
		полимера. Шуршащая прозрачная упаковочная пленка, часто ошибочно называемая 
		целлофаном — это ПЭТФ.

		ПЭТФ обычно прозрачный\footnote{Прозрачный в аморфном и белый в кристаллическом, 
		состояние зависит от скорости охлаждения.} пластик, выпускается в виде листов, 
		преформ для изготовления бутылок, в виде пленки.

		Отличить ПЭТФ от полиэтилена, полипропилена несложно — температура плавления 
		ПЭТФ порядка 250°С, поэтому паяльник разогретый до 200°С не должен вызывать 
		плавления материала. Впрочем, уже при температуре 100°С тару их ПЭТФ может 
		довольно сильно деформировать из-за внутренних напряжений без плавления.

		\startsubsubject[title={Примеры применения}]

			Помимо применений описанных выше используется в качестве диэлектрика в пленочных 
			конденсаторах. «Майларовые» или полиэтилентерефталатные конденсаторы обычно 
			отдельный раздел каталога радиодеталей. Есть довольно интересный старый 
			рекламный фильм\footnote{https://www.youtube.com/watch?v=EHJRHaaTvVU} 
			компании DuPont о майларе.

			\placefigure[here]{Фольговый пленочный конденсатор с изоляцией из 
			полиэтилентерефталатной пленки.
			}{\externalfigure[PET_cap.jpg][width=\textwidth]}

			\placefigure[here]{Пленочные электрические конденсаторы, слева — 
			полипропиленовые, справа — полиэтилентерефталатные. Отличить конденсаторы 
			можно только по маркировке.
			}{\externalfigure[PET_P_cap.jpg][width=\textwidth]}

			Полиэтилентерефталат иногда используется как материал одноразовых печатных плат,
			например для RFID\footnote{RFID — Radio Frequency IDentification, радиочастотная идентификация. RFID
			метка — это устройство, которое при облучении радиоволнами излучает в радиоэфир сигнал, с закодированной
			в ней информацией. Магазинные противокражные метки — частный случай RFID меток.} меток.

			\placefigure[here]{RFID метки, материал основы — полиэтилентерефталат, проводники антенны
			выполнены в виде алюминиевого напыления. В центре — микросхема.
			}{\externalfigure[PET-PCB.jpg][width=\textwidth]}

		\stopsubsubject

		\startsubsubject[title={Источники}]

			В зависимости от потребной толщины пленку из ПЭТФ можно получить:
			\startitemize[intro]

				\item 0,2–0,4 мм — стенки бутылок из под воды, газировки

				\item 0,1 мм — пленка для печати на лазерном принтере (используется для проведения презентаций с обычным проектором)

				\item 0,015 мм — кулинарные пакеты для запекания

				\item 0,012 мм (с металлизацией) — "спасательное одеяло" полотно из ПЭТФ пленки с металлизацией для отражения световых и ИК лучей, входит в состав аптечек.

				\item 0,125–0,08 мм — конверты для ламинирования документов, но имеют нанесенный по всей поверхности клеевой слой.
			\stopitemize


		\stopsubsubject

	\stopsubject

	%%%%%%%%%%%%%%%%%%%%%%%%%%%%%%%%%%%%%%%%%%%%%%%%%%%%%%%%%%%%%%%%%%%%%%%%%%%%%%%%%

	\startsubject[title={Силиконы}]

		Кремнийорганические соединения, коих превеликое множество. Основой полимера 
		является скелет из -Si-O-Si-O- атомов с различными боковыми хвостиками у кремния, 
		в отличие от -C-C-C-C- скелета полиэтилена/полипропилена и т.~д.

		Управляя химическим составом и степенью полимеризации при производстве получают 
		силиконы с различными свойствами — от жидких смазок и жидкостей, заканчивая 
		эластомерами и смолами. Несмотря на это, у силиконов прослеживаются общие свойства.

		\featuredesc{Силиконы химически инертны.} Не настолько, как политетрафторэтилен, 
		но достаточно, чтобы делать из него имплантаты, лить в бытовую химию, 
		добавлять в пищу\footnote{Например пищевая добавка Е900 — Диметилполисилоксан, 
		пеногаситель.}. Из пищевого силикона делаются формочки для выпечки, коврики для 
		выпекания, различную посуду.

		\featuredesc{Низкая адгезия} ко многим материалам. Следствие химической 
		инертности — к силиконам практически ничего не липнет. Это хорошо, если вы в нем 
		готовите, но плохо, если вам нужно приклеить отвалившуюся силиконовую ножку от 
		ноутбука\footnote{Из бытовых клеев хоть как то прилипает к силикону 
		цианоакрилатный (суперклей, жидкий, который мгновенно склеивает пальцы), 
		но всё равно держит плохо.}. Из-за химического сродства хорошо липнет к стеклу.

		\featuredesc{Высокая температурная стабильность.} Силиконовые эластомеры 
		остаются гибкими на лютом морозе и не оплывают при высокой температуре. 
		Некоторые силиконы выдерживают температуру +300°С.

		Силиконовую резину от других видов резин можно отличить если ее сжечь, 
		силикон оставляет белый пепел из диоксида кремния, обычная резина — чёрный 
		пепел из углерода.

		\startsubsubject[title={Примеры применения}]

			\usagedesc{Изоляция проводов.} Как только изоляция из ПВХ вызывает сомнения по 
			нагревостойкости её заменяют на силиконовую. Провода в силиконовой изоляции 
			используются как выводы мощных аккумуляторов с большими пиковыми токами, для 
			подключения ксеноновых ламп, галогеновых ламп. 
			Так получилось, что на постсоветском пространстве, если вам нужен термостойкий 
			тонкий монтажный провод — то проще купить провод МГТФ с фторопластовой 
			изоляцией, чем с силиконовой. Силовые же провода в силиконовой изоляции 
			купить проще, чем монтажные. 

			\placefigure[here]{Провод РКГМ 2,5 — термостойкий провод с изоляцией из 
			кремнийорганической (силиконовой) резины, многожильный с наружной оплеткой 
			из стекловолокна. Рабочая температура -60°С +180°С
			}{\externalfigure[silicone_wire.jpg][width=\textwidth]}

			\usagedesc{Эластичные элементы.} Трубки, демпферы, прокладки, уплотнители и т.~п. 

		\stopsubsubject

		\startsubsubject[title={Источники}]

			Силиконовые герметики, в том числе и термостойкие — в строительных магазинах, 
			в автомобильных магазинах.
			Силиконовый мат для выпекания — отличный материал для вырезания прокладок, мембран.
			Двухкомпонентные силиконовые литьевые составы — пригодны для отливки изделий 
			из силикона, в т.~ч. пищевого назначения — в магазинах для творчества.
			Силиконовые трубочки можно купить в магазинах самогоноварения.

		\stopsubsubject

		\stopsubject

		%%%%%%%%%%%%%%%%%%%%%%%%%%%%%%%%%%%%%%%%%%%%%%%%%%%%%%%%%%%%%%%%%%%%%%%%%%%%%%%%%

		\startsubject[title={Полиимид}]
			Термостойкий гибкий прозрачный полимер желтого цвета\footnote{Полиимиды — целый класс полимеров, но в 
			основном речь именно о каптоне}	в силу созвучности. Иногда фигурирует под торговой маркой «каптон»\footnote{См. 
			раздел «изоленты», там есть каптоновая лента} Держит температуру до +400°С, на 
			холоде не дубеет.

			\startsubsubject[title={Примеры применения}]

				\usagedesc{Термостойкий диэлектрик.} Нагревательный элемент клей-пистолета из 
				керамики наверняка завернут в пленку каптона, для изоляции электродов от корпуса.

				\usagedesc{Материал для изготовления гибких печатных плат.} Часто в электронных 
				устройствах можно встретить гибкие печатные платы, которые изгибаются и соединяют 
				блоки в роли шлейфа, попутно имея на себе припаянные радиоэлементы.

				\placefigure[here]{Разъём, гибкий шлейф, микросхема усилителя — смонтированы на 
				подложке из полиимида.
				}{\externalfigure[flexible_PCB.jpg][width=\textwidth]}

			\stopsubsubject
			
			Полиимидная пленка в целом является безальтернативным материалом для изготовления гибких печатных плат, ближайший "конкурент"
			в виде полиэтилентерефталата не достаточно термостоек, что бы выдержать температуру пайки, из-за этого радиоэлементы к платам из ПЭТФ
			приклеивают, а вот к платам из полиимида можно паять. Полиимидная пленка продается в магазинах	радиотоваров часто под названием "термоскотч",
			так как используется при ремонте электроники для закрытия от фена частей платы, с которых не хотелось бы сдуть элементы при пайке.
			Также полиимидной пленкой покрывают столы для печати у 3D принтеров (актуально для FDM принтеров со столом с подогревом).	

		\stopsubject

	\stopsubject

	%%%%%%%%%%%%%%%%%%%%%%%%%%%%%%%%%%%%%%%%%%%%%%%%%%%%%%%%%%%%%%%%%%%%%%%%%%%%%%%%%

	\startsubject[title={Полиамиды}]

		Еще один класс полимеров. Наверняка вы знакомы с полиамидом-6 и с 
		полиамидом-6.6, но не по химическому названию, а по торговой марке — это капрон и нейлон.

		Полиамиды используются широко, от оболочек некоторых колбас и заканчивая женскими колготками.

		Капрон в виде стержней, листов, блоков имеет название «Капролон», может быть 
		антифрикционным, за счет добавок графита, дисульфида молибдена. Из капролона, 
		к примеру, изготавливаются ходовые гайки механизмов, как дешевая альтернатива бронзе.

		Полиамид с наполнением из стекловолокна — очень прочный материал, из такого 
		пластика изготавливают механически нагруженные детали — детали мебели, шестеренки, корпуса.

		\startsubsubject[title={Примеры применения}]

			\usagedesc{Нейлоновые стяжки} — незаменимая вещь в организации жгутов из проводов, 
			быстром и надежном закреплении всего и вся. 

			\usagedesc{Волокна} — канаты, веревки, бечевка, нитки. В качестве армирующих 
			нитей в некоторых типах кабелей.

			\usagedesc{Ходовые гайки} — дешевая замена бронзе в ходовых гайках станков и 
			механизмов.


			\placefigure[here]{Различные изделия из нейлона — шестерни, стяжки.
			}{\externalfigure[nylon.jpg][width=\textwidth]}

		\stopsubsubject

	\stopsubject


	%%%%%%%%%%%%%%%%%%%%%%%%%%%%%%%%%%%%%%%%%%%%%%%%%%%%%%%%%%%%%%%%%%%%%%%%%%%%%%%%%

	\startsubject[title={Полиметилметакрилат — ПММА}]

		Другие названия — плексиглас, оргстекло, акрил.
		Прозрачный хрупкий пластик. Устойчив к УФ(с добавками), ГСМ.

		Довольно популярный материал среди самодельщиков — режется лазером, фрезеруется. 
		Хорошо формуется в разогретом состоянии, гнется. Прозрачные держатели товаров на 
		витринах, прозрачные полусферы, рельефные световые короба — это всё ПММА.

		\placefigure[here]{Полиметилметакрилат выпускается как прозрачным, так и 
		окрашеным. Стержни на фото  используются как световоды.
		}{\externalfigure[PMMA.jpg][width=\textwidth]}

		Растворяется в дихлорэтане, который часто ошибочно называют "клей для оргстекла", 
		при сгибании лопается, а не белеет в месте сгиба. Запах горящего ПММА ни с чем не спутать.

		Используется в различных световодах, светопрозрачных конструкциях. Низкая 
		пластичность и склонность трескаться ограничивает применение ПММА в задачах, 
		где нужна защита от ударов.

		Наверное самый доступный из прозрачных полимеров, можно купить как в листах, 
		так в и стержнях, блоках. Хорошо склеивается, полируется, обрабатывается. Бывает двух разновидностей,
		в зависимости от метода изготовления - литой и экструзионный. Экструзионный немного дешевле, но обладает 
		б{\it о}льшими внутренними напряжениями, что дает большую склонность к образованию трещин.

	\stopsubject

	%%%%%%%%%%%%%%%%%%%%%%%%%%%%%%%%%%%%%%%%%%%%%%%%%%%%%%%%%%%%%%%%%%%%%%%%%%%%%%%%%

	\startsubject[title={Поликарбонат}]

		Прозрачный прочный пластик. В отличие от ПММА обладает лучшей ударной вязкостью, 
		что делает его предпочтительнее в задачах где нужна прочность, там где 
		поликарбонат выдержит, ПММА покроется трещинами. Поликарбонат прозрачен в видимом  и ближнем ИК диапазоне,
		но поглощает весь ультрафиолет короче 400~нм. Поэтому даже прозрачные строительные очки защищают глаза
		не только от осколков, но и от ультрафиолета (например при работе с источниками УФ излучения, когда под рукой нет специализированной защиты.)
		% http://www.dspolychem.com/new/eng/Content.php?basecode=66

		Не стоек к органическим растворителям, контакт с бензином, маслами может вызвать 
		разрушение и появление трещин.

		\placefigure[here]{Изделия из поликарбоната — защитные очки и компакт диск.
		}{\externalfigure[PC.jpg][width=\textwidth]}

		\startsubsubject[title={Примеры применения}]
			\usagedesc{Компакт диски.} Прозрачная основа диска — поликарбонат.
			Основа оптических линз (чаще всего покрывается защитными слоями, поликарбонат 
			легко царапается).
			Благодаря высокой ударопрочности — различные защитные шлемы, маски, визоры, 
			защитные очки. 
			Сотовый поликарбонат — экструдированные панели из пластика — используются в теплицах.
	
		\stopsubsubject

		\startsubsubject[title={Недостатки}]

			Без добавления специальных присадок разрушается на солнце. Это явление 
			можно видеть к примеру на старых дешевых поликарбонатных теплицах или по помутнению фар старых автомобилей.

			\placefigure[here]{Фары автомобиля. Левая изготовлена из поликарбоната и от воздействия солнца пластик помунтнел. Правая изготовлена из силикатного стекла и устойчива к солнечному свету.
			}{\externalfigure[polycarbonate-sundectruction.jpg][width=\textwidth]}
			
			При резке лазером дает много дыма и обугленный край, поэтому предпочтительнее фрезеровка при обработке.

		\stopsubsubject

	\stopsubject

	%%%%%%%%%%%%%%%%%%%%%%%%%%%%%%%%%%%%%%%%%%%%%%%%%%%%%%%%%%%%%%%%%%%%%%%%%%%%%%%%%
	\startsubject[title={Термоклей}]

		%http://www.dupont.com//content/dam/dupont/products-and-services/packaging-materials-and-solutions/packaging-materials-and-solutions-landing/documents/elvax_thermal_properties.pdf
		Практически в любом магазине для творчества можно приобрести термоклеевой пистолет и стержни с клеем для него.
		Такой пистолет разогреваясь расплавляет стержни из клея, выдавливая его при нажатии на рычаг. Такой клей
		отлично липнет ко всем поверхностям, застывает не меняясь в объёме. 

		Удобство использования, быстрая фиксация, большой объём клея позволяющий заполнять им большие зазоры сделало
		термоклей очень популярным не только в творчестве, но и в производстве не очень качественной электроники. Таким клеем фиксируют
		провода в корпусе, светодиоды в отверстиях, переключатели на своих местах и т.~д. Жаргонное название "китайские сопли" термопластичный клей 
		получил как раз из-за популярности у производителей самых дешевых изделий из китая.

		Производители редко афишируют характеристики термоклея, доступного в строительных магазинах, хозтоварах, зачастую
		единственной характеристикой является диаметр стержня, иногда температура плавления. Но постараемся восполнить этот пробел.

		\placefigure[here]{Пистолет и стержни из клея.
		}{\externalfigure[glue_gun.jpg][width=\textwidth]}

		Стержни клея часто сделаны на базе этиленвинилацетата — смеси двух мономеров, из которых состоят известные нам полиэтилен и 
		поливинилацетат (ПВА) с которым мы сталкиваемся в виде водной дисперсии под названием "клей ПВА". Пропорции мономеров в составе клея
		влияют на его стоимость и на температуру плавления — чем больше доля винилацетата — тем она ниже.

		\startsubsubject[title={Плюшки}]

			\featuredesc{Клей не проводит ток.} Им можно спокойно заливать собранные на весу схемы для придания им хоть какой то прочности.
			\featuredesc{Условно разборное соединение.} Достаточно нагреть феном деталь и клей размягчится настолько, что можно произвести разборку.
			\featuredesc{Можно использовать как герметик.} При создании прототипов может быть удобно залить таким клеем все щели и обеспечить герметичность прибора,
			залив в том числе такие сложные места как ввод проводов.

		\stopsubsubject

		\startsubsubject[title={Минусы}]
			\problemdesc{Горячий.} Как пистолет, так и клей нагревается до 180–\unit{250 celsius} то можно получить ожоги, при этом клей в
			расплавленном состоянии очень липкий, так что быстро стряхнуть его не выйдет.

			\problemdesc{Узкий рабочий температурный диапазон.} Наверное наиболее важный пункт. Клей хорошо работает только при комнатной температуре.
			Примерно при температурах ниже \unit{5 celsius} становится хрупким, а при температурах выше \unit{60 celsius} слишком мягким. Автор практиковал способ
			"разборки" изделий обильно залитых таким клеем в виде замораживания в морозильной камере с последующим разбиванием — термоклей в холодном состоянии 
			легко выкрашивался и отлипал от поверхностей.

			\problemdesc{Не атмосферостойкий.} Не пригоден для использования под открытым небом — даже одного скудного уральского лета достаточно,
			чтобы клей пожелтел, стал хрупким и потрескался.

		\stopsubsubject




	\stopsubject

	%%%%%%%%%%%%%%%%%%%%%%%%%%%%%%%%%%%%%%%%%%%%%%%%%%%%%%%%%%%%%%%%%%%%%%%%%%%%%%%%%



	\startsubject[title={Сравнительная таблица свойств материалов}]

	\setuppagenumber [state=stop]

	\setuptabulate[indenting=yes, distance=small, bodyfont=6pt]
	\placetable[page, 90][tab:compare]{Сравнительная таблица характеристик материалов (конструкционные свойства)}
	\starttabulate[|l|cp(6em)|cp(6em)|cp(6em)|cp(6em)|cp(6em)|cp(6em)|cp(6em)|cp(6em)|]
	\HL
	\NC Название \NC Плот\-ность, \unit{gram per cubic cm} \NC Тем\-пе\-ра\-ту\-ра стек\-ло\-вы\-ва\-ния, \unit{celsius}
	\NC Макс. рабочая температура, \unit{celsius} \NC Температура плавления, \unit{celsius} \NC Склеиваемость, окрашиваемость
	\NC Водопоглощение, \% \NC Теплопроводность, \unit{watt per (meter * kelvin)}\NC Теплоемкость, \unit{kilo joules per (kilogram * kelvin)}
	\NC\SR
	\HL
	%				плотность	темпе стекло	рабочая темп. темп. плавления	склеиваемость	Водопогложение %  теплопроводность % теплоемкость
	\NC Фарфор                    \NC 2,3–2,5    \NC           \NC  1000–1200\NC              \NC    Отлично     \NC     0,1–0,8    \NC     0,25–1,6   \NC  0,7–1,5       \NC\FR
	\NC Оконное стекло            \NC 2,25	     \NC           \NC           \NC              \NC    Отлично     \NC                \NC     0,96       \NC                \NC\MR
	\NC Боросиликатное стекло     \NC 2,23	     \NC           \NC  450      \NC              \NC    Отлично     \NC                \NC     1,05       \NC  0,83          \NC\MR
	\NC Кварцевое стекло          \NC 2,0–2,2    \NC           \NC  1000     \NC              \NC    Отлично     \NC                \NC     1,38       \NC  1,05          \NC\MR
	\NC Слюда                     \NC 2,3–3,0    \NC           \NC  150–750  \NC              \NC    Отлично     \NC    1,3–5,5     \NC     0,46–0,71  \NC  0,8           \NC\MR
	\NC Асбест                    \NC 1,0–3,0    \NC           \NC  400–500  \NC              \NC    Отлично     \NC    10          \NC     0,16       \NC  1,05          \NC\MR
	\NC Алюмооксидная керамика    \NC 3,6–3,9    \NC           \NC  1000–1500\NC              \NC    Отлично     \NC    0,02–0,1    \NC     25-36      \NC                \NC\MR
	\NC Бумага                    \NC 0,7–1,2    \NC           \NC  90       \NC              \NC    Отлично     \NC                \NC     0,05       \NC                \NC\MR
	\NC Парафин                   \NC 0,88–0,91  \NC           \NC  35       \NC  45–65       \NC    Никак       \NC                \NC     0,25       \NC                \NC\MR
	\NC Масло                     \NC 0,89–0,95  \NC           \NC  80–90    \NC              \NC    -           \NC                \NC     0,15       \NC                \NC\MR
	\NC Фанера (дерево)           \NC 0,5–1,0    \NC           \NC  100–120  \NC  не плавится \NC    Отлично     \NC     5–10       \NC     0,14–0,17  \NC  2,3           \NC\MR
	\NC Карболит (бакелит)        \NC 1,25–1,30  \NC           \NC  105–120  \NC  не плавится \NC    Отлично     \NC                \NC                \NC                \NC\MR
	\NC Гетинакс                  \NC 1,35–4     \NC           \NC  185–193  \NC  не плавится \NC    Отлично     \NC                \NC     0,23       \NC  1,4           \NC\MR
	\NC Текстолит                 \NC 1,3–1,45   \NC           \NC  130–140  \NC  не плавится \NC    Отлично     \NC     0,7–0,9    \NC                \NC                \NC\MR
	\NC Стеклотекстолит (FR-4)    \NC 1,6–1,9    \NC           \NC  207–283  \NC  не плавится \NC    Отлично     \NC     0,1        \NC     0,25       \NC                \NC\MR
	\NC Лакоткань                 \NC 	     \NC           \NC  105–180  \NC              \NC    Отлично     \NC                \NC                \NC                \NC\MR
	\NC Натуральная резина        \NC 0,9–1,5    \NC   -40–20  \NC  70       \NC  не плавится \NC    Отлично     \NC                \NC     0,13–0,23  \NC  1,3–1,4       \NC\MR
	\NC Эбонит                    \NC 1,2        \NC           \NC  80       \NC  Не плавится \NC    Отлично     \NC     0,3–1      \NC     0,16–0,17  \NC  1,43          \NC\MR
	\NC Полиэтилен PEHD           \NC 0,94–0,96  \NC   -125    \NC  80–90    \NC  130–140     \NC    Никак       \NC     0,3        \NC     0,35–0,51  \NC  1,9–2,3       \NC\MR
	\NC Полиэтилен PELD           \NC 0,9–0,93   \NC           \NC  80–90    \NC  85–125      \NC    Никак       \NC     0,3        \NC     0,25–0,34  \NC  1,7           \NC\MR
	\NC Полипропилен PP           \NC 0,89–0,9   \NC   -10–20  \NC  100–130  \NC              \NC    Никак       \NC     0,01–0,1   \NC     0,1–0,22   \NC                \NC\MR
	\NC Полистирол                \NC 1,05	     \NC   90–110  \NC  65–90    \NC              \NC    Отлично     \NC     0,01–0,04  \NC     0,1–0,13   \NC                \NC\MR
	\NC Фторопласт-4 PTFE         \NC 2,3	     \NC   -120    \NC  260      \NC  разлагается \NC    Никак       \NC     0,01       \NC     0,25       \NC  1,04          \NC\MR
	\NC Поливинилхлорид PVC       \NC 1,4–1,7    \NC   60–100  \NC  50–80    \NC              \NC    Отлично     \NC     0,04–0,4   \NC     0,15–0,2   \NC                \NC\MR
	\NC Полиэтилентерефталат PET  \NC 1,37–1,45  \NC   70–80   \NC  80–140   \NC  260         \NC    Удовл.      \NC     0,1–0,3    \NC     0,15–0,4   \NC                \NC\MR
	\NC Силиконовые резины	      \NC 1,6–1,7    \NC           \NC  180–250  \NC              \NC    Очень плохо \NC                \NC     0,6        \NC                \NC\MR
	\NC Полиамиды (нейлон, капрон)\NC 1,12–1,15  \NC   40–60   \NC  80–160   \NC  220–265     \NC    Удовл.      \NC     1,6–3,0    \NC     0,25       \NC                \NC\MR
	\NC Полиимиды (каптон)        \NC 1,42–1,65  \NC   250–365 \NC  до 400   \NC              \NC    Хорошо      \NC     1,3–4      \NC     0,1–0,35   \NC                \NC\MR
	\NC Полиметилметакрилат	PMMA  \NC 1,2	     \NC   90–110  \NC  70–90    \NC              \NC    Хорошо      \NC     0,1–0,4    \NC     0,17–0,25  \NC                \NC\MR
	\NC Поликарбонат              \NC 1,19	     \NC   140–150 \NC  100–140  \NC  140         \NC    Хорошо      \NC                \NC     0,19–0,22  \NC                \NC\MR
	\NC Термоклей  (EVA)          \NC 0,92–0,94  \NC           \NC  45–70    \NC              \NC    Удовл.      \NC     0,05–0,13  \NC     0,35       \NC                \NC\MR
	\HL
	\stoptabulate
	% http://www.perkinelmer.com/cmsresources/images/44-74863tch_mptgandstructureofcommonpolymers.pdf
	%https://omnexus.specialchem.com/polymer-properties/properties/water-absorption-24-hours и другие таблицы оттуда
	% http://cncnc.ru/documentation/sprav-constr/html/tom1/pages/chapter2/ckm264.html

	\placetable[90][tab:compare_electric]{Сравнительная таблица характеристик материалов (электрические свойства)}
	\starttabulate[|l|cp(10em)|cp(10em)|cp(10em)|cp(10em)|cp(10em)|]
	\HL
	\NC Название \NC Удельное сопротивление \unit{ohm * mm} \NC Тангенс угла потерь \NC Электрическая абсорбция \NC Диэлектрическая проницаемость \NC Электрическая прочность \unit{kilo volt per mm} \NC\SR
	\HL
	%     Название                 %   сопротивление % тангенс угла %   Сорбция  %  проницаемость  %  Прибивнео напряжение 
	\NC Фарфор                    \NC $10^{11}$         \NC 0,009–0,02     \NC          \NC    5,7–7       \NC     25–30          \NC\FR
	\NC Оконное стекло            \NC                   \NC 0,001          \NC          \NC                \NC     9–13           \NC\MR
	\NC Боросиликатное стекло     \NC                   \NC 0,002          \NC          \NC    4,6–5       \NC     20–40          \NC\MR
	\NC Кварцевое стекло          \NC $10^{20}$         \NC 0,0002–0,002   \NC          \NC    3,8         \NC     25–67          \NC\MR
	\NC Слюда                     \NC                   \NC 0,0003–0,01    \NC 0,3-0,7\%\NC    4-9         \NC     15–118         \NC\MR
	\NC Асбест                    \NC                   \NC 0,7            \NC          \NC    3,1–4,8     \NC                    \NC\MR
	\NC Алюмооксидная керамика    \NC $10^{14}$         \NC 0,0002         \NC          \NC    9           \NC     8–15           \NC\MR
	\NC Бумага                    \NC                   \NC 0,002–0,03     \NC          \NC    2,5–3       \NC     8–40           \NC\MR
	\NC Парафин                   \NC                   \NC 0,0003         \NC          \NC    2,1–2,5     \NC     12–27          \NC\MR
	\NC Масло минеральное         \NC $10^{14}–10^{16}$ \NC 0,0006–0,001   \NC          \NC    2,2–2,4     \NC     6–20           \NC\MR
	\NC Фанера (дерево)           \NC                   \NC                \NC          \NC    1,4–2,9     \NC                    \NC\MR
	\NC Карболит (бакелит)        \NC $10^{11}–10^{12}$ \NC 0,05–0,12      \NC          \NC    4,5–5       \NC     15–20          \NC\MR
	\NC Гетинакс                  \NC $10^{10}–10^{11}$ \NC 0,15–0,4       \NC          \NC    5–6         \NC     25–35          \NC\MR
	\NC Текстолит                 \NC                   \NC 0,05–0,4       \NC          \NC    6–7         \NC     4,5–8          \NC\MR
	\NC Стеклотекстолит (FR-4)    \NC $10^{8}–10^{12}$  \NC 0,001–0,05     \NC          \NC    5–7         \NC     17–25          \NC\MR
	\NC Лакоткань                 \NC $10^{11}$         \NC                \NC          \NC                \NC     24–65          \NC\MR
	\NC Натуральная резина        \NC                   \NC 0,02–0,1       \NC          \NC    3–7         \NC     15–26          \NC\MR
	\NC Эбонит                    \NC $10^{12}–10^{13}$ \NC 0,005–0,08     \NC          \NC    3           \NC     17–30          \NC\MR
	\NC Полиэтилен PEHD           \NC $10^{13}–10^{15}$ \NC 0,03           \NC          \NC    2,1–2,3     \NC     20–30          \NC\MR
	\NC Полиэтилен PELD           \NC $10^{13}–10^{15}$ \NC 0,00035–0,0005 \NC          \NC    2,3         \NC     18–28          \NC\MR
	\NC Полипропилен PP           \NC                   \NC 0,0002–0,0005  \NC 0,05–0,1\%\NC    2,2–2,3     \NC     23–25          \NC\MR
	\NC Полистирол                \NC $10^{14}–10^{15}$ \NC 0,0002         \NC          \NC    2,2–2,8     \NC     16–28          \NC\MR
	\NC Фторопласт-4 PTFE         \NC $10^{15}–10^{18}$ \NC 0,0002         \NC <0,01\%  \NC    2,1         \NC     25–70          \NC\MR
	\NC Поливинилхлорид PVC жесткий\NC                  \NC 0,02–0,06      \NC          \NC    2,9–3,4     \NC     14–40          \NC\MR
	\NC Полиэтилентерефталат PET  \NC $10^{14}–10^{16}$ \NC 0,013–0,015    \NC 0,2–0,5\%\NC    3,0–4,0     \NC     15–60          \NC\MR
	\NC Силиконовые резины	      \NC $10^{12}–10^{14}$ \NC 0,01–0,03      \NC          \NC    3,5–5,0     \NC     18–50          \NC\MR
	\NC Полиамиды (нейлон, капрон)\NC 	            \NC 0,02           \NC          \NC    4,0–5,0     \NC     10–30          \NC\MR
	\NC Полиимиды (каптон)        \NC $10^{15}–10^{16}$ \NC 0,0006–0,007   \NC          \NC    3,1–3,9     \NC     87–180         \NC\MR
	\NC Полиметилметакрилат	PMMA  \NC $10^{11}–10^{12}$ \NC 0,02–0,08      \NC          \NC    2,8–3,8     \NC     10–30          \NC\MR
	\NC Поликарбонат              \NC 	            \NC 0,001–0,009    \NC 0.35\%   \NC    2,8–3,0     \NC     15–34          \NC\MR
	\NC Термоклей   (EVA)         \NC 	            \NC                \NC          \NC    2,5–3,0     \NC     27–28          \NC\MR
	\HL
	\stoptabulate

	%Калинин Н.Н. Электрорадиоматериалы 1981
	%http://polymerdatabase.com/polymer%20physics/Dielectric%20Strength.html
	% https://en.wikipedia.org/wiki/Dielectric_strength
	% по слюде ГОСТ 25045-81 "материалы электроизоляционные на основе щипанной слюды"
	% http://4hv.org/e107_files/public/1490831097_61521_FT179355_dielectric_strength_of_insulating_materials_-crc_handbook_of_chemistry__physics_.pdf
	%https://www.engineeringtoolbox.com/thermal-conductivity-d_429.html
	% http://audioakustika.ru/node/934
	% http://www.csgnetwork.com/dieconstantstable.html
	%http://www.spcable.ru/info/help/fiz_mat.html
	%http://window.edu.ru/resource/078/29078/files/samiit94.pdf
	%https://en.wikipedia.org/wiki/List_of_thermal_conductivities
	% https://www.professionalplastics.com/professionalplastics/ThermalPropertiesofPlasticMaterials.pdf
	%https://www.planetanalog.com/author.asp?section_id=3245&doc_id=563111

	\setuppagenumber [state=start]

	Значения в таблице очень приблизительные, данные для полимеров без наполнителей. Точные значения для конкретного материала можно найти в справочных
	данных производителя. Значения указаны для комнатной температуры.



	\stopsubject




	%%%%%%%%%%%%%%%%%%%%%%%%%%%%%%%%%%%%%%%%%%%%%%%%%%%%%%%%%%%%%%%%%%%%%%%%%%%%%%%%%

	\startsubject[title={График истории промышленного применения полимеров}]

		\placefigure[here]{График истории промышленного применения полимеров
		}{\externalfigure[polymer_history.jpg][width=\textwidth]}

		График появился из любопытства, стало интересно, из чего можно было изготовить 
		изоляцию проводов во время второй мировой войны.\footnote{Удивительно, насколько 
		много времени занял поиск и обработка информации всего лишь для одной картинки. 
		Но, возможно, она получилась единственной в своем роде.} Поискав информацию в 
		интернете и ничего не найдя, пришлось перелопачивать историю по каждому 
		материалу в отдельности. На графике линия начинается в год, когда полимер был 
		презентован как коммерческий продукт, который производится тоннами и его можно 
		купить. Плавное исчезновение линии показывает, что материал потерял популярность 
		и был вытеснен другими материалами.

		Время между открытием материала в лаборатории и его массовым синтезом на заводе 
		различалось от нескольких лет (Нейлон, Бакелит, ПММА) до десятков лет 
		(Полиэтилен, ПВХ). Одно дело, провести каскад реакций в лаборатории и из 
		килограммов сырья получить один грамм материала, и другое дело — наладить 
		быстрый недорогой синтез с хорошим выходом продукта. Кроме того перед 
		производителями стоит проблема «курицы и яйца»: Нет спроса на полимер у 
		производителей, так как нет завода по производству, и, следовательно, надежных 
		поставок. А завода не построено так как нет достаточного спроса на продукт.

	\stopsubject

\stopchapter

%%%%%%%%%%%%%%%%%%%%%%%%%%%%%%%%%%%%%%%%%%%%%%%%%%%%%%%%%%%%%%%%%%%%%%%%%%%%%%%%%

\startchapter[title={Дао изоленты}]

	Когда речь заходит об изоленте — первая мысль, которая приходит в голову, это 
	синяя ПВХ изолента времен СССР. Именно синяя изолента — повод для шуток и 
	написания комментариев, не несущих никакой ценности в некоторых онлайн сообществах. 
	Но изоленты выпускаются не только на базе ПВХ пленки, есть и множество 
	других типов изолент и ниже список самых популярных изолент. Тип изоленты 
	подбирается исходя из задач, условий эксплуатации, требований к надёжности.

	\placefigure[here]{Гора различных видов изоляционных и не очень лент, речь о которых пойдет ниже.
	}{\externalfigure[tapes.jpg][width=\textwidth]}


	\startsubject[title={Прорезиненная тканевая изолента}]

		До сих пор выпускается и продается артефакт советской эпохи — тканевая 
		изолента (ГОСТ 2162-97) — хлопчатобумажное полотно пропитанное резиной. 
		Из достоинств — при нагреве не плавится, а обугливается. На этом её 
		достоинства заканчиваются. Со временем высыхает, теряя клейкость. Пользуется особой 
		любовью у олдфагов\footnote{любителей старины и людей старой закалки.}.

		\placefigure[here]{Тканевая прорезиненная изолента современного производства.
		}{\externalfigure[rubber-fabric-tape.jpg][width=\textwidth]}


		Стоит дополнительно отметить — такая изолента не обеспечивает герметичной сплошной изоляции,
		резиновая пропитка не сплошная и содержит поры. Итогом служит способность накапливать и удерживать влагу,
		электрическая стойкость слоя изоленты эквивалентна такому же слою воздуха.


		\placefigure[here]{Видимые на просвет поры в тканевой изоленте.
		}{\externalfigure[fabric-tape-pores.jpg][width=\textwidth]}

	\stopsubject

	\startsubject[title={Тканевые изоленты}]

		Прорезиненную ХБ ленту упомянутую выше не стоит путать с современными тканевыми изолентами.
		Такие ленты используются не для электрической изоляции, а для крепления и организации проводов.

		Представляют собой ткань или нетканное полотно с нанесенным слоем клея.


		 \placefigure[here]{Лента из плотного синтетического тканного полотна. Видно переплетение нитей, делающее ее прочной. Рядом образец
		шлейфа, собранного в жгут при помощи такой ленты. Высокая гибкость этой ленты позволяет сохранить высокую гибкость жгута, избегая заломов и и пузырей.
		}{\externalfigure[fabric-tape-strong.jpg][width=\textwidth]}

		Разобрав ноутбук наверняка вы найдете что-либо закрепленное такой лентой. Другой разновидностью тканевой ленты является
		рыхлая тканевая лента для жгутования проводов в автомобилях, она обеспечивает целостность жгута, и в отличие от ПВХ изоленты при вибрации не "гремит", ткань
		хорошо гасит звуки от соприкосновений.

		\placefigure[here]{Лента из нетканного прошитого полотна, оно менее прочное, более рыхлое и гораздо дешевле. Рядом кусок автомобильного шлейфа, обмотанного подобной лентой
		}{\externalfigure[fabric-tape-weak.jpg][width=\textwidth]}


		 \placefigure[here]{Макрофотография, показывающая отличие в структуре тканевых лент, описанных выше.
		}{\externalfigure[fabric-tape-compr.jpg][width=\textwidth]}


		Ключевые слова для поиска в интернет-магазинах "harness wiring tape".

	\stopsubject

		\startsubject[title={Резиновые самовулканизирующиеся изоленты}]

		Иногда называются самослипающимися изолентами. Представляют собой толстую (0,5–1 мм) резиновую ленту с защитным слоем. 
		При изоляции соединения удаляется защитная лента, изолента с натягом 
		наматывается внахлест. От контакта с воздухом места нахлеста самосвариваются, 
		образуя монолитный слой. Такую намотку потом не размотать, только разрезать. 

		\placefigure[here]{Рулон самослипающейся (самовулканизирующейся) изоленты. 
		Защитный тонкий слой удаляется во время намотки, а сама лента натягивается, 
		плотно обвивая соединение.
		}{\externalfigure[selfvulkanizing-rubber-tape.jpg][width=\textwidth]}

	\stopsubject

	\startsubject[title={Силиконовые самослипающиеся ленты}]

		Аналогичны резиновым, но из силиконового полимера. Это отечественные ленты ЛЭТСАР (РЭТ-САР - 
		с армирующим слоем из стеклоткани). После намотки внахлест за пару дней 
		крепко слипаются. Силиконы в отличии от резины более термостойки, стойки к 
		химическим воздействиям, на морозе сохраняют эластичность.

		Ключевые слова для поиска в интернет-магазинах "self fusing silicone tape".


		\placefigure[here]{Рулон самослипающейся силиконовой изоленты. 
		Защитный тонкий слой удаляется во время намотки, а сама лента натягивается, 
		плотно обвивая соединение.
		}{\externalfigure[silicone-tape.jpg][width=\textwidth]}


	\stopsubject

	\startsubject[title={Полиимидная лента}]
		Известна также как "термоскотч", "каптон лента".\footnote{каптон, также 
		как и тефлон — зарегистрированная торговая марка, ставшая местами нарицательным.} 
		Желтая прозрачная термостойкая лента, часто можно увидеть в телефонах, 
		ноутбуках — ей закрепляются шлейфы, провода


		\placefigure[here]{Рулон самоклеящейся термостойкой изоляционной ленты.
		}{\externalfigure[kapton-tape.jpg][width=\textwidth]}

		Термостойкая (не плавится паяльником), не тянется, на морозе не дубеет как ПВХ. 
		За пределами электронной техники встречается редко. Используется при ремонте 
		аппаратуры, при пайке феном такой лентой можно заклеить элементы не 
		подлежащие пайке, чтобы не сдуть их случайно. Отменная термостойкость (до \unit{260 celsuis})
		позволяет применять такие ленты вместо слюды для изоляции нагревательных элементов от корпуса.

		Ключевые слова для поиска в интернет-магазинах "polyimide tape".

	\stopsubject

	\startsubject[title={Полиэфирная лента}]

		Изоляционная лента из пленки полиэтилентерефталата (ПЭТ, он же лавсан, он же майлар). Часто используется
		как изоляционная лента в трансформаторах бытового назначения (Часто имеет желтый цвет).

		Такая лента практически не растягивается, имеет термостойкость большую, чем у ПВХ 
		( \unit{130–160 celsuis}, против  \unit{90–105 celsuis}, у ПВХ). Для увеличения 
		прочности может армироваться стекловолокном.

		Ключевые слова для поиска в интернет-магазинах "mylar tape" или "polyester tape".

		\placefigure[here]{Рулон полиэфирной ленты с небольшими трансформаторами, межслойная изоляция которых выполнена
		такой же лентой разных цветов.
		}{\externalfigure[pet-tape.jpg][width=\textwidth]}

	\stopsubject


	\startsubject[title={Стеклотканевая лента с фторопластовой пропиткой}]

		Обладает отличной термостойкостью (до \unit{260 celsuis}), благодаря покрытию из фторопласта
		отлично скользит, к ее поверхности ничего не прилипает, устойчива к агрессивным реагентам. 
		Применяется для покрытия нагревательных поверхностей упаковочных машин, изоляции там, где
		важна химическая устойчивость.


		\placefigure[here]{Рулон стеклоармированной ленты из политетрафторэтилена.
		}{\externalfigure[ptfe-tape.jpg][width=\textwidth]}

		Ключевые слова для поиска в интернет-магазинах "ptfe adhesive tape".

	\stopsubject

	\startsubject[title={Мастичные ленты}]

		Представляют собой ленту из толстого слоя смолы (мастики), применяются для для изоляции,
		набивки, выравнивания изоляции. Толстый слой липкой пластичной мастики позволяет заполнить
		большие зазоры меньшим количеством слоев изоленты, при этом мастика хорошо слипается и
		обеспечивает герметичное соединение без нагрева. Так как мастика не высыхает, то такая
		изоляция не склонна к образованию трещин и щелей. Многим напоминает битумную ленту, используемую в кровельном деле


		\placefigure[here]{Рулон резино-мастичной изоляционной ленты.
		}{\externalfigure[mastic-tape.jpg][width=\textwidth]}

		Ключевые слова для поиска в интернет-магазинах "mastic tape".

	\stopsubject

	\startsubject[title={ПВХ изоленты}]
		Самый распространенный тип изолент. Говорим "изолента" — на уме сразу синяя 
		ПВХ изолента. Лента из пластифицированного ПВХ с клеевым слоем. Не боится влаги, 
		соединение заизолированное такой лентой не боится изгибов. Качество изолент 
		варьируется от "не липнет" у плохо хранившихся старых отечественных, до 
		"одно удовольствие работать" у фирменной продукции 3М или Tesa.

		\placefigure[here]{Разные расцветки изолент.
		}{\externalfigure[PVC-tape.jpg][width=\textwidth]}

		Ключевые слова для поиска в интернет-магазинах "electrical pvc adhesive tape" или просто "electrical tape".

		\startsubsubject[title={Плюшки}]
			Единственная изолента с широкой палитрой цветов: чёрная, зеленая, красная, 
			коричневая, серая... на любой цвет и вкус, а под заказ есть даже розовая, 
			фиолетовая и оранжевая. Также есть полосатая желто-зеленая — всё для цветовой 
			маркировки проводников согласно ПУЭ. И только клинический идиот замотает 
			все соединения желто-зеленой изолентой для PE проводников. 

			Хорошо тянется — есть возможность заизолировать сложные соединения без 
			складок и пузырей.

		\stopsubsubject

		\startsubsubject[title={Недостатки}]

			\problemdesc{Боится нагрева.} При перегреве стекает, правда в след за 
			изоляцией проводов. При длительном небольшом нагреве (\unit{80–100 celsuis}) теряет 
			эластичность и выкрашивается.

			\problemdesc{Боится солнца.} На солнце теряет окраску и эластичность.

			\problemdesc{Ползет.} Это недостаток всех изолент с несохнущим липким слоем. 
			Если изоленту намотать с большим усилием, она постепенно "ползет" по 
			клеевому слою. 

			\placefigure[here]{Деформированные рулоны изоленты в магазине.
			}{\externalfigure[PVC-deformed-tape.jpg][width=\textwidth]}

			\problemdesc{На морозе дубеет.} Недостаток присущий ПВХ, если попытаетесь 
			замотать соединение при температуре наружного воздуха -5°С можете столкнуться 
			с тем что изолента не липнет, а при суровых уральских морозах в -35°С 
			изолента может выпасть из руки и разбиться. Некоторые производители 
			выпускают специальные морозостойкие изоленты, которыми можно работать 
			при низких температурах.

		\stopsubsubject

	\stopsubject

	\startsubject[title={Канцелярская липкая лента «скотч»}]
		Scotch — это торговая марка, но как и термос, ксерокс, акваланг и другие 
		марки стала именем  нарицательным для прозрачной клейкой ленты. Хоть такая 
		липкая лента не используется в электротехнике стоит о ней сказать.

		Изготавливается чаще всего из БОПП — Биаксиально Ориентированного 
		ПолиПропилена. Скотч прозрачен, достаточно прочен и является хорошим 
		диэлектриком. К сожалению, такая клейкая лента склонна к «эффекту 
		расстегивающейся молнии» и наличие порезов позволяет ленте разорваться 
		с минимальным усилием.


		\placefigure[here]{Рулоны канцелярской липкой ленты
		}{\externalfigure[scotch-tape.jpg][width=\textwidth]}
		 
		Один слой скотча выдерживает напряжение порядка 1000В переменного тока, так 
		что если необходимо заизолировать соединение, а под рукой вообще ничего 
		другого нет, то это можно сделать скотчем, хотя качество и надежность такой 
		изоляции будет низкой.

	\stopsubject

	\startsubject[title={Duct tape}]
		Очень популярная зарубежом липкая лента, набирающая популярность у нас. Представляет собой
		полиэтиленовую пленку, покрытую толстым слоем клея и армированную тканью. Отлично липнет,
		легко рвется руками поперек, не боится воды, благодаря армированию тканью 
		не растягивается и обладает хорошей прочностью на разрыв. Именно этой лентой заклеивают
		рот жертве злодеи в голивудских фильмах.


		 \placefigure[here]{Рулон армированной липкой ленты
		}{\externalfigure[duct-tape.jpg][width=\textwidth]}

		Чаще всего применяется для экстренного ремонта вещей, может применяться и для временной
		электрической изоляции, но в установках невысокого напряжения — алюминиевая пудра, 
		придающая серый цвет ленте делает электроизоляционные свойства ленты невысокими. Неплохо горит, что
		для электрической изоляции крайне нежелательно.

	\stopsubject


	\startsubject[title={Проводящие ленты}]
		Иногда могут понадобиться "антиизоленты" — проводящие ленты для задач электромагнитного экранирования.
		Наиболее доступный вариант — алюминиевая самоклеющаяся фольга в виде ленты, 
		используется для заклеивания стыков в тепло- и пароизоляции и продается в строительных магазинах. 
		Иногда может быть армирована стекловолокном. Обладает отличной светостойкостью.


		\placefigure[here]{Рулоны алюминиевой и медной липкой ленты
		}{\externalfigure[al-cu-foil-tape.jpg][width=\textwidth]}

		Также существуют самоклеющиеся ленты из медной фольги. Выбор материала ленты зависит от 
		окружения, и для исключения гальванических пар, лента должна быть из того же материала,
		что и окружающие ее проводники.


		При необходимости обеспечения гибкости экранирующего покрытия может использоваться 
		лента из металлизированной ткани — волокна в такой ленте покрыты тонким слоем металла.


		\placefigure[here]{Рулон металлизированной ткани. Рядом образец шлейфа, экранированного такой лентой.
		}{\externalfigure[shielding-tape.jpg][width=\textwidth]}

		Ключевые слова для поиска в интернет-магазинах "copper shielding tape" если вам нужна медная фольга, 
		и "conductive fabric  shielding tape" если вам нужна металлизированная ткань.

	\stopsubject

\stopchapter

%%%%%%%%%%%%%%%%%%%%%%%%%%%%%%%%%%%%%%%%%%%%%%%%%%%%%%%%%%%%%%%%%%%%%%%%%%%%%%%%%

\startchapter[title={Изоляционные трубки}]

	Иногда использование в качестве диэлектрика трубок предпочтительнее изолент — в 
	силу трудоемкости использования при массовой сборке аппаратуры, для 
	гарантированной герметичности, при трудностях в доступности соединения для 
	намотки изоленты.

	Жаргонное название изоляционных трубок — "кембрик".\footnote{История по ссылке: http://www.gradiant.ru/spravochnik/38-cambric-ethimology}

	\startsubject[title={Трубка из ПВХ}]

		Была широко распространена в СССР для изоляции соединений проводов к клеммам, 
		соединений проводов меж собой, везде, где сейчас используются термоусадочные 
		материалы. 

		\placefigure[here]{Места пайки провода к разъему закрыты кембриками.
		}{\externalfigure[PVC-tube.jpg][width=\textwidth]}

		До сих пор продается и используется. Обладает малой эластичностью, поэтому 
		нужно принимать меры, чтобы трубка не соскочила и не скользила по проводу.

		В отдельных случаях вполне можно использовать в качестве изолирующей трубки 
		различные ПВХ шланги, в т.ч. для компрессоров, аквариумов и т. д. Все 
		преимущества и недостатки поливинилхлорида остаются прежними и описаны в 
		разделе выше посвященном ПВХ.

		\placefigure[here]{ПВХ трубка набухшая в ацетоне рядом с исходной трубкой. Диаметр трубки увеличился почти на треть.
		}{\externalfigure[pvc-acetone.jpg][width=\textwidth]}

		Во время СССР не была так широко доступна термоусадочная трубка, как сейчас, и иногда
		использовался приём, как замачивание ПВХ трубки в ацетоне. ПВХ трубка от ацетона набухает, увеличиваясь в размере.
		Такую трубку натягивали к примеру на ручки пассатижей, бокорезов и другого ручного инструмента. По мере высыхания
		такая трубка скукоживалась, формируя прочное пластиковое покрытие на ручках гораздо эстетичнее, чем просто ручки обмотанные изолентой.
		К сожалению от вымачивания в ацетоне ПВХ теряет пластификатор и становится жестче. Но в отличии от термоусадочной трубки, такое покрытие ручек 
		инструмента не склонно деформироваться и требовать повторного осаждения. (если какую либо активно использующуюся рукоятку затянуть в термоусадку, то со временем от 
		использования термоусадка вытягивается и соскальзывает, что требует повторного нагрева для осаждения.)

	\stopsubject

	\startsubject[title={Фторопластовая трубка}]

		Используется как термостойкая изоляция, особенно в паре с проводом МГТФ. Так 
		же как и фторопласт не любит длительные механические нагрузки, но зато держит 
		довольно высокую температуру. Скользкая и не эластичная.


		\placefigure[here]{Фторопластовая изоляционная трубка. В центре отрезок 
		пневматической фторопластовой трубки.
		}{\externalfigure[PTFE-tube.jpg][width=\textwidth]}

		В механизмах подачи пластика 3Д принтеров, в качестве термобарьера используется 
		фторопластовая трубка для пневмосистем (диаметр наружный 4 мм, внутренний 2 мм). 
		При подачи филамента пластика через такую трубку в экструдер важную роль играет 
		скользкость фторопласта.

	\stopsubject

	\startsubject[title={Армированные трубки}]

		Существуют нескольких видов, в зависимости от материала армирования. \footnote{Подробнее можно
		ознакомиться в ГОСТ 17675-87}.

		Трубка из хлопчатобумажного чулка пропитанного лаком носит названия 
		"линоксиновая трубка"\footnote{тип 110 по ГОСТ 17675-87}., "Трубка изоляционная ТЛВ", "трубка изоляционная лаковая", 
		и имеет характерный желтый цвет с видимой текстурой плетения. 
		Используется для изоляции мест соединений обмоток электрических машин с многожильными проводами.

		Там, где есть постоянный нагрев — чайники, утюги, 
		тепловентиляторы, термопоты и т. д. используется более термостойкий вариант — "стеклоармированная изоляционная трубка". 
		Представляет собой силиконовую трубку, покрытую снаружи стеклотканевым чулком, или сам чулок пропитанный силиконом. 
		Силикон сам по себе термостоек и достаточен для изоляции, но стеклоткань 
		добавляет прочности и препятствует прилипанию. Врочем силиконовые трубки без армирования иногда встречаются в качестве изоляционных.


		\placefigure[here]{Стеклотканево-силиконовые термостойкие изоляционные трубки. 
		Стеклотканевый армирующий слой может быть как внутри трубки, так и снаружи.
		}{\externalfigure[silicone-tube.jpg][width=\textwidth]}

		Стоит отметить, что при использовании таких трубок необходимо принимать во внимание 
		их фиксацию на месте, при неудачном расположении такая трубка может "сьехать" по
		проводу и оголить соединение. 

	\stopsubject

	\startsubject[title={Термоусадочная трубка}]

		Широко используется для изоляции соединений и практически полностью вытеснила 
		со своих позиций ПВХ кембрик, так как удобна в работе.
		Представляет собой полимерную\footnote{Конкретный тип полимера зависит от 
		производителя, к сожалению нельзя однозначно указать что это только полиэтилен 
		к примеру.} трубку с памятью формы — трубка после изготовления растягивается 
		в холодном\footnote{На самом деле температурные режимы при изготовлении 
		несколько сложнее но для удобства будем считать что в холодном состоянии} 
		состоянии, что создает внутренние напряжения. При нагреве до температуры 
		размягчения, (но не плавления) полимер стремится восстановить свою форму и 
		трубка «скукоживается», уменьшаясь в диаметре весьма значительно.
		Термоусадочная трубка плотно обхватывает соединения, гарантируя, что трубка 
		не соскочит и не съедет.

		Выпускается в множестве цветов, диаметров, типов. 

		\placefigure[here]{Различные отрезки термоусадочной трубки.
		}{\externalfigure[heatshrink-tube.jpg][width=\textwidth]}

		Для герметизации сложных соединений, например при разделке силового кабеля, выпускаются
		термоусадочные муфты (полужаргонное название "перчатка", в некоторых каталогах значится именно как "термоусаживаемая перчатка").
		Такая муфта не позволит влаге попадать в зазор между наружной и внутренней изоляцией кабеля.

		\placefigure[here]{Термоусаживаемая муфта для силового кабеля. Муфта изнутри покрыта слоем термоклея для лучшей адгезии и герметичности.
		}{\externalfigure[shrink-cover.jpg][width=\textwidth]}

		Работа с термоусадкой проста — отрезать, надеть, нагреть.

		\placefigure[here]{Паяное соединение, надвигаем подходящую по диаметру 
		термоусадочную трубку и нагреваем. После усадки трубка надежно зафиксирована.
		}{\externalfigure[heatshrink-tube-steps.jpg][width=\textwidth]}

		Для обеспечения герметичного соединения выпускаются термоусадочные трубки с 
		клеевым слоем — они покрыты изнутри слоем клея, который при нагреве и 
		осаживании намертво приклеивает трубку к поверхности провода. Такое соединение 
		может понадобиться, например, при наращивании провода погружного насоса.

		Иногда корпус прибора вообще состоит из одной термоусадки.

		Термоусадку при наличии сноровки вполне реально растянуть на дополнительный 
		миллиметр в диаметре когда подходящей под рукой нет, а та, что есть, не налезает 
		ну совсем чуть чуть.

		Если термоусадочную трубку нагреть сильно и «защипнуть» конец, то она 
		слипнется, таким образом можно заглушить концы кабелей от попадания влаги.

		Нагревать термоусадочную трубку следует феном с регулировкой температуры. При некоторой
		сноровке удается неплохо усаживать трубку газовой горелкой. В нагретом состоянии
		термоусадочная трубка мягкая и склонна рваться, если ей пытаются покрыть острые выступающие
		элементы.

	\stopsubject

\stopchapter

%%%%%%%%%%%%%%%%%%%%%%%%%%%%%%%%%%%%%%%%%%%%%%%%%%%%%%%%%%%%%%%%%


%%%%%%%%%%%%%%%%%%%%%%%%%%%%%%%%%%%%%%%


\startchapter[title={Заключение}]


	Так как установка при написании данного пособия была на минимум брехни, я писал 
	о том, что сам пощупал, использовал, с чем работал. Некоторые темы я не раскрыл, 
	в силу малого опыта\footnote{или малого количества собранного материала} 
	в этих областях, но их стоило бы раскрыть. Переписывать 
	бездумно то, что описано в специализированной литературе я не стал, зачем 
	искажать источник? Поэтому, если вы можете что-то рассказать по теме — я буду 
	рад включить ваш текст и вас в соавторы. Кроме того, проверьте у меня в 
	блоге, возможно вышла новая версия руководства и я раскрыл эти темы.

	Данное руководство распространяется свободно, вы можете скачать самую последнюю версию у меня в блоге 
	по адресу http://serkov.me совершенно бесплатно. Если вам понравилась моя работа, я буду рад услышать от вас 
	пожелания и предложения, а также замечания и указания на допущенные ошибки. 



\stopchapter

%%%%%%%%%%%%%%%%%%%%%%%%%%%%%%%%%%%%%%%%%%%%

\startchapter[title={Список рекомендуемой литературы, сайтов и видосиков.}]

	http://sermir.narod.ru/lec/lect1.htm 
	— курс электротехнического материаловедения от проф. Коробейникова С.М.

	http://polimer1.ru/catalog
	каталог фирмы занимающейся поставкой полимеров, в каталоге очень много полезной информации.

	\hyphenatedurl{http://www-materials.eng.cam.ac.uk/mpsite/interactive_charts/default.html}
	 — Замечательные графики по материалам в координатах прочность/термостойкость, 
	прочность/цена и т. п.

	http://www.barvinsky.ru/guide/guide-materials.htm
	 — Справочник по литью пластмасс. Полезная информация по свойствам различных полимеров.

	http://omnexus.specialchem.com/polymer-properties/properties/max-continuous-service-temperature
	 — таблица с максимальными рабочими температурами для различных полимеров.

	Манфред Беккерт. Мир металла. Издательство "Мир" 1980. — Книга о металлах расчитанная на школьников, но тем не менее очень простым языком и наглядно обьясняет половину университетского курса металловедения.


\stopchapter

%%%%%%%%%%%%%%%%%%%%%%%%%%%%%%%%%%%%%%%%%%%%%%%%%

\startchapter[title={Благодарности.}]

	Выражаю признательность Алексею Gall Галахову за ценные дополнения руководства и помощь в верстке руководства.

	Александру @Talion_amur за предоставленный ртутный счетчик времени наработки.

	Пользователям @Firz, @GavrisAS, @4sadas4, @Leon010203, @rexen, @juray, @Osnovjansky,
	@NickyX3, @impetus, @ploop, @BarsMonster, @OldGrumbler, @YRevich, @Nubus, @jar_ohty,
	@dlinyj, @ioccy, @immaculate, @playnet, @tormozedison, @Samoglas, @Psychosynthesis,
	@Inine, @Serge78rus, @Otard, @Ocelot, @Goruhin чьи комментарии стали причиной правок в руководстве.

	Также хотелось бы поблагодарить Парк Чудес Галилео за помощь в подготовке тиража.
	\placefigure[force]{Логотип спонсора. Подробнее про парк можно прочитать на сайте www.galileopark.ru}{\externalfigure[Galileo_logo.jpg][width=\textwidth]}


\stopchapter

\page[odd]

\stoptext



